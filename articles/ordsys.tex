\documentclass[10pt]{article}
\usepackage{amsfonts}
\usepackage{amsmath}
\usepackage{amssymb}
\usepackage{dina4}
\usepackage{pictex}


% Here are the formatting declarations as required by CUP

\usepackage{fancyheadings}
\setlength{\headrulewidth}{0pt}
\lhead[\textnormal\thepage]{}
\rhead[]{\textnormal\thepage}
\chead[\textit{The Author}]%
{\textit{Ordinal Systems}}
\lfoot{}\cfoot{}\rfoot{}
\addtolength{\headheight}{2.5pt}


\setlength{\textheight}{215mm}
\setlength{\textwidth}{138mm}
\setlength{\parskip}{0pt}
\setlength{\parindent}{16pt}
\pagestyle{fancy}
%
%
%
\bibliographystyle{/opt/tex/lib/tex/macros/alpha} 
%
%
%
%
%
%
%%%%%%%%%%%%%%%%%%%%%%%%%%%%%%%
%                             %
%    Verwaltungsmakros        %
%                             %
%%%%%%%%%%%%%%%%%%%%%%%%%%%%%%%
%
\def\j#1#2{\setcounter{section}{#2}\input{#1}}
\def\n#1#2{{ }}


%
%
%%%%% \laprint is \label and defines \actlabel to be
%%%%% pointing to the label introduced
%%%%% allows to use \refact{a} for \ref{\actlabel{newlabelused}}
%%%% to refer to part a), b), c) ... of a Lemma
%
%
% \laprint erzeugt ein Label, und notiert erzeugt ein
% commando \sublabel das ein label aus dem Hauptlabel und
%  dem jeweiligen Label erzeugt (\laprint {label}) \begin{enumerate}
%  \item\sublabel{c} erzeugt dann ein Label {labelc}
%  das auch ein macro \myreflabelc erzeugt das zu a expandiert
%
% Mit \subref{labelc} wird dann (a) erzeugt, mit
% \subrefcom{label}{c} die Nummer des Lemmas + (a) 
%  Beispiel f\"ur die Anwendung
%
%
%\begin{lemma}
%\laprint{lemma1rhd}
%\begin{enumerate}
%\item\sublabel{a}Text
%\item\sublabel{b}Text
%\end{enumerate}
%\end{lemma}
%\begin{lemma}
%\laprint{ee}
%\begin{enumerate}
%\subitem a Text2
%\subitem {b} Text2
%\end{enumerate}
%\end{lemma}
%Text: \ref{lemma1rhda} \par
%Lemma \ref{lemma1rhdb}, \ref{eea}, \ref{eeb}, Lemma 
%
%
%
\def \laprint#1{{ \label {#1}} \expandafter \global 
    \def \actlabel{#1}}

\def \ssubitem#1{\item \label{\actlabel#1} }
                                                %creates a labeled item
                                                % the label is composed of
                                                % the label of the lemma
                                                % and the parameter
\def \refsub#1#2{\ref{#1#2}}                    % forms the subitems of a 
                                                %Lemma like (a), (b)
\def \refact#1{\ref{\actlabel#1}}               %like \refsub but
                                                %refers to the last defined
                                                %lemma so
                                                %\refact b means (b)
                                                % if in the last lemma
                                                % subitem b was the second
\def \refcom#1#2{\ref{#1}\ \ref{#1#2}}% forms the subitems 
                                      % together with the number of the
                                      % Lemma like 1.2 (a), 1.3 (b)
%
%%%%%%%%%%%%%%%%%%%%%%%%%%%%%%%
%                             %
%    Gr"o"sere Macros         %
%                             %
%%%%%%%%%%%%%%%%%%%%%%%%%%%%%%%
%
%
%
\def \fornum#1#2{{\renewcommand{\theequation}{#2}%
\begin{equation} #1 \end{equation}%
\renewcommand{\theequation}{\arabic{equation}}}}
%%%%%%%%%%%%%%%%%%%%  Macht Formeln mit eigener Nummerierung
%%%%%%%%%%%%%%%%%%%% Bsp \fornum{a+b=c}\ast
\def \fornumeqnarray#1#2{{\renewcommand{\theequation}{#2}
                  \begin{eqnarray} #1 \end{eqnarray}
                  \renewcommand{\theequation}{\arabic{equation}}}}
%%%%%%%%%%%%% fornumeqnarray wie fornum, nur mit eqnarray
%
%
%
\def\mycases#1{\begin{cases}#1\end{cases}}
\def\picture#1{\bigskip \bigskip \bigskip \bigskip \par 
\centerline{!!!! Bild wurde hier ausgelassen}
\bigskip \bigskip \bigskip \bigskip\par }
%\def\picture#1{\noindent\input{#1}\par}
%
%
%
%
%%%%% Boxes
%
%
%
\def\condition#1{\indent \hbox to 2.0cm{$(#1)$\hfill}}
\def\conditionbox#1{\indent \hbox to 2.0cm{#1\hfill}}
%
%
%%%% Environments
%
%
%
\newtheorem {lemma}{Lemma} [section]
\newtheorem {definition} [lemma]{Definition} 
\newtheorem {deflemma} [lemma]{Definition and Lemma} 
\newtheorem {lemdef} [lemma]{Lemma and Definition} 
\newtheorem {corollary} [lemma]{Corollary} 
\newtheorem {theorem} [lemma]{Theorem}
\newtheorem {remark} [lemma] {Remark}
\newtheorem {notation}[lemma]{Notation}
\newtheorem {assumption}[lemma]{Assumption}
\newtheorem {example}[lemma]{Example}
%\newtheorem{assumption}[theorem]{Assumption}{\bfseries}{\rm}
%\newtheorem{deflemma}[theorem]{Assumption}{\bfseries}{\rm}
%\newtheorem{notation}[theorem]{Assumption}{\bfseries}{\rm}
%
%
%%%%%%%%%%%%%%%%%%%%%%%%%%%%%%%%%%%%%%%%%
%                                       %
%    To allow enumerate in the form (a) %
%      erm\"oglichen                    %
%                                       %
%%%%%%%%%%%%%%%%%%%%%%%%%%%%%%%%%%%%%%%%%
%
%
\renewcommand{\labelenumi}{(\alph{enumi})}
\renewcommand{\theenumi}{(\alph{enumi})}
%
%
%
%
%%%% Connectives and similar mathematical symbols
%%%% General ones
%
%
%
\def\Andd{\bigwedge}
\def \prove{\vdash}
\def\all{\forall}
\def\ar{\rightarrow}
\def\longar{\longrightarrow}
\def \Ar{\Rightarrow}
\def \iff{\leftrightarrow}
\def \Iff{\Leftrightarrow}
\def \cross{\times}
%%% calligraphic
\def \powset {{\mathcal P}}      %%%%% Potenzmenge
\def \calA{{\mathcal A}}
\def \calL{{\mathcal L}}
\def \calC{{\mathcal C}}
\def \calO{{\mathcal O}}
\def \calS{{\mathcal S}}
\def \calW{{\mathcal W}}
%
%
%
%
%
%%%% Special symbols
%
%
\def\qedsquare{\hbox{\rlap{$\sqcap$}$\sqcup$}}
\def\QED{\relax\ifmmode\hskip2em \Box\else\unskip\nobreak\hskip1em $\qedsquare$\fi}
%
%
%
%%%%% Natural numbers
%
%
\def\nat{\mathbb{N}}
\def\nathat{\widehat{\mathbb{N}}}
\def\length{\mathrm{length}}
\def\lengthvec{\vec{\mathrm{length}}}
\def\seqlength{\mathrm{seqlength}}
\def\seq#1{\mathopen{<}#1\mathclose{>}}
%
%
%%%% Other basic data structures
%
%
%
\def\potsetfin{{\mathcal{P}^\mathrm{fin}}}
\def\potset{{\mathcal{P}}}
%
%
%
%%%% Ordinal systems
%
%
%
\def\Arg{\mathrm{Arg}}
\def\k{\mathrm{k}}
\def\lrm{\mathrm{l}}
\def\krmhat{\widehat{\mathrm{k}}}
\def\krmhattil{\widehat{\widetilde{\mathrm{k}}}}
\def\lrmhat{\widehat{\mathrm{l}}}
\def\Lhat{\widehat{\mathrm{L}}}
\def\krmtil{\widetilde{\mathrm{k}}}
\def\krmcheck{\check{\mathrm{k}}}
\def\krmvec{\vec{\mathrm{k}}}
\def\OS{{\mathrm{OS}}}
\def\Cl{{\mathcal{C}l}}
\def\lex{\mathrm{lex}}
\def\des{\mathrm{des}}
\def\weakdes{\mathrm{weakdes}}
\def\alle{\mathrm{all}}
\def\Mbb{\mathbb{M}}
\def\calGvec{\vec{\mathcal G}}
\def\calFvec{\vec{\mathcal F}}
\def\eval{\mathrm{eval}}
\def\plusbar{\underline{+}}
\def\alphabar{\underline{\alpha}}
\def\cdotbar{\underline{\cdot}}
\def\omegabar{{\underline{\omega}}}
\def\phibar{{\underline{\varphi}}}
\def\psibar{{\underline{\psi}}}
\def\thetabar{{\underline{\theta}}}
\def\schuette{\mathrm{Sch\ddot{u}tte}}
\def\APN{\mathbb{A}}
\def\CNF{\mathrm{CNF}}
\def\wrm{\mathrm{w}}
\def\zero{\mathrm{zero}}
\def\lesstil{\widetilde{<}}
\def\leqtil{\widetilde{\leq}}
\def\lengthhat{\widehat{\mathrm{length}}}
\def\levelhat{\widehat{\mathrm{level}}}
%
%%%% Orderings
%
%
%
%%%%\def\preccirc{\mathrel{\bigcirc \kern-15.5pt \prec}} %% works for 12pt
\def\preccirc{\mathrel{\bigcirc \kern-12pt \prec}} %% works for 10pt
%
%%%% Ordinal notations
%
\def\C{\mathrm{C}}
\def\NF{\mathrm{NF}}
\def\T{\mathrm{T}}
\def\OT{\mathrm{OT}}
\def\calOT{{\mathcal{OT}}}
\def\Ord{\mathrm{Ord}}
\def\TI{\mathrm{TI}}
\def\Acc{\mathrm{Acc}}
\def\ordinal{{\mathrm{o}}}
\def\M{\mathrm{M}}
\def\Succ{\mathrm{S}}
\def\L{\mathrm{L}}
\def\Lvec{\vec{\mathrm{L}}}
\def\level{\mathrm{level}}
\def\levelvec{\vec{\mathrm{level}}}
\def\B{\mathrm{B}}
\def\I{\mathrm{I}}
\def \segment{\sqsubseteq}
\def \Ag{\mathrm{Ag}}
\def\Car{\mathrm{Car}}
\def\p{\mathrm{p}}
\def\S{\mathrm{S}}
\def\Sigmabar{\underline{\Sigma}}
%
%
%
%%%% Theories
%
%
\def\PRA{\mathrm{PRA}}
\def\ID{\mathrm{ID}}
\def\BI{\mathrm{BI}}
\def\CA{\mathrm{CA}}
\def\KPI{\mathrm{KPI}}
\def\kpomega{\mathrm{KP\omega}}
\def\HA{\mathrm{HA}}
%
%%%% tilde
%
%
%
\def \atil{{\widetilde {a}}}
\def \btil{{\widetilde {b}}}
\def \ctil{{\widetilde {c}}}
\def \dtil{{\widetilde {d}}}
\def \itil{{\widetilde {i}}}
\def \xtil{{\widetilde {x}}}
\def \ytil{{\widetilde {y}}}
\def \ztil{{\widetilde {z}}}
%
% Some objects in \rm
%
%
\def\Below{\mathrm{Below}}
%
%
%
% caligraphic
%
%
\def\calF{\mathcal{F}}
\def\calG{\mathcal{G}}
%
%
%
%  vector
%
%
\def \avec{{\vec a}}
\def \bvec{{\vec b}}
\def \cvec{{\vec c}}
\def \svec{{\vec s}}
\def \tvec{{\vec t}}
\def \xvec{{\vec x}}
\def \yvec{{\vec y}}
\def \zvec{{\vec z}}
\def \Avec{{\vec A}}
\def \Bvec{{\vec B}}
\def \Cvec{{\vec C}}
\def \Dvec{{\vec D}}
\def \Evec{{\vec E}}
\def\gammavec{{\vec \gamma}}
%
%
%%%%% \mathrm 
\def\ck{\mathrm{ck}}
\def\rec{\mathrm{rec}}
%
%
\def\underlinedef#1{{\it #1}}
%
%
%
%
%
%
\begin{document}
\title{Ordinal Systems}
\author{Anton Setzer
\thanks{Department of Mathematics, Uppsala University,
P.O. Box 480, S-751 06 Uppsala, Sweden,
email: {\tt setzer@math.uu.se}}}
\maketitle
\begin{abstract} 
Ordinal systems are structures for describing ordinal notation
systems, which  extend the more predicative approaches to ordinal notation 
systems, 
like the Cantor normal form, the Veblen function and the Sch{\"u}tte 
Klammer symbols, up to the Bachmann-Howard ordinal. $\sigma$-ordinal
systems, which are natural extensions of this approach,
reach without the use of cardinals the strength of the theories
for transfinitely iterated inductive definitions $\ID_{\sigma}$
in an essentially predicative way.  We explore the relationship
with the traditional approach to ordinal notation systems via cardinals 
and determine, using ``extended Sch{\"u}tte Klammer symbols'',
the exact strength of $\sigma$-ordinal systems.
\end{abstract} 


\begin{tabbing} 
\cite{buchholzanewsys}\cite{buchholzschuettebook}
\cite{buchholzasimplified}\cite{pohlersbook}\cite{rathjenweiermann}
\cite{schuettebook}\cite{Schuette54}\cite{seisenbdipl}
\cite{setz93}\cite{setz95a}\cite{setzvenedig}\cite{girardinductivepub}
\cite{girardpionepartone}
\cite{girardbooktwo}\kill
\end{tabbing} 

\setcounter{section}{0}
\section{Introduction}
\laprint{sectintro}
\subsection{Motivation}
The original problem, which motivated the research in this article,
seemed to be a {\em pedagogical} one. We have been trying to teach ordinal 
notation systems above the Bachmann-Howard ordinal several times. 
The impression we had was that we were
able to teach the technical development of these ordinal notation systems,
but there always remained some doubts in the audience. It remained 
unclear, why one could get a well-ordered
notation system by denoting small ordinals by big cardinals.\par  
%The audience was able to understand
%the formal verification, but this formal proof was not satisfactory.
%And we had these problems not only when teaching others but we felt ourselves
%uncomfortable as well.\par 
The situation was completely different with typical ordinal
notation systems below the Bachmann-Howard ordinal. We had the impression
we always succeeded in teaching it after the audience had overcome
some technical problems.  And this included the Sch{\"u}tte Klammer symbols
(an ordinal notation system extending the Veblen hierarchy ---
they will essentially be defined in this article):
Although they are technically more complicated
than the systems using one uncountable cardinal, they seem to be 
far more acceptable. Therefore the reason behind our
pedagogical problems was not a technical one. 
The real problem was about {\em foundations}.\par 
The original task of proof theory as understood by
Hilbert was to show the consistency of systems in which
mathematical reasoning can be formalized. 
After the proof of G{\"o}del's 
incompleteness theorem, one had to modify this and demand the 
reduction of the consistency of a theory to some principles, for which
we have good reasons to believe that  they are  correct. 
One reason, why Gentzen's result was so much appreciated, when it was
presented, was 
that he reduced the consistency of Peano Arithmetic to the principle
of well-ordering
up to $\epsilon_0$, of which we believe intuitively that it is  correct.
The argument presented in Lemma \ref{lemintuitos} is an attempt to 
formalize what we believe is the reason for our confidence with
this principle: the usual notation system for $\epsilon_0$ 
is built from below and has therefore 
an intuitive well-ordering proof.\par 
For the Veblen function and for
the Sch{\"u}tte Klammer symbols the same holds. So, when 
analyzing a theory using these systems 
we have
really gained more than only the reduction of the consistency of theory
to a primitive recursive well-ordering: the reduction to the
well-foundedness of an ordering, of which we intuitively believe that it is
correct.\par 
This leads to another aspect: the relationship to the notion of 
``natural well-ordering''. There exists a trivial ordinal analysis for any
consistent theory. Take as elements of a well-ordering 
essentially pairs consisting of a 
well-ordering proof in the theory and elements of the 
corresponding well-ordering 
and order them lexicographically by
the G{\"o}del-number of the proof and the ordering we are
referring to. (A slight refinement
is necessary in order to make it primitive recursive: replace the
elements of the well-ordering by triples $\seq{a,b,c}$ where
$a$ is an element of the well-ordering, $b$ a calculation that
determines that $a$ is an element of this ordering and
$c$ is a calculation that determines for all $x<a$, where
$<$ is the ordering on the natural numbers, whether $x$ belongs
to the well-ordering and, if yes, the order relation between
$x$ and $a$. Order triples $\seq{a,b,c}$ by the 
ordering of $a$).
The corresponding ordering has as order type the proof theoretic
ordinal of the theory.\footnote{We have heard this example from
Richard Sommer, but do not know of its origins.}
In order to make clear that what one was
doing is not trivial, one usually states that 
one determines the proof theoretic strength not in some arbitrary
primitive  recursive notation system, but in a natural one.
%This and other nonstandard examples, which seem to
%disqualify what  is done in ordinal analysis altogether, have been
%known since long time ago and the
%answer to these examples has always been to say: 
%what we were doing is not only the
%reduction to  some primitive recursive well-ordering, but to some
%natural well-ordering. 
However, nobody succeeded up to now in defining precisely, 
what a natural well-ordering is. But there might be some systematic
reason, why we will never be able to 
formalize, what a natural well-ordering is: if one has a precise
notion, one will probably find a system diagonalizing over it, and this
system can no longer be natural, although it will be in an intuitive sense.\par 
After the considerations before, we suggest that 
one should replace ``natural well-ordering'' by ``ordering with an
intuitive well-ordering proof''.
We believe that the reduction to such well-orderings  is in fact the 
real motivation for designing stronger and stronger ordinal notation systems.
We will see in the following that the usual systems as developed
by the Sch{\"u}tte school (we have not studied ordinal diagrams 
sufficiently yet)
fulfill essentially this requirement.
With a little bit more structure it is easy to see intuitively, why the
system is consistent.

%The focus in proof theory has in the past 
%often been on the use of cardinals or their recursive analogue involved.
%But this made the result of proof theory look 
%rather trivial for non-specialists, because that the proof theoretic 
%strength of  a set theory with say
%one recursive inaccessible can be expressed using one recursive inaccessible
%does not give a lot of information. The real problem was however
%how to set up the structure of  the ordinal notation system,
%and the cardinals involved provided some help for defining it.
%We believe, that this structure in itself has to be put 
%more in the foreground --- which does not mean that development of 
%the corresponding set theoretic ordinal functions should be neglected ---
%and hope that  ordinal systems provide
%a frame work for making the structure more transparent.\par 
%%
%%It has been always very easy for the main
%%reference systems to guess, how the ordinal notation
%%will be written down, even before an ordinal notation system
%%of that strength was defined: what is the strength of Kripke Platek
%%theory with one recursive inaccessible ordinal $\I^\rec$? Of course
%%$\psi(\epsilon_{\I+1})$. And what is the strength if we have
%%one recursive Mahlo ordinal $\M^\rec$? Of course $\psi(\epsilon_{\M +1})$.
%%So the result of proof theoretic analysis looks almost trivial. 
%%But one has completely overlooked, what all this is about.
%%ththe real problem was, how to define an ordinal
%%notation  system that has the strength
%%of the corresponding theory. Such a system captures the  information of the
%%cut elimination procedure. We hope that ordinal systems 
%%provide a framework for making this information more transparent and believe
%%that the structure of ordinal notation systems has to be put more in the 
%%foreground.\par  
In the following we are going to explore three types of (iterated)
ordinal systems: (non-iterated) ordinal systems,
$n$-ordinal systems and $\sigma$-ordinal systems.
For each of these systems we will proceed as follows: First we motivate
and introduce the structure. We will motivate that each of the steps
taken is very natural --- it is not the only natural
way of proceeding but one possible one.
We will then present, what we hope is an intuitive
well-ordering proof, which will be formalized rigorously in a 
theory of the appropriate strength and
provides therefore as well upper bounds for the order types of the structure.
In the case of $\sigma$-ordinal systems the rigorous formalization
is not completed yet and we will omit the argument.
After this intuitive well-ordering proof we will give constructive
well-ordering proofs, which will be far shorter than the
intuitive argument. However, although they
have the advantage that they can be formalized in constructive theories, which
does not hold for the intuitive well-ordering proofs, we personally believe
in the well-ordering of the system essentially because of the intuitive 
well-ordering proof, not the constructive ones.
We do not understand
yet why this is the case, an analysis  of this needs still to be 
done.
It will become clear however that  both proofs
will be closely related.\par 
Next we are going to introduce a sequence of  ordinal notation systems, which
exhausts the strength of the structure. It is no problem to 
develop it in a mere syntactical way. However, we want as well to
develop functions acting on ordinals and not only on notations, and,
when developing first the functions, we get the notation system
almost for free. So we will take the detour via ordinal functions, which
might not be as convincing for those, who like to cut out any ideal concepts
like ordinals, but might be quite satisfactory for those, who want to
compare the approach taken here with the traditional approach.
We will succeed in recovering the functions used in the traditional approach
in our setting. From the development taken, it will be very clear 
that we can introduce the notation system without
referring to ordinals, not even for heuristic purposes, and it would be
a boring task to rewrite these subsections so that no ordinals are used.\par 
{\em Technical advantages:} This approach separates 
properties of ordinal notation systems, which do not have an influence
on the whole strength of it,
from structural properties, which are crucial for achieving the strength.
The well-ordering proofs itself will become easier
(for instance it works in one step, one does not have  first to
verify that the accessible part is closed under $+$, then under $\omega^\cdot$,
then under $\psi$ etc.) and focuses on, what is actually
needed from the ordinal system,
whereas the specific ordinal notation system plays only
a role when verifying that it is an instance of a 
(non-iterated or $n$- or $\sigma$-) ordinal system.
As a side-result we get simultaneously well-ordering proofs in the 
formal theories considered for a complete 
family of notation systems, not only for one specific system.\par  
{\em Background needed:} We will in this article work quite 
lot with
ordinals. However, we believe that one can understand quite a lot of it,
even without having a deep insight. Without the knowledge of 
ordinal notation systems up to $\epsilon_0$ it will be probably
difficult to understand
anything. Knowledge of the Veblen function will be 
useful, but not necessary. We will consider
Sch{\"u}tte Klammer symbols as
examples.
But, who does not know the Veblen function or the Klammer symbols,
can take the equation we state between
those functions and the corresponding
ordinal function generators as the definition. In this case 
we  recommend to skip the versions with fixed points and just consider the
fixed-point free versions of $\varphi$ and the Klammer symbols, which
are in our setting more natural than the usual ones.
The comparison with the $\vartheta$-
and $\psi$-function is only relevant for those, who know those systems.
It will be useful --- but not necessary ---   to read the 
first four sections of \cite{setzvenedig}, especially of the  motivation
given there.\par  
{\em Future developments:} With $\sigma$-ordinal system we have not exhausted the power of 
combinations of ordinal systems, in the meantime we have 
developed ordinal systems which cover the strength of 
one recursive inaccessible and one recursive Mahlo ordinal, and
this is certainly
no the end of what can be done with our approach. For the author, this was 
quite satisfactory, since one can see, that even up to Mahlo we can work
with extensions of Sch{\"u}tte's Klammer symbols in a way which is still
from below. Before carrying out this research,
we  could work with and calculate with the strong
ordinal notation systems, but always had some feeling that what we
were doing, was dubious. The only real argument for the
well-foundedness of the systems was the 
well-ordering proof in a corresponding constructive theory and the
main justification for carrying out ordinal analysis seemed to be to 
provide good tools for reducing the consistency of one theory 
to another where a direct way was impossible. But now we are
really convinced that  one gains something more: the reduction of the 
consistency to the well-ordering of a structure, for which we have
an intuitive well-ordering proof, and which is --- in an
extended sense --- built from below.
%
%
%

\subsection{Notations}
%Gentzen's proof gave some justifcation of the consistency of $\PA$. He
%reduced this consistency to quantifierfree induction over a notation
%system up to $\epsilon_0$. Why does this give now a justification? Because
%we somehow believe in the well-foundedness of the notation system he
%used. And why to we believe in it? Because we think it is 
%impossible to get any infinite descending sequence it. Why do we
%believe there is no infinite descending sequence in it? Now
%his
%
%
%
%%%%%%%%%%%%%%%%%%%%%%%%%%%%%%%%%%%%%%%%%%%
%%
%                                         %
%%%%%%%%%%%%%%%%%%%%%%%%%%%%%%%%%%%%%%%%%%%
%
%
%
\begin{definition}{\rm 
\laprint{notations}
\begin{enumerate} 
\ssubitem a 
As in \cite{setzvenedig}, $A^\ast$ will be the set of
finite sequences of elements of $A$, coded, if $A \subseteq \nat$
as natural numbers, $(\avec)_i$ will be the $i$th element of the
sequence $\avec$ and (a slight change relative to \cite{setzvenedig})
$\seqlength(\avec)$ will be the length of the sequence $\avec$.\\
A class is an object $\{ x \mid \varphi \}$, where $\varphi$ is a 
formula. In this case, after some possibly necessary
$\alpha$-conversion, $a \in \{ x \mid \varphi \} := 
\varphi[x:= a]$. We identify unary predicates $Q$ with the class
$\{ x \mid Q(x) \}$.\\
An \underlinedef{ordering} is a pair $(A, \prec)$ where $A$ is a 
class and $\prec$ is a binary relation on $A$. It is 
\underlinedef{primitive recursive}, if $A$ is a primitive
recursive subset of $\nat$ and $\prec$ is primitive recursive,
and \underlinedef{linear},
if $\prec$ is a linear ordering on $A$.\\ 
If $A = (B, \prec)$,
then \underlinedef{$|A|$}$:= B$,
\underlinedef{$\prec_A$}$:= \prec$.\\
If $C \subseteq |A|$,
we will write $(C,\prec)$ instead of $(C, \prec \cap (B \cross B))$.\\
{\em Transfinite induction over $(A,\prec)$ with respect to the class
$B$}, in short  $\TI_{(A,\prec)}(B)$, is defined as
$\all x \in A(\all y \in A(y \prec x \ar y \in B) \ar x \in B) \ar 
\all x \in A.x \in B$.\\
As in \cite{setzvenedig},
we define $\PRA^+$ as the extension of $\PRA$ by additional
predicates (called free predicates) without having induction over
formulas containing these predicates.
{\it Transfinite induction over $(\prec,A)$ is in ${\mathrm{PRA}}$ reducible to
transfinite induction over $(A_i,\prec_i)$} ($i=1 , \ldots , n$),
in short {\it ${\mathrm{TI}}_{(A,\prec)}$ is ${\mathrm{PRA}}$-reducible to
${\mathrm{TI}}_{(A_i,\prec_i)}$}, if 
there exist $n_i \in \mathbb{N}$, variables $z_{i,j,k}$,
classes $B_{i,j}$ with free variables $\subset 
\{ z_{i,j,1} ,\ldots, z_{i,j,m_{i,j}}\} $, such that
${\mathrm{PRA}^+} \vdash (\bigwedge_{i=1}^n \bigwedge_{j=1}^{n_i}
(\forall z_{ij1} ,\ldots, z_{ijm_{i,j}}.
{\mathrm{TI}}_{(A_i,\prec_i)}(B_{ij} ))
 \rightarrow {\mathrm{TI}}_{(A,\prec)}(Q)$ 
for some free unary predicate $Q$.
\ssubitem {ab} An ordering $(B,\prec)$ is an {\em elementary 
construction from orderings
$(A_1,\prec_1)$ ,\ldots,  $(A_m,\prec_m)$}, 
if $(B,\prec)$ is primitive recursive, linear,
linear provably in $\PRA$ respectively, provided  this property holds for 
$(A_i,\prec_i)$, and,
if, under the condition that $(A_i,\prec_i)$ are primitive recursive
and linear provably in $\PRA$, transfinite induction over
$(B,\prec)$ reduces to transfinite induction over $(A_i,\prec_i)$.
\ssubitem b 
If $f:A \ar B$, $M \subseteq A$, then 
\underlinedef{$f[M]$}$:= \{ f(x) \mid x \in M \} $.\\
\underlinedef{$\potsetfin(A)$} is 
the set of finite sets of elements of $A$ 
coded, if $A \subseteq \nat$, 
as natural numbers, such that the usual properties, especially 
primitive recursiveness and  decidable subset-relation hold.\\
In case $B \in \potsetfin(A)$, we write $t \in B$ for the
statement expressing $t$ is an element of $B$ expressed in
the language of $\PRA$.\\
If $B \in \potsetfin(A)$, $a \in A$, $\prec$ is
a binary relation of $A$, then
\underlinedef{$B \prec a$}$:\Iff \all x \in B.x \prec a$,
\underlinedef{$a \prec B$}$:\Iff \exists x \in B.a \prec x$.
This definitions extends to arbitrary classes $B$ as well.\\
If $\prec$ is a binary relation on $A$, 
\underlinedef{$a \preceq b$}$:\Iff
a \prec b \lor a = b$, similarly we define
$\prec '$ from $\prec$, $\leq$ from $<$ etc.\\
If $k: A \ar \potsetfin(B)$ and $C \subseteq B$, then
\underlinedef{$k^{-1}(C)$}$:= \{ x \in A \mid k(x) \subseteq C \} $.\\
We write \underlinedef{$f: A \ar_\omega B$} for $f: A \ar \potsetfin(B)$.
If $f: A \ar_\omega B$, $f': A \ar_\omega B$, $g:B \ar_\omega C$, 
then \underlinedef{$g \circ f$}$:A \ar_\omega C$, $(g \circ f)(a):= g[f(a)]$,
and $f \subseteq f' :\Iff \all x \in A.f(x) \subseteq f'(x)$.
\ssubitem c
If $A_1 ,\ldots, A_m$ are orderings,
$f: (|A_1| \cross  \cdots \cross |A_m|) \ar M$ injective,
then $f[A_1 ,\ldots, A_m]$ denotes the ordering
$(f[|A_1| \cross \cdots \cross  |A_m|],\prec)$ where
$f(a_1 ,\ldots, a_m) \prec f(b_1 ,\ldots, b_m)$ if
$(a_1 ,\ldots, a_m)$  lexicographically less then
$(b_1 ,\ldots, b_m)$ with respect to the orderings
$A_1 ,\ldots, A_m$. 
We will use this definition only in case where 
$f(a_1 ,\ldots, a_n)$ is the result of substituting
in a term $t$ 
(such as $\varphi_{x_1}{x_2}$, $\psi(x_1)$, $(x_1,x_2,x_3)$, 
$x_1+x_2$, $x_1 \choose x_2$)
variables $x_i$ by $a_i$, i. e. $f = \lambda x_1 ,\ldots, x_n.t$,
and will write in this case the result of replacing in $t$
the variable $x_i$ by $A_i$ instead of $f[A_1 ,\ldots, A_m]$
(i.e. $\varphi_{A_1}A_2$, $\psi(A_1)$,\ldots instead of
$(\lambda x,y.\varphi_xy)[A_1,A_2]$, $(\lambda x.\psi(x))[A_1]$).
The convention is here that the variables $x_i$ are ordered from
left to right and in case of $x \choose  y$ from bottom to top
(i.e. $\varphi_{A_1}A_2$ is ordered by lexicographic ordering
on $(A_1,A_2)$, $A_1 \choose A_2$ by lexicographic ordering
on $(A_2,A_1)$).
Note that, after some standard G{\"o}delization
of terms,
the new ordering in the examples with $f= \lambda \xvec.t$
is an elementary construction from the
orderings $A_i$.
\ssubitem d
If $A$ is an ordering, $A^\ast_\des$ ($A^\ast_\weakdes$) is the set/class of
--- possibly empty --- strictly descending (weakly descending) sequences 
ordered lexicographically, which is an elementary construction from $A$.
We omit double brackets for the elements of $A^\ast_\des$ and
$A^\ast_\weakdes$, writing for instance 
$(a_1, b_1 ,\ldots, a_m,b_m)$ instead of
$((a_1,b_1) ,\ldots, (a_m,b_m))$ for an element of
$(A,B)_\des$ and 
${a_1 \cdots a_m \choose b_1 \cdots b_m}$
%$\left(\begin{array}{ccc}b_1& \cdots& b_m \\
%a_1  &\cdots &a_m \end{array} \right)$  
instead of $\left ({a_1 \choose b_1} \cdots {a_m \choose
b_m}\right )$  
for an element of  ${A \choose B}^\ast_\des$.
Obviously $A^\ast_\des$ is an elementary construction from $A$.
\ssubitem {da} (Generalization of Sch{\"u}tte's Klammer symbols 
\cite{Schuette54}).
If $A$, $B$ are orderings, let\\
$\schuette {A \choose B}:=
{A \choose B}_\des \cap 
\{ {a_1 \cdots a_m \choose b_1 \cdots b_m}
%\left(\begin{array}{ccc}b_1& \cdots& b_m \\
%a_1  &\cdots &a_m \end{array} \right)
\mid m \in \omega,a_i \in A, b_i \in B, b_1 >_B \cdots >_B b_m \} $. Note that we have
reversed the order relative to \cite{Schuette54}.
Further we define
$\schuette(A,B)  $\\
$:=((A ,B))_\des \cap 
\{(a_1,b_1 ,\ldots, a_m,b_m)\mid 
m \in \omega, a_i \in A, b_i \in B, a_1 >_A \cdots >_A a_m \} $.
Both are elementary constructions from $A$, $B$.
\ssubitem e
If $A_1 ,\ldots, A_m$ are orderings and $|A_i|$ are  disjoint, then
$B:= $\underlinedef{$A_1 \preccirc A_2 \preccirc \cdots \preccirc A_m$}
is the ordering with
$|B|$ being the union of the $A_i$ and
$\prec_B$ being the union of the $\prec_{A_i}$ 
together with the pairs $(a,b)$ for $a \in |A_i|$, $b \in |A_j|$,
$1 \leq i < j \leq n$.
This is an elementary construction from the orderings
$A_i$.
\ssubitem f We identify the one element set $A$ with the
ordering $(A,\emptyset)$.
\ssubitem g If $A$ is an ordering,
$B$ a set, let
$A \cap B
:= (|A| \cap B,\prec_A \cap ((|A|\cap B) \cross (|A|\cap B)))$ and
$A \setminus B:= A \cap (|A| \setminus B)$.
Note that this is an elementary construction from $A$,
if $B$ is primitive recursive.
\ssubitem h If $A$ is an ordering, let $\Acc(A)$ be the
accessible part of $|A|$ with respect to $\prec_A$,
i.e. the largest well-founded part of $A$,
$\bigcup \{ X \subseteq |A| \mid (X,\prec)$ well-ordered $\} $.
$\Acc_\prec(B):= \Acc(B,\prec)$. 
\ssubitem i If $A$ is an ordering, $a,b \in |A|$,
$B \subseteq |A|$, let $B \cap a:= \{ c \in B \mid 
c \prec_A a \} $,
and $[a,b]:= \{ c \in |A| \mid a \preceq_A c \preceq_A b \}$. The
half open and open intervals 
$[a,b[$, $]a,b]$ and $]a,b[$ are defined similarly.
Again we obtain elementary constructions from $A$, if
$B$ is primitive recursive.
\ssubitem j If $A$ is an ordering, $B,C \subseteq |A|$, then
$B \segment C$ ({\em $B$ is an initial segment of $C$}) 
$:\Iff B \subseteq C \land 
\all x \in B. B \cap x = C \cap x$.
\ssubitem k $\Ord$ is the class of ordinals,
$\APN$ the class of additive principal numbers $>0$.\\
We identify classes of ordinals $A$ with the ordering
$(A,<)$, where $<$ is the usual ordering on $\Ord$.
\end{enumerate} }
\end{definition}
%%%%%%%%%%%%%%%%%%%%%%%%%%%%%%%%%%%%%%%%%%%%
%%%
%%                                         %
%%%%%%%%%%%%%%%%%%%%%%%%%%%%%%%%%%%%%%%%%%%%
%%
%%
%%
%\begin{definition}{\rm 
%\laprint{notations}
%\begin{enumerate} 
%\ssubitem a 
%If $A$ is a set, let
%\underlinedef{$A^\ast$} be the set of finite (possible empty)
%sequences of elements of $A$
%coded, if $A \subseteq \nat$ 
%as natural numbers such that the usual properties, especially 
%primitive recursiveness  hold.\\
%\underlinedef{$\seqlength(\avec)$}
%is the length of an element of $\avec$, \underlinedef{$(\avec)_i$} the
%$i$-th element of $\avec$.\\
%An extension $\PRA^+$ of $\PRA$ by \underlinedef{free predicate} is the
%result of adding new predicates to the language and logic
%of $\PRA$ without adding induction over formulas involving such
%predicates.\\
%An \underlinedef{ordering} is a pair $(A, \prec)$ where $A$ is a 
%set and $\prec$ is a binary relation on $A$. It is 
%\underlinedef{primitive recursive}, if $A$ is a primitive
%recursive subset of $\nat$ and $\prec$ is primitive recursive,
%and \underlinedef{linear}
%if $\prec$ is a linear ordering on $A$.\\ 
%If $A = (B, \prec)$,
%then \underlinedef{$|A|$}$:= B$,
%\underlinedef{$\prec_A$}$:= \prec$.
%If $C \subseteq |A|$,
%we will write $(C,\prec)$ instead of $(C, \prec \cap (B \cross B))$.\\
%\underlinedef{$\TI_{(A,\prec)}$} is transfinite
%induction over the linear ordering $(A,\prec)$,
%\underlinedef{$\TI_{(A,\prec)}(\phi(x))$} the special instance of it
%for the formula $\phi$ (with respect to the variable $x$.\\
%Transfinite induction over $(B,\prec)$ is $\PRA$-reducible to
%transfinite induction over $(A_i,\prec_i)$ ($i=1 ,\ldots, n$), 
%if in an extension $\PRA^+$
%of $\PRA$ by a free predicate $R$ 
%if there exist formulas $phi_{i,j}(x,\zvec)$ such that 
%$$\PRA^+ \prove
%(\Andd_{i=1}^n \Andd_{j=1}^{n_i} \all \zvec (\TI_{(A_i,\prec_i)}(\phi_{i,j}(x,
%\zvec)))) \ar \TI_{(B,\prec)}\enspace .$$
%\ssubitem {ab} If $(B,\prec)$ is an ordering, construced from orderings
%$(A_1,\prec_1)$ ,\ldots,  $(A_m,\prec_m)$, this is an 
%\underlinedef{elementary
%construction}, if $B,\prec$ is primitive recursive, linear,
%linear provable in $\PRA$, provided  this property holds for 
%$(A_i,\prec_i)$ and
%if, under the condition that $(A_i,\prec_i)$ are primitive recursive
%and linear provably in $\PRA$, transfinite induction over
%$(A,\prec)$ reduces to transfinite induciton over $(A_i,\prec_i)$.
%\ssubitem b 
%If $f:A \ar B$, $M \subseteq A$, then 
%\underlinedef{$f[M]$}$:= \{ f(x) \mid x \in M \} $.\\
%\underlinedef{$\potsetfin(A)$} is 
%the set of finite sets of elements of $A$ 
%coded, if $A \subseteq \nat$, 
%as natural numbers, such that the usual properties, especially 
%primitive recursiveness, decidable subset-relation hold.\\
%In case $B \in \potsetfin(A)$, we write $t \in B$ for the
%statement expressing $t$ is an element of $B$ expressed in
%the language of $\PRA$.\\
%If $B \in \potsetfin(A)$, $a \in A$, $\prec$ is
%a binary relation of $A$, then
%\underlinedef{$B \prec a$}$\Iff \all x \in B.x \prec a$,
%\underlinedef{$a \prec B$}$\Iff \exists x \in B.a \prec x$.\\ 
%If 
%$\prec$ is a binary relation on $A$, define \underlinedef{$a \preceq b$}$\Iff
%a \prec b \lor a = b$, similarly for $\prec '$ etc.\\
%If $k: A \ar \potsetfin(B)$ and $C \subseteq B$, then
%\underlinedef{$k^{-1}(C)$}$:= \{ x \in A \mid k(x) \subseteq C \} $.\\
%We write \underlinedef{$f: A \ar_\omega B$} for $f: A \ar \potsetfin(B)$.
%If $f: A \ar_\omega B$, $f': A \ar_\omega B$, $g:B \ar_\omega C$, 
%then \underlinedef{$g \circ f$}$:A \ar_\omega C$, $(g \circ f)(a):= g[f(a)]$,
%and $f \subseteq f' \Iff \all x \in A.f(x) \subseteq f'(x)$.
%\ssubitem c
%If $A_1 ,\ldots, A_m$ are orderings,
%$f: (|A_1| \cross  \cdots \cross |A_m|) \ar M$,
%then $f[A_1 ,\ldots, A_m]$ denotes the ordering
%$(f[A_1 ,\ldots, A_m],\prec)$ where
%$f(a_1 ,\ldots, a_m) \prec f(b_1 ,\ldots, b_m)$ if
%$(a_1 ,\ldots, a_m)$  lexicographically less then
%$(b_1 ,\ldots, b_m)$ with respect to the orderings
%$A_1 ,\ldots, A_m$. \par 
%We will use this definition only in case $f(a_1 ,\ldots, a_m)
%= t[x_1:= a_1 ,\ldots, x_m:= a_m]$ for some term $t$
%(like $\varphi_xy$, $\psi(x)$, $(x,y,z)$, $x+y$, $x \choose y$)
%and will write in this case the result of replacing in $t$
%the variable $x_i$ by $A_i$ instead of $f[A_1 ,\ldots, A_m]$
%(i.e. $\varphi_{A_1}A_2$, $\psi(A_1)$ ,\ldots).
%The convention is here that the variables $x_i$ are ordered from
%left to right and in case of $x \choose  y$ from bottom to top
%(i.e. $\varphi_{A_1}A_2$ is ordered by lexicographic ordering
%on $(A_1,A_2)$, $A_1 \choose A_2$ by lexicographic ordering
%on $(A_2,A_1)$).
%Note that, after some standard G{\"o}delization,
%the new ordering is an elementary construction from the
%orderings $A_i$.
%\ssubitem d
%If $A$ is an ordering, $A^\ast_\des$ is the set of
%(possibly empty) strictly descending sequences of $A$ ordered
%lexicographically, which is an elementary construction from $A$.
%In case of $({B \choose A})^\ast_\des$,
%we write $\left(\begin{array}{ccc}b_1& \cdots& b_m \\
%a_1  &\cdots &a_m \end{array} \right)$ 
%instead of the element$\left ({b_1 \choose a_1} \cdots {b_m \choose
%a_m}\right )$ of this ordering.
%\ssubitem e
%If $A_1 ,\ldots, A_m$ are orderings and $|A_i|$ are  disjoint, then\\
%$B:= $\underlinedef{$A_1 \preccirc A_2 \preccirc \cdots \preccirc A_m$}
%is the ordering with
%$|B|$ being the union of the $A_i$ and
%$\prec_B$ being the union of the $\prec_{A_i}$ 
%together with the pairs $(a,b)$ for $a \in |A_i|$, $b \in |A_j|$,
%$1 \leq i < j \leq n$.
%This is again an elementary construction from the orderings
%$A_i$.
%\ssubitem f We indentify the one element set $A$ with the
%ordering $(A,\emptyset)$.
%\ssubitem g If $A$ is an ordering,
%$B$ a set, let\\ 
%$A \restriction B
%:= (|A| \cap B,\prec_A \cap ((|A|\cap B) \cross (|A|\cap B)))$.
%Note again that this is an elementary construction from $A$,
%if $B$ is primitive recursive.
%\ssubitem h If $A$ is an ordering, let $\Acc(A)$ be the
%accessible part of $|A|$ with respect to $\prec_A$,
%i.e. the largest well-founded part of $A$,
%$\bigcap \{ X \subseteq |A| \mid (X,\prec)$ well-ordered $\} $.
%$\Acc_\prec(B):= \Acc(B,\prec)$. 
%\ssubitem i If $A$ is an ordering, $a,b \in |A|$,
%$B \subseteq |A|$, let $B \cap a:= \{ c \in |A| \mid 
%c \prec_A a \} $,
%and $[a,b]:= \{ c \in |A| \mid a \preceq_A c \preceq_A b \}$.
%$[a,b[$, $]a,b]$ and $]a,b[$ are defined similarly.
%Again we get elementary constructions from $A$, if
%$B$ is primitive recursive.
%\ssubitem j If $A$ is an ordering, $B,C \subseteq |A|$, then
%$B \segment C$ ($B$ is an initial segment of $C$) $:\Iff B \subseteq C \land 
%\all x \in B. B \cap x = C \cap x$.
%\end{enumerate} }
%\end{definition}
%
%
%%
%The essential definition in \cite{setzvenedig} was now:
%%
%%%%%%%%%%%%%%%%%%%%%%%%%%%%%%%%%%%%%%%%%%%%
%%%
%%                                         %
%%%%%%%%%%%%%%%%%%%%%%%%%%%%%%%%%%%%%%%%%%%%
%%
%%
%%
%\begin{definition}{\rm 
%\laprint{defordfrombelow}
%Let $\T$ be a subset of the natural numbers,
%$\NF$ be a subset of $\T^\ast$, the set of finite sequences of elements in
%$\T$, $\prec$, $\prec'$ be binary relations on
%$\T$ and $\NF$, respectively and $f:\NF \ar \T$ injective
%such that $\T$ is the closure under $f$, i.e.
%the least set $M$ such that if $\avec \in \NF \cap M^\ast$, then
%$f(\avec) \in M$.
%Assume that $\T$, $\prec$, $\NF$, $\prec '$ 
%and $f$ are primitive recursive, and that
%$\PRA$ proves, that $\prec$ is linear and
%$\T$ is the closure under $f$ (i.e. the corresponding induction principle
%can be proved for all formulas in extensions of $\PRA^+$ by
%free predicates.
%Then $(\T,\prec, \NF,\prec',f)$ 
%is an \underlinedef{ordinal notation system from below}, iff the 
%following holds:\\
%%
%%
%%
%\condition{\Below\ 1}
%${\PRA} \prove \all \avec \in \NF.\all i < \seqlength(\avec).
%(\avec)_i \prec f(\avec)$.\\
%%
%%
%\condition{\Below\ 2} 
%${\PRA} \prove \all \avec \in  \NF.\forall b  \in \T.
%b \prec f(\avec) \ar \exists i < \seqlength(\avec).$\\
%\conditionbox{}$b \preceq \avec_i \lor 
%\exists \cvec  \prec ' \avec .b = f(\cvec)$.\\
%%
%%
%%
%\condition{\Below\ 3} 
%$\TI_{(\NF\cap R^\ast,\prec')}$ is
%$\PRA$-reducible to $\TI_{(\T \cap R,\prec)}$ for
%a unary\\
%\conditionbox{} free predicate $R$.}
%%
%%
%%
%%
%%
%\end{definition}
%
%
%
%The intuition was that notations for ordinals refer only to smaller
%ordinals (Below 1), that if we introduce a new ordinal, all
%ordinals are introduced previously, i.e. they are either below one
%of the ordinals from which the new ordinal is defined 
%($b \preceq a_i$) or it is smaller with respect to the
%ordering $\prec '$, which determines the order in which new ordinals
%are introduced. $\prec '$ should be always a well-ordering for the
%set of ordinals we can built from the ones which were 
%already introduced, and this reduction was supposed to be provable
%in an elementary way, which is expressed by condition (Below 3).
%In all the examples we needed only a combination of 
%lexicographic ordering on pairs, lexicographic ordering on descending
%sequences, selection of subsets of orderings and renaming, in
%order to introduce $\prec$ from $\prec '$, which guaranteed
%property (Below). We will give examples after having revised the
%definition slightly.\par 
%
%
%
\section{Elementary Ordinal Systems}
\laprint{sectelem}
%
%
\subsection{Definition of Ordinal Systems}
We will in the following develop first the notation of ordinal systems
from that of ordinal notation systems from below as defined
in the first four chapters of \cite{setzvenedig}. However, it is
not necessary to read this article, since for the reader who doesn't know the
previous approach we will then
repeat the motivation given there, adapted to the new
setting.\par 
In \cite{setzvenedig}, the underlying structure of ordinal notation
systems from below consisted of the set of notations $\T$, a subset
$\NF$ of $\T^\ast$, linear orderings $\prec$ on $\T$ and $\prec '$ on
$\NF$ and a function $f: \NF \ar \T$. In order to deal with stronger
systems, which extend the notion of ordinal notation system from below
and allow to define ordinals beyond the Bachmann-Howard ordinal, one needs
to deal with arguments that have more structure. Even in the
case of ordinals below the Bachmann-Howard ordinal some more
structure on the argument is needed. For instance 
in case of extended Sch{\"u}tte Klammer symbols (in English:
``parenthesis symbols'' or  better ``matrix symbols''), which 
will be defined in Example \refcom{exmplordfun}i below,
we need the information about the size of each of the sub-matrices and
which ordinal notations belong to which sub-matrix. We could handle
this by using some coding (see Subsection \ref{equivonsfb}).
However, the more elegant
approach is to replace the set $\NF \subseteq \T^\ast$ by an
arbitrary set $\Arg$ together with a function $\k: \Arg \ar_\omega \T$.
The intuition is, that $a \in \Arg$ is an argument for the function
$f$ which is built from ordinal notations $\k(a)$, but has some
additional structural information. An ordinal notation system from 
below can be translated into the new structure by defining
$\Arg:= \NF$ and $\k(a_1 ,\ldots, a_m):= \{ a_1 ,\ldots, a_m \}$.\par 
Now $f$ was  always a bijection, and therefore 
we can identify $\Arg$ with $\T$, define $f:= \lambda x.x$ and
omit $f$ completely. We have  therefore 
two orderings on $\T$, $\prec$ and  $\prec '$,
$\prec$ being the ordering on the ordinal notation system and 
$\prec '$ being the ordering which determines the  order, in which new
ordinal notations are introduced.\par 
In order to express that $\T$ is the closure under the above process, we
add a function $\length: \T \ar \nat$ and require
$\length(a) < \length(b)$ for $a \in \k(b)$. We replace the long
name ``ordinal notation system from below'' by ``ordinal system''.\par 
The motivation for the resulting structure is now as follows: Ordinal notations
$t$ are built from a finite set of notations $\k(t)$. We want
that the system is built from below: First,
an ordinal should be
denoted using smaller ones, i.e. $\k(t) \prec t$. Second, whenever
we introduce a new notation $t$, we want to have constructed all smaller 
notations $s$ before  $t$. Either $s$ could
be below one of the components of $t$, i. e. $s \preceq \k(t)$, since, whenever
we introduce an ordinal, we assume that we have constructed its
components and therefore all  ordinals below them as well. 
Or $s$ must have been introduced before $t$ with respect to the
termination ordering $\prec'$, 
i. e. $s \prec ' t$.  In \cite{setzvenedig} we showed 
that in case of simple systems like
the standard system up to $\Gamma_0$ or the Sch{\"u}tte Klammer symbols,
we can construct $\prec'$ from $\prec$ by using the lexicographic ordering on 
pairs and on strictly descending sequences together with some simple operations.
For instance we ordered terms for the Cantor normal form by
the lexicographic ordering on strictly descending sequences and terms
$\varphi_ab$ by the lexicographic ordering on pairs $(a,b)$.
Orderings $\prec'$ constructed like this have the property that, whenever a 
set of notations is $\prec$-well-ordered, the set of notations
built from it is $\prec'$-well-ordered, and in the definition
of ordinal systems we will demand this condition.
However, in the above examples it is possible to show 
this reduction in $\PRA$, i.e. essentially in logic, 
and we define elementary
ordinal systems by demanding additionally this stronger 
requirement. Elementary ordinal systems will
have strength below the Bachmann-Howard ordinal, whereas with ordinal
systems we  will have no upper bound (choose for an arbitrary
well-ordering $(\T,\prec)$, $\k(a) := \emptyset$, $\length(a):= 0$,
$\prec':= \prec$).
%
%
%
%%%%%%%%%%%%%%%%%%%%%%%%%%%%%%%%%%%%%%%%%%%
%%
%                                         %
%%%%%%%%%%%%%%%%%%%%%%%%%%%%%%%%%%%%%%%%%%%
%
%
%
\begin{definition}{\rm 
\laprint{defordinalsystem}
\begin{enumerate}
\ssubitem a An \underlinedef{Ordinal System Structure}, in short
\underlinedef{OS-structure}, 
is a quintuple $\calF = (\T, \prec,\prec',\k,\length)$ where 
$\T$ is a set, 
$\prec '$, $\prec$ are linear orderings on
$\T$, $\k:\T \ar_\omega \T$ and
$\length: \T \ar \nat$.\\
We will sometimes regard the first two or three elements of this
quintuple  as a unity, so when writing $\calF$ is an OS-structure of the 
form $\calF=(A,\prec',\k,\length)$ or $(\calG,\k,\length)$ we mean
that $A$ is of the form $(\T,\prec)$ and
$\calG$ is of the form $(\T,\prec,\prec ')$ with $\T$, $\prec$
and $\prec '$ as above.
\ssubitem b In the following, if not mentioned otherwise, let
$\calF$ be as in \refact a, and for any index $i$
$\calF_i = (\T_i,\prec_i,\prec'_i,\k_i,\length_i)$.
\ssubitem {ba} If $\calF$ is an ordinal system structure as above,
$A,B \subseteq \T$, then $B \restriction A := B \cap \k^{-1}(A)$,
the {\it restriction of $B$ to arguments built from $A$} or shorter
{\it restriction of $B$ to $A$}.
\ssubitem c An \underlinedef{Ordinal System}, in short \underlinedef{OS},
is an OS-structure $\calF$, such that
the following holds:\\ 
\condition{\OS\ 1} $\all t \in \T.\k(t) \prec t$.\\
\condition{\OS\ 2} $\length[\k(t)] < \length(t)$.\\
\condition{\OS\ 3} $\all t \in \T.\all s \in \T
(s \prec t \ar (s \preceq \k(t) \lor 
s \prec ' t))$.\\
\condition{\OS\ 4} If $A \subseteq \T$,
$A$ is $\prec$-well-ordered, then $\T \restriction A$ is 
$\prec'$-well-ordered.
\ssubitem e An OS $\calF$ is \underlinedef{primitive recursively represented},
if $\T$ is a primitive recursive subset of $\nat$,
$\prec$, $\prec '$, $\k$, $\length$ are primitive recursive,
and 
all the properties of an OS, including linearity of
$\prec$, $\prec '$, but except of (OS\ 4), can be shown
in $\PRA$.\par 
\ssubitem f A primitive recursively represented  OS $\calF$ 
is \underlinedef{elementary}, if additionally $(\OS\ 4)$ can 
be shown in $\PRA$, i.e.
for a free predicate $R$
$\TI_{(\T \restriction R,\prec ')}$ is $\PRA$-reducible
to $\TI_{(\T  \cap R,\prec)}$.
\ssubitem g An OS-structure is \underlinedef{well-ordered}, 
if $\T$ is well-ordered. The
\underlinedef{order type} of a well-ordered OS-structure $\calF$ 
is the order type of 
$\T$.
\ssubitem h Two OS-structures are isomorphic, if there exists a bijection
between the underlying sets, which respects $\prec$, $\prec'$, $\k$,
$\length$.
\end{enumerate}}
\end{definition}
We illustrate
an ordinal system by 
the following picture:\\
%%% begin pic3.pictex
%% pic3.fig
\font\thinlinefont=cmr5
%
\begingroup\makeatletter\ifx\SetFigFont\undefined
% extract first six characters in \fmtname
\def\x#1#2#3#4#5#6#7\relax{\def\x{#1#2#3#4#5#6}}%
\expandafter\x\fmtname xxxxxx\relax \def\y{splain}%
\ifx\x\y   % LaTeX or SliTeX?
\gdef\SetFigFont#1#2#3{%
  \ifnum #1<17\tiny\else \ifnum #1<20\small\else
  \ifnum #1<24\normalsize\else \ifnum #1<29\large\else
  \ifnum #1<34\Large\else \ifnum #1<41\LARGE\else
     \huge\fi\fi\fi\fi\fi\fi
  \csname #3\endcsname}%
\else
\gdef\SetFigFont#1#2#3{\begingroup
  \count@#1\relax \ifnum 25<\count@\count@25\fi
  \def\x{\endgroup\@setsize\SetFigFont{#2pt}}%
  \expandafter\x
    \csname \romannumeral\the\count@ pt\expandafter\endcsname
    \csname @\romannumeral\the\count@ pt\endcsname
  \csname #3\endcsname}%
\fi
\fi\endgroup
\mbox{\beginpicture
\setcoordinatesystem units <1.00000cm,1.00000cm>
\unitlength=1.00000cm
\linethickness=1pt
\setplotsymbol ({\makebox(0,0)[l]{\tencirc\symbol{'160}}})
\setshadesymbol ({\thinlinefont .})
\setlinear
%
% Fig CIRCULAR ARC object
%
\linethickness= 0.500pt
\setplotsymbol ({\thinlinefont .})
%
% arrow head
%
\plot  3.553 21.288  3.334 21.431  3.460 21.202 /
%
\circulararc 39.308 degrees from  3.810 20.320 center at  2.016 20.209
%
% Fig CIRCULAR ARC object
%
\linethickness= 0.500pt
\setplotsymbol ({\thinlinefont .})
%
% arrow head
%
\plot  2.795 22.006  2.540 22.066  2.736 21.893 /
%
\circulararc 22.620 degrees from  3.334 21.431 center at  1.349 19.764
%
% Fig CIRCULAR ARC object
%
\linethickness= 0.500pt
\setplotsymbol ({\thinlinefont .})
%
% arrow head
%
\plot  4.353 21.984  4.604 21.907  4.419 22.093 /
%
\circulararc 53.130 degrees from  4.604 21.907 center at  3.413 19.923
%
% Fig CIRCULAR ARC object
%
\linethickness= 0.500pt
\setplotsymbol ({\thinlinefont .})
%
% arrow head
%
\plot  5.629 20.567  5.715 20.320  5.756 20.579 /
%
\circulararc 59.863 degrees from  5.715 20.320 center at  3.781 20.149
%
% Fig POLYLINE object
%
\linethickness= 0.500pt
\setplotsymbol ({\thinlinefont .})
\putrule from  2.540 23.530 to  2.540 20.320
\putrule from  2.540 20.320 to  6.350 20.320
\linethickness= 0.500pt
\setplotsymbol ({\thinlinefont .})
%
% Fig INTERPOLATED PT SPLINE
%
\plot  4.445 20.320 	 4.432 20.387
	 4.421 20.449
	 4.398 20.561
	 4.378 20.656
	 4.359 20.737
	 4.323 20.863
	 4.286 20.955
	 4.228 21.059
	 4.187 21.121
	 4.136 21.190
	 4.075 21.270
	 4.001 21.362
	 3.913 21.468
	 3.810 21.590
	/
%
% arrow head
%
\plot  4.023 21.438  3.810 21.590  3.926 21.355 /
%
\linethickness= 0.500pt
\setplotsymbol ({\thinlinefont .})
%
% Fig INTERPOLATED PT SPLINE
%
\plot  3.810 21.590 	 3.718 21.659
	 3.638 21.718
	 3.569 21.768
	 3.509 21.809
	 3.411 21.869
	 3.334 21.907
	 3.218 21.947
	 3.146 21.965
	 3.060 21.984
	 2.959 22.003
	 2.840 22.022
	 2.773 22.033
	 2.701 22.043
	 2.624 22.055
	 2.540 22.066
	/
%
% arrow head
%
\plot  2.800 22.094  2.540 22.066  2.783 21.969 /
%
%
% Fig TEXT object
%
%\put{\SetFigFont{12}{14.4}{rm}$t$} [lB] at  2.064 22.066
\put{\SetFigFont{12}{14.4}{rm}$t$} [lB] at  2.264 21.966
%
% Fig TEXT object
%
%\put{\SetFigFont{12}{14.4}{rm}$r$} [lB] at  3.810 20.003
\put{\SetFigFont{12}{14.4}{rm}$r$} [lB] at  3.710 20.003
%
% Fig TEXT object
%
%\put{\SetFigFont{12}{14.4}{rm}$s$} [lB] at  4.445 20.003
\put{\SetFigFont{12}{14.4}{rm}$s$} [lB] at  4.345 20.003
%
% Fig TEXT object
%
%\put{\SetFigFont{12}{14.4}{rm}$t$} [lB] at  5.715 20.003
\put{\SetFigFont{12}{14.4}{rm}$t$} [lB] at  5.615 20.003
%
% Fig TEXT object
%
\put{\SetFigFont{12}{14.4}{rm}$k$} [lB] at  3.810 20.955
%
% Fig TEXT object
%
\put{\SetFigFont{12}{14.4}{rm}$k(t)=$} [lB] at  1.287 21.273
%
% Fig TEXT object
%
\put{\SetFigFont{12}{14.4}{rm}$\{r,s\}$} [lB] at  1.287 20.765
%
% Fig TEXT object
%
\put{\SetFigFont{12}{14.4}{rm}$(T,\prec')$} [lB] at  2.657 22.895
%
% Fig TEXT object
%
\put{\SetFigFont{12}{14.4}{rm}$(T,\prec)$} [lB] at  6.032 20.637

\linethickness=0pt
\putrectangle corners at  1.587 23.555 and  6.375 19.952
\endpicture}
%%% end pic3.pictex
%
%
%

\noindent (The arrow from $t$ in $(\T,\prec ')$ to $t$ in $(\T,\prec)$ represents
the function $f$ in the original definition, which is now the identity).\par 
%
%
%
%
\subsection{Equivalence of Elementary 
Ordinal Systems and Ordinal Notation Systems from below.}
\laprint{equivonsfb}
Let $\prec''$ be the ordering on $\NF$ in an ordinal notation system 
from below, $f$ the function used there,
define $\prec'$ on $f[\NF]$ 
by  $f(\avec) \prec' f(\bvec) :\Iff 
\avec \prec '' \bvec$,
$\k(f(\avec)):= \{ a_1 ,\ldots, a_n \}$ and $\length$ in the
obvious way. Then, if we started with 
an ordinal notation system from below
in which
$\prec ''$ is linear,  provably in $\PRA$, 
we obtain an elementary OS.\par
In the other direction, assume an elementary OS, together
with an explicit enumeration of the natural numbers,
i. e. there exists a primitive recursive
function $g: \omega \ar \T$,
where we write $\underbar{n}$ for $g(n)$, such that
$\all n.n \leq  \underbar{n}$ and
$\all n \in \nat.\underbar{n} =$ $n$th element of $\T$ provable in 
$\PRA$ (more precisely the formula
$\all n \in \nat(n  \leq \underline{n} \land 
\all x \in \T(x \prec \underline{n} \iff \exists l<n.
x = \underbar{l}))$ is provable in $\PRA$). 
Note that therefore  $g[\omega]$ and $g^{-1}: g[\omega] \ar \omega$ are 
primitive recursive. Define $\T':= \T$,
$\NF' \subseteq {\T'}^\ast$,\\
$\NF:= \{ () \} \cup
\{ (\underbar{0},\underbar{n}) \mid n \in \omega \}
\cup$\\
\hbox{\qquad}$\{ (\underbar{1},\underbar{t},s_1 ,\ldots, s_m ) \mid 
t \in \T \setminus  g[\omega] \land 
m \in \omega \land
s_1 \prec \cdots \prec s_m
\land \k(t) = \{ s_1 ,\ldots, s_m \} \}$;\\
$f': \NF' \ar \T'$,
$f'():= \underbar{0}$, 
$f'(\underbar{0},\underbar{n}):= \underbar{n+1}$,
$f'(\underbar{1},\underbar{t},s_1 ,\ldots, s_m):=
t$;\\ 
$\prec ''$ on $\NF$ by
$() \prec '' (\underbar{0},\underbar{0})
\prec '' (\underbar{0},\underbar{1})
\prec '' (\underbar{0},\underbar{2})
\prec '' \cdots \prec '' 
(\underbar{1},\underbar{t},s_1 ,\ldots, s_m ) $
for all $t$ and\\
$(\underbar{1},\underbar{t},s_1 ,\ldots, s_m ) 
\prec '' 
(\underbar{1},\underbar{t}',s_1' ,\ldots, s_l' )$
iff $t \prec ' t'$.\\  
Then one sees easily
that $(\T,\prec,\NF,\prec '',f')$ is an ordinal notation system from
below, which has
the same order type as the original ordinal system 
and has essentially ``the same form'',
therefore under the above mentioned
weak conditions, elementary OS can be seen as
ordinal notation systems from below.\par 
%
%
%
\subsection{\bf Intuitive Argument why OS are Well-ordered.}
%The idea for what we regard as an intutive well-ordering proof comes from
%the way one explains ordinals to non-specialists, it works even for 
%people with no background in mathematics and children. What works
%very well is to explain it as follows. We start counting $1,2,3,\ldots$.
%When we have exhausted the natural numbers, we continue with
%$\omega$ as a symbol for infinity, and count $\omega,\omega+1,\omega+2 ,
%\ldots $. And continue $\omega+\omega,\omega+\omega+1, \omega+\omega+\omega+2
%,\ldots \omega \cdot 3 ,\ldots, \omega \cdot 4 ,\ldots, \omega \cdot \omega = 
%\omega^2,\ldots, \omega^3 ,\ldots,\omega^4 ,\ldots,  \omega^\omega ,\ldots, 
%\omega^{\omega^\omega} ,\ldots, \omega^{\omega^{\omega^\omega}}
%,\ldots, \varphi_00$ (let us take the $\varphi$-function with
%$\varphi_0 \alpha = \epsilon_\alpha$),
%$,\ldots, \varphi_01 ,\ldots, \varphi_02 ,\ldots, \varphi_0(\varphi_00)
%,\ldots, \varphi_0(\varphi_0(\varphi_00)) ,\ldots, \varphi_10$ etc. etc. 
%(How far we reach depends on the mental capacity of the person we
%are talking with).\par 
%Now in this process of enumberating all the ordinals, the $\ldots$
%between two notations meant: continue with the current process 
%indefitely or take a process we already understand (like how
%to count $\omega \cdot 3, \omega \cdot 3 +1, \omega \cdot 3+2 ,\ldots$)
%indefinetely. Using ordinal systems we can explain what is going on.
We will first illustrate, what we regard as an intuitive argument for the
well-ordering of ordinal system, by taking a standard example.\par 
Take the notation system built from the
Cantor Normal form and the Veblen-function, based on $\varphi_0 \alpha = 
\epsilon_\alpha$. More precisely, it is  defined as follows:
Let $\nathat$ be the set of terms $\{ 0,\S(0),\S(\S(0)) ,\ldots \}$.
$1:= \S(0)$.
The set of ordinal notations $\T$ together with the ordering
$\prec$ on $\T$ are simultaneously defined by:
$\nathat \subseteq \T$, if
$a_1 ,\ldots, a_m \in \T $, $n_i \in \nathat \setminus \{ 0 \} $, 
$m>0$, 
then 
$\varphi_{a_1}a_2 \in \T $
provided $m=2$ and $a_2$ is not of the form
$\varphi_{b_1}b_2$ with $a_1 \prec b_1$, 
and 
$\omega^{a_1} \cdot n_1 +\cdots+ \omega^{a_m} \cdot n_m \in \T$
provided $a_m \prec \cdots \prec a_1$
and if $m=1$, then $a_1 \not = 0$ and $a_1$ is not of the form
$\varphi_{b_1}b_2$.  
$\S^k(0) \prec \S^l(0): \Iff k < l$,
$\S^k(0) \prec \omega^{a_1}\cdot n_1 +\cdots+ \omega^{a_m}\cdot n_m$,
$\S^k(0) \prec \varphi_ab$, 
$\omega^{a_1} \cdot n_1 +\cdots+ \omega^{a_m}\cdot n_m 
\prec 
\omega^{b_1} \cdot l_1 +\cdots+ \omega^{b_k}\cdot l_k
:\Iff  (a_1,n_1 ,\ldots, a_m,n_m)$ is lexicographically
smaller than $(b_1,l_1 ,\ldots, b_k,l_k)$,
$c:= \omega^{a_1} \cdot n_1 +\cdots+ \omega^{a_m}\cdot n_m
\prec \varphi_ab :\Iff a_1 ,\ldots, a_m,n_1 ,\ldots, n_m 
\prec \varphi_ab$ and, if this condition does not hold,
then $\varphi_ab \prec c$, 
and $\varphi_ab \prec \varphi_{a'}b' :\Iff 
(a \prec a' \land b \prec \varphi_{a'}b')
\lor (a = a' \land b \prec b') \lor 
(a' \prec  a \land \varphi_ab \prec b')$.
The termination ordering $\prec'$ on $\T$ is now defined by
$\S^k(0) \prec' \omega^{a_1}\cdot n_1 +\cdots+ \omega^{a_m}\cdot n_m 
\prec' \varphi_ab$,
$\S^k(0) \prec ' \S^l(0)$ iff $k<l$, $\prec'$ on terms
$\omega^{a_1}\cdot n_1 +\cdots+ \omega^{a_m}\cdot n_m$ is the 
lexicographic ordering, and $\prec'$ on terms $\varphi_ab$ 
is the lexicographic ordering on pairs $(a,b)$ (the last two with respect
to $\prec$).
Further define $\k(0)= \emptyset$, $\k(\S(t)):= \{ t \}$,
$\k(\omega^{a_1} \cdot n_1 +\cdots+ \omega^{a_m}\cdot n_m):= 
\{ a_1 ,\ldots, a_m,n_1 ,\ldots, n_m \}$
 and $\k(\varphi_ab):= \{ a,b \}$.\par 
It is easy to see that with the obvious definition of
$\length$ we get an ordinal system
$(\T,\prec,\prec ',\k,\length)$, especially
we have: If $A \subseteq \T$ is $\prec$-well-ordered,
the set of notations ``built from $A$'' or ``with components in
$A$'', namely
$\k^{-1}(A)$, is $\prec'$-well-ordered.
Now we proceed by selecting ordinal notations 
as follows: At the beginning we have no
notation at all. The only notation which has components in the empty
set is
$0$ and we select it. Now, if we take the the ordinals with components
in the set of notations selected, namely in $\{ 0 \}$, we choose
the $\prec '$-least
one not selected before, namely $\S(0)$.
Again, we take all notations with
components in $\{ 0,\S(0) \}$, select the $\prec'$-least one not
chosen before,
$\S(\S(0))$, etc. Once we have selected all the natural numbers,
the least one will be $\omega^1 \cdot 1$. The $\prec '$ least one
in $\T \restriction \{ 0,\S(0),\S(\S(0)) ,\ldots \} \cup \{ \omega^1 \cdot 1 \}$
not selected before will be $\omega^1 \cdot 1 + \omega^0 \cdot 1$ and 
one sees easily that we  will proceed selecting
systematically all ordinals built by the Cantor normal form, 
i. e. all notations below $\varphi_00$.\par 
The next ordinal will selected will be $\varphi_00$. And here we will make for
the first time a jump in the $\prec'$-ordering: there are notations 
$\prec'\varphi_00$ we
have not chosen yet, but they are built  from notations which
we have not constructed before. After $\varphi_00$ we will choose
$\omega^{\varphi_00} \cdot 1 + \omega^0 \cdot 1$ --- we 
will therefore move downwards with respect to $\prec'$ ---
and then we will exhaust again the  Cantor normal form and choose
all ordinals below $\varphi_01$.
Now $\varphi_01$ will be selected and we will
then proceed selecting (with the steps for building
the Cantor normal form in between) 
$\varphi_02$, $\varphi_03 ,\ldots,\varphi_0(\varphi_00) ,\ldots, 
\varphi_0(\varphi_0(\varphi_00)) ,\ldots$,
then $\varphi_10$. etc.\par 
In order to be able to select the $\prec'$-least element of the
sets considered, we need that the set of 
notations with components in the notations previously selected is
$\prec'$-well-ordered. Now by (OS 4) this will be the case, if  we have that
the sequence of ordinals previously selected is $\prec$-increasing
and therefore $\prec$-well-ordered. Now we can easily
see by induction over the process above that, whenever we select an ordinal, we select in fact the
$\prec$-least one not chosen before and therefore the set of ordinals
previously chosen will be a $\prec$-segment of $\T$:\par 
Assume $A$ is the ordinals previously selected, which is by IH a segment
of the ordinals and let $a$ be the new notation chosen. $a$ will be bigger
than $A$, since $A$ is a segment. We need to show, which we will do
by induction over the length of $b$,  that, if $b \prec a$, $b \in A$.
If $b \prec a$, then $b \preceq \k(a) \subseteq A$, so $b \in A$ by
$A$ being a segment, or $b \prec ' a$.
$b$ is built from smaller components, they must be by side-IH all in $A$,
therefore $b$ has components in $A$, $a$ was the $\prec'$-least such 
not selected before, so $b$ must have been selected before, $b \in A$.\par 
Now we assume that ``we do not run out of ordinals''. Therefore we
must  reach a point, where we cannot  select an ordinal any more. But then
one sees easily that all notations must be selected, so the set of 
notations must be well-ordered (since we have selected them in
increasing order) and we are done.\par 
The argument ``we do not  run out of ordinals'', can now be  
formalized in set theory for elementary ordinal systems by using  
$\omega_1^\ck$: if we could select as above $\omega_1^{\ck}$ 
ordinals, we would have a primitive recursive well-ordering
of order type $\omega_1^{\ck}$, a contradiction. For this argument
we need the  big ordinal $\omega_1^\ck$.
However it is only needed to provide an ordinal ``big enough''. In
fact, the process of selecting ordinals will terminate after $\alpha$
steps for some $\alpha$ which is below $\omega_1^\ck$. Something similar
will be the case for all recursively large ordinals needed for 
stronger ordinal notation systems: they always play only the role
of ordinals ``big enough'', in fact, all processes involved will terminate
after $\alpha$ steps for some ordinal $< \omega_1^\ck$.\par 
We are now going to formalize the well-ordering argument above rigorously,
in order to get absolute precision. We will
then show that for elementary ordinal systems, it can be formalized
in Kripke-Platek set theory (with natural numbers as urelemente). 
Therefore we will get an upper bound for the strength of elementary
ordinal systems.
%u
%The argument is essentially as follows: Define by recursion over $\alpha$ 
%$a_\alpha \in \T$ as long as possible by:
%$a_\alpha$ is the $\prec'$-least element of $\T \restriction\{a_\beta \mid
%\beta < \alpha \}$, which has not been chosen as an $a_\gamma$ before.
%This works, if $a_\beta$ are increasing and therefore well-ordered, since 
% then 
%$\T \restriction\{a_\beta \mid \beta<\alpha \}$ is
%$\prec '$-well-ordered. $a_\gamma$ will actually be an
%increasing sequence. 
%This process, which is some kind of 
%game played between player $(\T,\prec)$ and $(\T,\prec ')$
%(it is not a real game since the players cannot make any decisions),
%will eventually stop for elementary OS at some
%$\gamma < \omega_1^\ck$, 
%$\{ a_\alpha \mid \alpha< \gamma \}$ is well-ordered and exhausts
%$\T$, therefore $\T$ is well-ordered.\par 
%
%
%
%%%%%%%%%%%%%%%%%%%%%%%%%%%%%%%%%%%%%%%%%%%
%%
%                                         %
%%%%%%%%%%%%%%%%%%%%%%%%%%%%%%%%%%%%%%%%%%%
%
%
%
\begin{lemma}
\laprint{lemintuitos}
\begin{enumerate}
\ssubitem a Let $\calF$ be an ordinal system,  and
assume $\bot \not \in \T$. 
By recursion on ordinals $\gamma$  we define
$a_\gamma \in \T \cup \{ \bot \} $ as follows:\\
Let $A_{<\gamma} := \{ a_\alpha \mid \alpha < \gamma \} $.
Define ``$A_{<\gamma}$ is increasing'' iff
$A_{<\gamma} \subseteq \T$ and 
$\all \alpha,\beta < \gamma(\alpha < \beta \ar a_\alpha \prec a_\beta)$.\\ 
$a_\gamma := \begin{cases}
\min_{\prec '}((\T \restriction A_{<\gamma}) \setminus
A_{<\gamma})&\text{if
$A_{<\gamma}$ is increasing and  $\T  \restriction A_{<\gamma} \not \subseteq 
A_{<\gamma }$,}\\
\bot&\text{otherwise.}
\end{cases} $\\
Then there exists a $\gamma$ such that
$A_{<\gamma}$ is increasing and $A_{<\gamma} = \T$. Therefore $(\T,\prec)$ 
is well-ordered.
\ssubitem c $\gamma $ as in \refact a is $< \omega^{\ck}$,
if $\calF$ is primitive recursively represented.
\ssubitem b If $\calF$ is elementary, \refact a can be 
formulated in  $\kpomega$.
\end{enumerate}
\end{lemma}
%
%
%
{\bf Proof:}\\
\refact a: The definition is well-defined, since, if
$A_{<\gamma}$ is increasing, $\T \restriction A_{<\gamma}$ is well-ordered
with respect to $\prec'$, therefore as well
$(T \restriction A_{<\gamma}) \setminus A_{<\gamma}$
and $a_\gamma$ can be defined.\\
We show: If $A_{<\gamma}$ is increasing, then $A_{<\gamma}$ is an initial
segment of $\T$ w.r.t. $\prec$ by recursion on $\gamma$:
The cases $\gamma =0$ or $\gamma$ limit ordinal follow trivially or 
by IH. Let now $\gamma = \delta+1$. We need to show
$a_\delta$ is the $<$-least element in $\T \setminus A_{<\delta}$:
$a_\delta \not \in A_{<\delta}$, which is an initial segment of $\T$,
therefore $A_{<\delta} \prec a_\delta$. We show by induction on
$\length(b)$ $\all b \prec a_\delta.b \in A_{<\delta}$, from which the
assertion follows. $b \prec a_\delta$, therefore
$b \preceq\k(a_\delta) \subseteq A_{<\delta} \segment \T$ and
therefore $b \in A_{<\delta}$ or
$b \prec ' a_\delta$. In the last case, by IH, since
$\k(b) \prec b \prec a_\delta$, $\k(b) \subseteq A_{<\delta}$,
$b \in \T \restriction A_{<\delta}$, by $\prec '$-minimality of $a_\delta$
$b \not \in (\T \restriction A_{<\delta}) \setminus A_{<\delta}$,
$b \in A_{<\delta}$.\\ 
Let $\kappa$ be an admissible ordinal such that $\calF \in \L_\kappa$
(if $\calF$ is primitive recursively represented, 
$\kappa$ can be chosen as $\omega_1^\ck$).
If $A_{<\kappa}$ were now increasing, then
$A_{<\kappa} = \{ a \in \T \mid a \prec a_{\kappa+1} \}$ would
be a well-ordering, which is an element of $\L_\kappa$ and
has order type $\kappa$,
a contradiction. Therefore there exists a least $\gamma<\kappa$ such that 
$A_{<\gamma}$ is not increasing. $\gamma = \gamma_0+1$,
$A_{<\gamma_0}$ is increasing, therefore a well-ordered subset
of $\T$. $\T \restriction  A_{<\gamma_0} \subseteq A_{<\gamma_0}$, therefore
by induction on $\length(a)$ follows for  all $a \in \T$
$a \in A_{<\gamma_0}$, $A_{<\gamma_0} = \T$ and the assertion.\\ 
\refact c: by the proof of \refact a.\\
\refact b: The formalization is straight-forward, the only difficulty
is the argument referring to $\kappa$, which we replace by the 
following: Let $C:= \{ \gamma \in \Ord \mid A_{<\gamma}$ increasing
$\}$,   $C \segment \Ord$, $f: C \ar \T$, $f(\alpha):= a_\alpha$.
$f[C]$ is a well-ordered initial segment of $\T$. Assume
$f[C] \not = \T$. Then $
(\T \restriction f[C])\not \subseteq f[C]$, let
$a \in (\T \restriction f[C]) \setminus f[C]$ be $\prec'$ minimal. (Here we use
reducibility of transfinite induction). 
As in the argument before follows $a$ is the least
element of $\T$ not in $f[C]$, therefore
$f[C] = \{ b \in \T \mid b \prec a \}$, a set, $f: \Ord \ar f[C]$ bijective,
$f^{-1}: f[C] \ar \Ord$ bijective, therefore $\Ord = f^{-1}[f[C]]$ is a 
set, a contradiction. Therefore $f[C]= \T$ is well-ordered.\QED 
%{\bf Intuitive Argument while OS are well-ordered.} Since we want to 
%have ordinal notation systems, which are based on OS, as another
%justification for the consistency of systems which can be analyzed used here,
%we need some intuitive argument, while OS are well-ordered. The
%best we can do is the following argument: Assume the following process:
%Enumerate systematically new ordinal notations, by taking always the
%(with respect to $\prec '$) least element of $\Arg$, which
%we have not taken before and the components of which are in the notations
%we have already generated, and apply $f$ to it. This can be
%done as long as the set of notations chosen is always well-ordered,
%which is the case, if the result of our process is an
%increasing sequence. But this is the case, if we select
%$s \in \T$, by our definition all elements
%of $\C(s)$ must have been selected before $s$,
%so all notations below $s$ have been selected before, and we are done.
%Slightly more formally, define by
%recursion on $\alpha$ $g(\alpha):= \min_{\prec '} \{ a \in \Arg \mid 
%a \not \in g[\alpha] \land \k(a) \subseteq f[g[\alpha]] \}$.
%%Assume for some $\alpha<\beta$ we have
%%$f(g(\beta)) \preceq f(g(\alpha))$, and assume 
%%$\beta$ is chosen minimal, 
%%$\alpha$ is chosen minimal relative to $\beta$. 
%%If $f(g(\beta))  \prec f(g(\alpha))$, then
%%$f(g(\beta)) \preceq k(g(\alpha)) \subseteq f[g[\alpha]]$,
%%$f(g(\beta)) \preceq f(g(\alpha '))$ with $\alpha ' < \alpha$,
%%contradicting the minimality of $\alpha$. Therefore
%%$g(\beta) \prec ' g(\alpha)$, $k(g(\beta)) \prec g(\beta) \prec 
%%g(\alpha)$. By minimality of $\beta$ and $k(g(\beta)) \subseteq f[g[\beta]]$
%%follows $k(g(\beta)) \subseteq f[g[\alpha]]$. But then
%%$f(\alpha)$ was not chosen $\prec '$-minimal.\par
%
%
%
%%%%%%%%%%%%%%%%%%%%%%%%%%%%%%%%%%%%%%%%%%%
%%
%                                         %
%%%%%%%%%%%%%%%%%%%%%%%%%%%%%%%%%%%%%%%%%%%
%
%
%
\begin{theorem}
\laprint{theostrengthordsys}
The order type of elementary ordinal systems is less than the
Bachmann-Howard ordinal.
\end{theorem}
%
%
%
{\bf Proof:} by Lemma \refcom{lemintuitos}b and since $|\kpomega|$ is
the Bachmann-Howard ordinal.\QED 


\subsection{Constructive Well-ordering Proof}
We regard the above well-ordering proof as rather intuitive. 
However, for treating intuitionistic theories we will need 
a constructive  argument as well. This argument will as well
be shorter and can be formulated for instance in
intuitionistic $\ID_1$ or type theory having the accessible
part or one unnested $\mathrm{W}$-type, which yields again
the upper bound Bachmann-Howard for the order type
of elementary OS.\par 
%
%
{\bf Constructive well-ordering proof:} Let $\Acc:= \Acc_\prec(\T)$ 
and $\Acc ':= \k^{-1}(\Acc)$, which is well-ordered w.r.t.
$\prec '$.
We show $\all t \in \Acc'.t \in \Acc$ by 
induction on $t \in \Acc'$. It suffices to show
$\all s \prec t.s \in \Acc$ by side-induction
on $\length(s)$.
If $s \preceq \k(t)$, then $s \in \Acc$ since
$\k(t) \subseteq \Acc$. Otherwise
$s \prec ' t$, $\k(s) \prec s \prec t$.
By side-IH $\k(s) \subseteq \Acc$ and by main-IH
therefore $s \in \Acc$ and we are done. Now
follows by induction on $\length(s)$
$\all s \in \T.s \in \Acc$,
$\T$ is well-ordered.
%{\bf \refact b} The argument above can be shown in
%$\ID_1$, therefore the ordertype of $\calF$ is
%below the proof theoretical strength of $\ID_1$, which
%is BHO.\\
%
%
%
%%%%%%%%%%%%%%%%%%%%%%%%%%%%%%%%%%%%%%%%%%%
%%
%                                         %
%%%%%%%%%%%%%%%%%%%%%%%%%%%%%%%%%%%%%%%%%%%
%
%
%
\subsection{Ordinal Function Generators.}

We are going to define elementary 
ordinal systems, which exhaust the strength of this concept. 
As in the introduction, we
want to get functions, which act on ordinals as well. This does not
require much extra work. As we said there, one can easily develop this
without any reference to ordinals.
The advantage of our approach is that we will get 
definitions for the usual ordinal functions like
$+$, $\lambda \alpha.\omega^\alpha$, $\varphi$ and the Sch{\"u}tte
Klammer symbols and further new definitions for 
extensions of the Klammer symbols without much extra work.\par 
%This will be
%done in a purely syntactic way. However, we want to obtain a little bit more:
%the functions, which transfers arguments into ordinals, should be defined 
%not only for ordinal notations but for general set theoretical ordinals.
%This generalization will require almost no extra work, 
%and we can will get definitions for the usual ordinal functions like
%$+$, $\lambda \alpha.\omega^\alpha$, $\varphi$ the Sch{\"u}tte
%Klammer symbols and as well new definitions for 
%extensions of the Klammer symbols.\par 
%We can see easily that with almost the
%same definitions one can obtain directly in a purely syntactical
%and as well natural way the corresponding ordinal systems without
%any need to refer to ordinals (not even for heuristic purposes).\par 
There are typically two kinds of ordinal functions usually used: 
Versions having
fixed points and fixed point free versions. One example is
the Veblen function versus its fixed point free version (see for instance 
the function $\psi$ in \cite{schuettebook}, p. 84). 
When using versions with fixed points, we will usually not have the 
property, that the arguments of an ordinal functions
are strictly smaller than the value of it. This is no harm, since
in a next step one selects normal forms.
The following definition allows to introduce both fixed point versions and
fixed point free versions:
%
%
%
%%%%%%%%%%%%%%%%%%%%%%%%%%%%%%%%%%%%%%%%%%%
%%
%                                         %
%%%%%%%%%%%%%%%%%%%%%%%%%%%%%%%%%%%%%%%%%%%
%
%
%
\begin{definition}
\laprint{defordgen}{\rm
\begin{enumerate}
\ssubitem a An {\em ordinal function generator}, in short
{\em OFG} is a quadruple
$\calO:= (\Arg,\k,\lrm,<')$, where $\Arg$ is a class (in set theory),
$\k,\lrm: \Arg \ar_\omega \Ord$ such that
$\all a \in \Arg.\lrm(a)\subseteq \k(a)$ 
and $<'$ is a well-ordered relation on $\Arg$. We define
$<'_\calO:= <'$.
\ssubitem b If $\calO$ is as above, we define by recursion on 
$a \in \Arg$ simultaneously sets $\C_\calO(a):= \C(a) \subseteq \Ord $ and 
$\eval_\calO(a):= \eval(a) \in \Ord$:
$\C(a):= \bigcup_{n < \omega}\C^n(a)$, 
where  $\C^0(a):= (\bigcup \k(a)) \cup  \lrm(a) $,
$\C^{n+1}(a):= \C^n(a) \cup \{\eval(b) \mid 
b \in \Arg, b<' a,\k(b) \subseteq \C^n(a) \}$.\\
$\eval(a):= \min \{\alpha \in \Ord \mid \alpha \not \in \C(a) \} $.
\ssubitem c Define for $\calO$ as above,
$\NF:= \{ a \in \Arg \mid \k(a) < \eval(a) \}$.
%\ssubitem d Define for $\calO$ as above
%$\Cl' \subseteq \Arg$ by:
%If  $a \in \T$, $\k(a) \subseteq \eval[\Cl]$, then
%then $a \in \Cl'$.\\
%Define $k': \Cl' \ar_\omega  \Cl'$, $\k'(a):= \eval^{-1}[\k(a)]$.\\
%For $a \in \Cl$ define $\length(a)$ by
%$\length(a):= \max (\length[k'(a)])$ (we will need here
%that $\eval$ is injective as shown in the next Lemma).\\
%Define for $a,b \in \Cl$ $a \prec b \Iff \eval(a) < \eval(b)$.
\ssubitem d Define for $\calO$ as above
$\Cl \subseteq \NF$ inductively by:
If  $a \in \NF$, $\k(a) \subseteq \eval[\Cl]$, 
then $a \in \Cl$.\\
$\Arg[\Cl]:= \{ a \in \Arg \mid \k(a) \subseteq \eval[\Cl] \}$.
Note that $\Cl \subseteq \Arg[\Cl]$.\\
Assuming $\eval \restriction \NF$ is injective, which will be 
shown later, we define $\k^0: \Arg[\Cl] \ar_\omega  \Cl$ and
$\length: \Arg[\Cl] \ar \nat$ by\\ 
$\k^0(a):= \eval^{-1}[\k(a)] \cap \Cl$,
$\length(a):= \max(\length[\k^0(a)] \cup \{ -1 \}) +1$\\ 
for $a \in \Arg[\Cl]$
with $-1 < n$ for $n \in \nat$, $-1+1:=0$.\\ 
Further let for $a,b \in \Arg[\Cl]$ $a \prec b :\Iff \eval(a) < \eval(b)$.
\end{enumerate}}
\end{definition}
%
%
%
Note that the definition of $\eval(a)$ expresses something similar as for
OS: When defining $\eval(a)$, we know all $\alpha< \eval(a)$, since
$\alpha < \k(a)$, $\alpha \leq \lrm(a)$, or $\alpha$ was introduced 
before $a$, i.e. $\alpha = \eval(b)$ for some $b<' a$. Further, if
$a$ is in normal form, then $a$ is a notation for 
$\eval(a)$ referring
to smaller  ordinals only, namely $\k(a)$.
%
%
%
%%%%%%%%%%%%%%%%%%%%%%%%%%%%%%%%%%%%%%%%%%%
%%
%                                         %
%%%%%%%%%%%%%%%%%%%%%%%%%%%%%%%%%%%%%%%%%%%
%
%
%
\begin{lemdef}
\laprint{lemordgen}
Let $\calO$ be as above.
\begin{enumerate}
\ssubitem a $\C(a)$ is the least set such that 
$(\bigcup \k(a)) \cup \lrm(a) \subseteq \C(a)$ and, if 
$b \in \NF$, $b <' a$, 
$\k(b) \subseteq \C(a)$, then $\eval(b) \in \C(a)$.
\ssubitem b $\all a \in \Arg(\k(a) \leq \eval(a) \land \lrm(a)< \eval(a))$.
\ssubitem c $\C(a)$ is an initial  segment of $\Ord$.
\ssubitem d $\eval$ restricted to $\NF$  is injective.
\ssubitem e $\eval[\Cl]$ is an initial segment of $\Ord$.
\ssubitem{eb} For $a,b \in \Arg[\Cl]$ it follows
\begin{eqnarray*} 
\eval(a) < \eval(b) &\Iff& (a <' b \land \k(a) < \eval(b))
\lor \eval(a) < \k(b) \lor \eval(a) \leq \lrm(b)\\
\eval(a) = \eval(b) &\Iff& (a <' b \land \eval(b) = \max( \k(a)) \land
\eval(b) \not \in \lrm(a)) \lor \\
&& (b <' a \land \eval(a) = \max(\k(b)) \land \eval(a) \not \in \lrm(b)) \lor \\
&&a=b
\end{eqnarray*} 
\ssubitem {ea} For $a,b \in \NF$ we have
$\eval(a) < \eval(b) \Iff (a <' b \land \k(a) < \eval(b))
\lor \eval(a) \leq \k(b)$.
\ssubitem f $\calF:= (\Cl,\prec,<',\k^0,\length)$ is an ordinal system.
We will call any OS-structure  isomorphic to $\calF$ 
an {\em ordinal system} based on $\calO$.
%\ssubitem g Let $f: \T \ar \Cl$ be a bijection, and
%$\k_1: \T \ar_\omega \T$, $\prec ' \subseteq \T \cross \T and
%$\length_1: \T \ar \nat$ defined such that 
%for $a,b \in \T$,  $f[\k_1(a)] = \k(f(a))$
%$a \prec' b \iff f(a)\prec f(b)$ and $\length(a) := \length(f(a))$.
\end{enumerate}
\end{lemdef}
%
%
%
{\bf Proof:}\\
\refact a, \refact b: easy.\\ 
\refact c Note that, if $\C(a)$ is an initial segment, 
$\eval(a) = \C(a)$. Proof of the assertion by induction on $a$. We show
$\all \alpha \in \C(a).\alpha \subseteq \C(a)$ by induction on the 
definition of $\C(a)$. If $\alpha  \in \bigcup \k(a)$, then
$\alpha \subseteq \C(a)$. If $\alpha \in \lrm(a) \subseteq \k(a)$,
$\alpha \subseteq \bigcup \k(a) \subseteq \C(a)$. 
If $\alpha = \eval(b)$, $b <' a$, $\k(b) \subseteq \C(a)$,
by side-IH $\bigcup \k(b) \subseteq \C(a)$, and by transitivity
of $<'$ and $\lrm(a) \subseteq \k(a)$ 
follows easily $\C(b) \subseteq \C(a)$ and by main IH
$\alpha = \C(b) \subseteq \C(a)$.\\
\refact d  Assume $a \not = b$, $a,b \in \NF$, $\eval(a) = \eval(b)$.
By linearity of $<'$, $a <' b$ or $b <'  a$, so w.l.o.g. $a <' b$.
$\k(a) < \eval(a) = \eval(b)$, therefore $\k(a) \subseteq \C(b)$,
$a <' b$, therefore $\eval(a) \in \C(b) = \eval(b)$, a contradiction.\\ 
\refact e We show $\all a \in \Cl.\eval(a) \subseteq \eval[\Cl]$ by induction
on $a \in \Cl$: Using IH, \refact a and
$\neg \NF(b) \ar \eval(b) \in \k(b)$ it follows 
$\C(a) \subseteq \eval[\Cl]$, $\eval(a) \subseteq \eval[\Cl]$.\\ 
\refact {eb} First formula ``$\Leftarrow$'': $\eval(a) \in \C(b) = \eval(b)$.\\
Second formula ``$\Leftarrow$''
consider only the case 
$(a <' b \land \eval(b) = \max (\k(a)) \land
\eval(b) \not \in \lrm(a))$.
$\eval(b) \in \k(a) \leq \eval(a)$.
Assume $\eval(b) < \eval(a)$. Then, since $\eval(b) \not < \k(a)$,
$\eval(b) \not \leq \lrm(a)$, by $\eval(b) \in \C(a)$ it follows
$\eval(b) = \eval(b')$ for some $b' <' a$,
$\k(b') \subseteq \C(a)$. By the definition of $\C(a)$ and
$\eval(b) \in \C(a)\setminus \C^0(a)$ there
exists $b' <' a$ and $n$ such that $\k(b') \subseteq \C^n(a)$,
$\eval(b') \in \C^{n+1}(a) \setminus \C^n(a)$ and $\eval(b') = \eval(b)$.
$b' <' b$, $\eval(b') = \eval(b)$, so $\k(b') \not < \eval(b)
= \eval(b')$, $\eval(b') \in \k(b')$, contradicting the choice of 
$b'$. Therefore $\eval(b) = \eval(a)$.\\
``$\Rightarrow$'' first formula: if the right
hand side is false 
the right hand holds for $\eval(a) = \eval(b)$ or $\eval(b) < \eval(a)$,
therefore the left hand side is false. For the second formula
``$\Rightarrow$'' follows in the same way.\\
\refact {ea} ``$\Leftarrow$'' is immediate, and
``$\Rightarrow$'' follows as before.\\
\refact f $(\OS\ 1)$, $(\OS\ 2)$, $(\OS\ 4)$ are clear.
$(\OS\ 3)$: If $s \prec t$, $s,t \in \NF$, 
then $\eval(s) \in \C(t)$, $\eval(s) \leq 
\k(t)$ or $s < ' t$ and the assertion.\QED
%
%
%

%
%
%%%%%%%%%%%%%%%%%%%%%%%%%%%%%%%%%%%%%%%%%%%
%%
%                                         %
%%%%%%%%%%%%%%%%%%%%%%%%%%%%%%%%%%%%%%%%%%%
%
%
%
\begin{definition}
\laprint{defcombineordfun}{\rm
If $\calO_i=(\Arg_i,<_i',\k^i,\lrm^i)$ are OFG
($i=0,1$), $\Arg_0 \cap \Arg_1 = \emptyset$,
Let $\calO_0 \preccirc \calO_1:= (\Arg,<',\k,\lrm)$,
be defined by $(\Arg,<'):= (\Arg_0,<_0') \preccirc  (\Arg_1,<_1')$ and
$\k(r):= \k^i(r)$, $\lrm(r):= \lrm^i(r)$ for $r \in \Arg_i$.
$\calO_0 \preccirc \calO_1$ is obviously an OFG and
$\eval_\calO(a)= \eval_{\calO_0}(a)$ for $a \in \Arg_{\calO_0}$.}
\end{definition}
%
%
%
%%%%%%%%%%%%%%%%%%%%%%%%%%%%%%%%%%%%%%%%%%%
%%
%                                         %
%%%%%%%%%%%%%%%%%%%%%%%%%%%%%%%%%%%%%%%%%%%
%
%
%
\begin{example}
\laprint{exmplordfun}{\rm
\begin{enumerate}
\ssubitem a Let $\alpha \in \Ord$ be fixed,
$(\Arg_\alpha ,<'_\alpha ):= \alpha \plusbar \Ord$ (i.e.
expanding Definition \refcom{notations}c, 
$\Arg_\alpha= \{ \alpha \plusbar \beta \mid \beta \in \Ord \}$,
$\alpha \plusbar \beta <'_\alpha  \alpha \plusbar \gamma \Iff
\beta < \gamma$; here and in the following $\alpha \plusbar \beta$,
$\phibar_\alpha \beta$ etc. are formal terms defined from 
ordinals $\alpha$, $\beta$, etc. coded in set theory in some way). 
Let
$\k(\alpha \plusbar \beta) := \{ \alpha,\beta \}$,
$\lrm(\alpha \plusbar \beta):= \mycases{\emptyset &\text{if $\beta=0$,}\\
\{ \alpha \} &\text{otherwise,}}$
$\calO_\alpha:= (\Arg_\alpha,<'_\alpha,\k,\lrm)$. 
Then $\eval(\alpha \plusbar \beta) = \alpha + \beta$. 
Note that $\Cl = \emptyset$.\\
(We could change the definition of OFGs by
omitting the condition ``$<'$ linear''. Then Lemma \refcom{lemordgen}a
- \refsub{lemordgen}c and \refsub{lemordgen}e 
go through as well and one could define 
$(\Arg,<')$ as the union of $(\Arg_\alpha ,<'_\alpha)$ with
$\alpha \plusbar \beta$ and $\gamma \plusbar \rho$ incomparable
for $\alpha \not = \gamma$.
$\eval(\alpha \plusbar \beta) = \alpha + \beta$, so this way we would get
a definition of the full function $+$.)
%
%
\ssubitem b Let $(\Arg,<) := \Sigmabar(\APN_\weakdes)$,
$\k(\Sigmabar(\alpha_1 ,\ldots, \alpha_m)):= 
\{ \alpha_1 ,\ldots, \alpha_m \}$,
$\lrm(\Sigmabar(\alpha_1 ,\ldots, \alpha_m))$\\
$:= \mycases{\{ \alpha_1 \} &
\text{if $m>1$,}\\ \emptyset&\text{otherwise.}}$
Let $\calO_{+,\wrm}$ be the resulting OFG. Then one easily
verifies $\eval(\Sigmabar()) = 0$ (similar cases occurring
in future examples will  not be mentioned below)
$\eval(\Sigmabar(\alpha_1 ,\ldots, \alpha_m)) = 
\alpha_1 +\cdots + \alpha_m$ for $m>0$. 
The fixed point free version of it, i.e. $\lrm(t) := \k(t)$ yields
the same result, except 
$\eval(\Sigmabar(\alpha,\underbrace{1 ,\ldots, 1}_{l}))= 
\alpha + l+1$
\ssubitem c Let $(\Arg,<):= \Sigmabar'(\schuette(\APN,\omega \setminus \{ 0 \} ))$,\\
$\k(\Sigmabar'(\alpha_1,n_1 ,\ldots, \alpha_m,n_m)):=
\{ \alpha_1 ,\ldots, \alpha_m,n_1,\ldots, n_{m-1},n_m-1 \}$,\\
$\lrm(\Sigmabar'(\alpha_1,n_1 ,\ldots, \alpha_m,n_m)):=
\mycases{\{ \alpha_1,n_m-1\} &
\text{if $m>1$ or $n_m> 1$,}\\ \emptyset&\text{otherwise.}}$\\
Let $\calO_+$ be the resulting OFG. 
$\eval_{\calO_+}(\Sigmabar'(\alpha_1,n_1 ,\ldots, \alpha_m,n_m)) = 
(\alpha_1 \cdot n_1) +\cdots+ (\alpha_m \cdot n_m)$.
\ssubitem d If we replace in \refact b, \refact c 
$\APN$ by $\Ord$ and $\Sigmabar$ by $\CNF$, we obtain
ordinal function generators $\calO_{\CNF,\wrm}$,
$\calO_{\CNF}$ for the 
Cantor normal form, i.e. $\eval_{\calO_{\CNF,\wrm}}
(\CNF(\alpha_1 ,\ldots, \alpha_m)) = 
\omega^{\alpha_1} +\cdots+ \omega^{\alpha_m}$
and 
$\eval_{\calO_{\CNF}}
(\CNF'(\alpha_1 ,n_1,\ldots, \alpha_m,n_m)) = 
\omega^{\alpha_1}\cdot n_1  +\cdots+ \omega^{\alpha_m} \cdot n_m$
\ssubitem e Let 
$\calO_0$ be $\calO_{+,\wrm}$ or $\calO_+$,
$\calO_1:= (\omegabar^\Ord,\k,\lrm)$ with
$\k(\omegabar^\alpha):= \{ \alpha \}$,
$\lrm(\omegabar^\alpha):= \emptyset$,
$\calO:= \calO_0 \preccirc \calO_1$,
then $\eval(\omegabar^\alpha)= \omega^\alpha$. 
The fixed point free version ($\lrm(\omegabar^\alpha):= \{ \alpha \}$)
yields the same result, except
$\eval(\omegabar^{\alpha +n})= \omega^{\alpha+n+1}$ for 
epsilon numbers $\alpha$.
\ssubitem f Let 
$\calO_0$ be $\calO_{+,\wrm}$ or $\calO_+$,
$\calO_1:= (\phibar_\Ord\Ord,\k,\lrm)$,
$\k(\phibar_\alpha \beta):= \{ \alpha,\beta \}$,\\
$\lrm(\phibar_\alpha \beta):= 
\mycases{\{ \alpha \} &\text{if $\beta > 0 $,} \\
\emptyset&\text{otherwise,}}$
$\calO:= \calO_0 \preccirc \calO_1$.
Then $\eval(\phibar_\alpha \beta) = \varphi_\alpha \beta$,
where $\varphi$ is the Veblen function with $\varphi_0 \beta:= \omega^\beta$.
If we use the OFG from \refact d or \refact e instead for $\calO_0$ we 
obtain the $\varphi$-function starting with $\varphi_0 \alpha:= \epsilon_\alpha$,
and defining $\lrm(t):= \k(t)$ yields the fixed point free version
of the Veblen function.\par 
\ssubitem g Let $\calO_0$ be $\calO_{+,\wrm}$ or $\calO_+$,
$\calO_1:= \phibar(\schuette{\Ord  \setminus \{ 0 \} \choose \Ord})$.
If $A=  
{\beta_1 \cdots \beta_m \choose \alpha_1 \cdots \alpha_m}$,
define\\
%\left(\begin{array}{ccc}\beta_1& \cdots&\beta_m \\
%\alpha_1  &\cdots &\alpha_m \end{array} \right)$, 
$\k(\phibar(A)):= 
\{ \beta_1 ,\ldots, \beta_m,\alpha_1 ,\ldots, \alpha_m \}$ and\\
$\lrm(\phibar(A)):= 
\{ \beta_1 ,\ldots, \beta_{m-1},\alpha_1 ,\ldots, \alpha_{m-1}\} \cup 
\mycases{ \{ \alpha_m \} &\text{if $\beta_m>1$,}\\
\emptyset&\text{otherwise,}}$\\
$\calO:= \calO_0 \preccirc \calO_1$.\\
As in \cite{Schuette54} we allow addition of 
columns $0 \choose \alpha$ and
identify matrices which differ in such columns only.
If $A= 
{\beta_1 \cdots \beta_m \choose \alpha_1 \cdots \alpha_m}$,
%\left(\begin{array}{ccc}\beta_1& \cdots&\beta_m \\
%\alpha_1  &\cdots &\alpha_m\end{array} \right)$,
$(A,\beta_{m+1},\alpha_{m+1}):
= {\beta_1 \cdots \beta_m \beta_{m+1} \choose 
\alpha_1 \cdots \alpha_m \alpha_{m+1}}$,\\
%\left(\begin{array}{cccc}\beta_1& \cdots&\beta_m&\beta_{m+1}   \\
%\alpha_1  &\cdots &\alpha_m&\alpha_{m+1}\end{array} \right)$,\\
$(A,\beta_{m+1},\alpha_{m+1},\beta_{m+2},\alpha_{m+2})$
etc. are defined similarly.\\ 
%If $A = \left(\begin{array}{cccc}\beta_1& \cdots&\beta_m&\beta_{m+1}\\
%\alpha_1  &\cdots &\alpha_m&0 \end{array} \right)$,
%$A':= \left(\begin{array}{cccc}1+\beta_1& \cdots&1+\beta_m&\beta_{m+1}\\
%\alpha_1  &\cdots &\alpha_m&0 \end{array} \right)$.\\
Then $\eval(\phibar(A)) = \varphi(A)$,
where $\varphi(A)$ is defined as in \cite{Schuette54}, based
on $\varphi(\alpha):= \omega^\alpha$, but
with reversed order of the columns. We prove this by 
induction on $<'_\calO$:\\ 
If $A = \phibar(\alpha,0)$, then we can easily show
$\eval(A) = \omega^\alpha= \varphi(\alpha,0)$.\\ 
If $A= \phibar(B,\alpha,\beta,\gamma,0)$, then it follows that 
$\C(\phibar(A))$ is closed under $+$,\\
$\lambda \delta.\eval(\phibar(B,\alpha^\ast,\beta,\delta,\beta^\ast))$
for $\alpha^\ast<\alpha$, $\beta^\ast<\beta$ and 
contains further\\ 
$\eval(\phibar(B,\alpha,\beta,\gamma^\ast,0))$ for $\gamma^\ast<\gamma$,
therefore using the IH 
$\eval(\phibar(B,\alpha,\beta,\gamma,0))\geq \varphi(B,\alpha,\beta,\gamma,0)$.
On the other hand, $\varphi(A) \geq \k(\phibar(A))$,
$\varphi(A)> \lrm(\phibar(A))$, and, if $B <' A$, $\k(\phibar(B))< 
\varphi(A)$, 
by the calculations in \cite{Schuette54}
$\varphi(B) < \varphi(A)$, therefore  using the IH
$\all \gamma \in \C(\phibar(A))(\gamma < \varphi(A))$, 
$\varphi(A) \geq \eval(\phibar(A))$.
\ssubitem h If we define in \refact g $\lrm(\phibar(A)):= \k(\phibar(A))$,
we obtain a fixed point free version of the 
Klammer symbols or equivalently
$\eval(\phibar 
{\beta_1 \cdots \beta_m \choose \alpha_1 \cdots \alpha_m})= 
%\left(\begin{array}{ccc}\beta_1& \cdots&\beta_m\\
%\alpha_1  &\cdots &\alpha_m\end{array} \right) = 
\vartheta(\Omega^{\alpha_1}\beta_1 +\cdots+ \Omega^{\alpha_m}\beta_m)$,
where $\vartheta$ is defined as in \cite{rathjenweiermann}, but
without closing $\C(\alpha,\beta)$ under $\lambda \gamma.\omega^\gamma$.
(We obtain the original $\vartheta$ function, if we take as
$\calO_0$ the OFG for $\CNF$). See \cite{seisenbdipl} for details.
If we restricting the definition to those $\alpha_i,\beta_i$ such that
for $\gamma:= \Omega^{\alpha_1}\beta_1 +\cdots+ \Omega^{\alpha_m}\beta_m$
$\gamma \in \C(\gamma)$, where $\C(\gamma)$ is defined as for the 
ordinary $\psi$-function, 
then we will get the same result, but with $\vartheta$ replaced
by the $\psi$-function ($\psi = \psi_0$ as in \cite{buchholzanewsys}
or $\psi= \psi_{\Omega_1}$ as in \cite{buchholzasimplified}, 
however with $\C(\alpha,\beta)$ not closed under $\varphi$).
Note that this restriction has the effect that 
$\prec'$ and $\prec$ coincide in the corresponding
OS for terms of the form $\phibar(a)$.
\ssubitem i Let $\calS_0:= \Ord$,
$\calS_{n+1}:= \schuette{\Ord \setminus \{ 0 \}  \choose \calS_n}$,\\
$\k^i_0: \calS_i \ar_\omega \Ord$,
$\k^0_0(\alpha):= \{ \alpha \}$,
$\k^{i+1}_0 
{\beta_1 \cdots \beta_m \choose A_1 \cdots A_m}:=
%\left(\begin{array}{ccc}\beta_1& \cdots&\beta_m\\
%A_1  &\cdots &A_m\end{array} \right):= 
\{ \beta_1 ,\ldots, \beta_m \} \cup \bigcup_{j=1}^m \k^i_0(A_j)$.\\
Let $\k^i(\phibar(A)):= \k^i_0(A)$,\\
$\calO^i_1:= (\phibar(\calS_i),\k^i,\k^i)$,
$\calO^i:=\calO_0 \preccirc \calO_1$ ($\calO_0$ as before).
We obtain, what we can call the ``fixed point free version of
extended Sch{\"u}tte Klammer symbols''. Let
$f_i: \calS_i \ar \Ord$,
$f_0(\alpha):= \alpha$,
$f_{i+1} 
{\beta_1 \cdots \beta_m \choose A_1 \cdots A_m}:=
%\left(\begin{array}{ccc}\beta_1& \cdots&\beta_m\\
%A_1  &\cdots &A_m\end{array} \right):=  
\Omega^{f_i(A_1)}\beta_1 +\cdots+ \Omega^{f_i(A_m)} \beta_m$,
then one can easily see
$\eval_{\calO^i}(\phibar(A))= 
\vartheta(f_i(A))$, $\vartheta$ as in \refact h.
Applying a similar restriction as in \refact h yields
the $\psi$-function restricted to 
ordinals $<\underbrace{\Omega^{\Omega^{\cdots^{\Omega}}}}_{i
\text{ times}}$.\par 
\ssubitem j As in \refact g and \refact h, we can define a 
version of the extended Sch{\"u}tte Klammer symbols with fixed points.
The verifications takes more space than is available here.
In some sense we believe however that in the context of 
OS, the fixed point free versions are at least as natural or 
even more natural than the versions with fixed points.
\ssubitem k Let $\calO^i$ be as in \refact i. 
The union of $\calO^i$ can be described as follows:
Let $\calS'_{-1}:= \{ 0 \}$, $\calS'_0:= \Ord$,
$\calS'_{n+1} := 
\calS'_n \preccirc 
(\schuette{\Ord \setminus \{ 0 \} \choose \calS'_n} \setminus 
|\schuette{\Ord \setminus \{ 0 \} \choose \calS'_{n-1}}|)$,
$\k^i_0: \calS'_i \ar_\omega \Ord$,
$\k^0_0(\alpha):= \{ \alpha \} $,
$\k^{i+1}_0(A)= \k^i_0(A)$ for $A \in \calS'_i$,
$\k^{i+1}_0(A)$ is defined as in 
\refact i for $A \in \calS'_{i+1} \setminus 
\calS'_i$.
Let $|\calS'|:= \bigcup_{i< \omega }|\calS'_i|$,
$\prec_{\calS'} := \bigcup_{i<\omega}\prec_{\calS'_i}$,
$\k^i(\phibar(A)):= \k^i_0(A)$,
$\k:= \bigcup \k^i$,
$\calO:= \calO_0 \preccirc (\phibar(\calS'),\k,\k)$, which is an OFG.\\
Define $\iota_i:|\phibar(\calS_i)| \ar |\phibar(\calS'_i)|$ by
$\iota_0(\phibar(\alpha)):= \alpha$,
$\iota_1(\phibar{\choose}):= \phibar(0)$,
$\iota_1(\phibar{\alpha \choose 0}):= \phibar(\alpha)$,
$\iota_1(a):= a$ otherwise,
$\iota_{i+2}(a):= \iota_{i+1}(a)$ for $a \in \phibar(\calS_{i+1})$,
$\iota_{i+2}(\phibar{\beta_1 \cdots \beta_m \choose A_1 \cdots A_m}):=
%\left(\begin{array}{ccc}\beta_1& \cdots&\beta_m\\
%A_1  &\cdots &A_m\end{array} \right):= 
\phibar{\beta_1 \cdots \beta_m \choose \iota_{i+1} (A_1)\cdots\iota_{i+1} (A_m)}$
otherwise.
%\left(\begin{array}{ccc}\beta_1& \cdots&\beta_m\\
%\iota_i(A_1)  &\cdots &\iota_i(A_m)\end{array} \right)$,
%if $m>1$ or $m=1 \land A_m \not = 0,{\hbox{\quad}\choose \hbox{\quad}}$,
%$\iota_i{\hbox{\quad}\choose \hbox{\quad}}:= 0$,
%$\iota_1 {\alpha \choose 0}:= \alpha$,
%$\iota_i+2{\alpha \choose {\hbox{\quad}\choose \hbox{\quad}}}
%:= \alpha$. 
Then $\iota_i$ is an isomorphism from $\phibar(\calS_i)$ to 
$\phibar(\calS_i')$, 
$\phibar(\calS_0') \segment \phibar(\calS_1') \segment \cdots$, the
isomorphism can be extended to an 
isomorphism from $\Arg_{\calO^i}$ to an initial segment of
$\Arg_{\calO}$ which preserves the component sets $\k_i$.
In this sense $\calO$ is the union of the $\calO_i$.
\end{enumerate}}
\end{example}
%
%%%%%%%%%%%%%%%%%%%%%%%%%%%%%%%%%%%%%%%%%%%
%%
%                                         %
%%%%%%%%%%%%%%%%%%%%%%%%%%%%%%%%%%%%%%%%%%%
The following lemma, which will be used in the proof
of \ref{theoexampleos}, helps to derive from an OFG an OS based on 
it, which is primitive recursive: One introduces a set of terms $\T''$,
which have the syntactical form of being arguments, if we replace
its components by ordinals. From these terms we select now a set
$\T$ of terms and $\T'$ of arguments based on $\T$ such that
$\T$ represents all elements of $\Cl$ and $\T'$ all elements
of $\Arg[\Cl]$. Whether an element belongs to $\T'$ or
$\T$ depends now on the ordering of its components, which
need to be in $\T$ and then one can show that under the assumptions formulated
in the next lemma, $\T$, $\T'$, $\prec$ and $\prec'$  will be 
primitive recursive. Especially condition \refsub{defprimreksys}e
will have some flavor of $\Pi_1^2$-logic, however, the
ordering of the elements of $\T$ and whether and element of $\T ''$ belongs
to $\T$ will depend not only on the order of its components, which
allows some more generality.
%
%
%
%%%%%%%%%%%%%%%%%%%%%%%%%%%%%%%%%%%%%%%%%%%
%%
%                                         %
%%%%%%%%%%%%%%%%%%%%%%%%%%%%%%%%%%%%%%%%%%%
%
%
%
\begin{lemma}
\laprint{defprimreksys}
Let $\calO$ be an OFG as usual, and
let $\T''$ be a primitive recursive set,
$\T \subseteq \T' \subseteq \T ''$,
$f: \T' \ar \Arg$,
$\krmhat: \T'' \ar_\omega \T''$, $\lrmhat: \T'' \ar_\omega \T''$,
$\length': \T'' \ar \nat$,
all primitive recursive.
Let for $a,b \in \T'$,
$a \prec' b: \Iff f(a) <' f(b)$, 
$a \prec  b: \Iff \eval(f(a)) < \eval(f(b))$.
Assume the following conditions:
\begin{enumerate} 
\ssubitem a $\all a \in \T '. \lrmhat(a) \subseteq 
\krmhat(a) \subseteq \T$.
\ssubitem b $\all a \in \T'(f[\krmhat(a)] = \k^0(f(a)) \land 
f[\lrmhat(a)] =  \lrm(f(a)))$.\\
$\all a \in \T''.\length'[\krmhat(a)]< \length'(a)$.
\ssubitem c For every subset $A$ of $\T$ 
$f[\{ t \in \T ' \mid \krmhat(t) \subseteq A \}] =
\{ a \in \Arg  \mid \k(a) \subseteq \eval[f[A]] \} $.
\ssubitem d If $A \subseteq \T'$, $f \restriction A$ is injective,
then $f \restriction \{ t \in \T ' \mid \krmhat(t) \subseteq A \} $
is injective.
\ssubitem e For $a,b \in \T''$ such that $\krmhat(a) \cup \krmhat(b) \subseteq 
\T$ we can determine
from $\prec$ restricted to $\krmhat(a) \cup \krmhat(b)$ 
(coded as a finite list which is coded as a natural number)
in a primitive recursive way, whether $a,b \in \T '$ and, if this is the
case, whether $a \prec ' b$.
\ssubitem f $a \in \T  \Iff a \in \T ' \land f(a) \in \NF \land \krmhat(a) \subseteq \T$.
\end{enumerate} 
Then $\T$, $\T'$, $\prec'$, $\prec$, are primitive recursive,
we can define $\lengthhat: \T' \ar_\omega \T$  such that
$\lengthhat(a) = \length(f(a))$ for $a \in \T'$,
and $(\T,\prec,\prec ',\krmhat,\lengthhat)$ is an
OS based on $\calO$.
\end{lemma}
%
%
%
{\bf Proof:} 
We determine for $t \in \T ''$ whether
$t \in \T '$, $t \in \T$ and for $r,s \in \T'$ such that $\length'(r)$,
$\length'(s)\leq \length'(t)$ whether  $r \prec' s$ and whether  $r \prec s$ holds
by recursion on $\length'(t)$ and side-recursion
on $\length'(r)+\length'(s)$:
If $\krmhat(t) \not \subseteq \T$, $t \not \in \T '$. Otherwise we can decide
whether $t \in \T '$. Assume now $r,s$ as above.
Whether $r \prec ' s$ 
follows from the induction hypothesis 
and then using Lemma 
\refcom{lemordgen}{eb} we can determine whether
$r \prec s$ holds. Now
$t \in \T :\Iff t \in \T' \land \krmhat(t) \prec t$.\\
$f: \T \ar \Arg$ is injective by condition \refact d. 
$f[\T] \subseteq \NF$, $f[\T] \subseteq \Cl$.
We show $f[\T] = \Cl$. It suffices to show
$\all a \in \Cl.\exists b \in \T. f(b) = a$.
Induction on $a$. $\krmhat(a)  \subseteq f[\T]$. Therefore
there exists a $t \in \T '$, $\krmhat(t) \subseteq \T$,
such that $f(t) = a$. Since $\NF(a)$, follows
$t \in \T$.\\
$\lengthhat(t)$ is defined now by
$\lengthhat(t):= \max(\lengthhat[\krmhat(t)]
\cup \{ -1 \})+1$.\QED 
%
%
%
%
%
%
%%%%%%%%%%%%%%%%%%%%%%%%%%%%%%%%%%%%%%%%%%%%
%%%
%%                                         %
%%%%%%%%%%%%%%%%%%%%%%%%%%%%%%%%%%%%%%%%%%%%
%%
%%
%%
%\begin{remark}
%\laprint{remofgdil}
%The conditions of Lemma \ref{defprimreksys} are almost like the 
%definition of ordinal denotations as studied by Girard in the area
%of $\Pi^1_2$-logic. However, we want to keep a little bit more flexibility
%in the sense, that whether an element belongs to the set $\T '$ might
%not only depend on the ordering of its components but as well
%on some syntactical  properties of it. For instance, when using later
%ordinary sum of  $n$ additive principal numbers, the notation
%$(a_1 ,\ldots, a_n)$ for  the sum of $a_1 ,\ldots, a_n$ will not
%only depend on the ordering of the $a_i$, but as well
%on, whether they are represent additive principal numbers (which
%one can see 
%\end{remark}
%%
%
%
%{\bf Important note:} The development of the ordinal systems corresponding
%to the examples above can be done purely syntactically,
%i.e. without using the
%OFG and therefore without referring to ordinals, and this is as easy
%as the above approach. The advantage of developing as OFG is that
%we get  a generalization of the functions for arbitrary ordinals and
%not only for notations.
%
%
%
\begin{lemma}
\laprint{theoexampleos}
For all the OFG in the Example \refcom{exmplordfun}a
- \refsub{exmplordfun}i,
except for those examples involving the lexicographic ordering on weakly 
descending sequences (\refsub{exmplordfun}b and the examples referring to it)
there exists an elementary OS represented by them.
\end{lemma}
%
%
%
%
%%%%%%%%%%%%%%%%%%%%%%%%%%%%%%%%%%%%%%%%%%%
%%
%                                         %
%%%%%%%%%%%%%%%%%%%%%%%%%%%%%%%%%%%%%%%%%%%
%
%
%
\begin{remark}
\laprint{remnoteordinals} Note that the development of the OS 
can be done without having to refer to ordinals.
\end{remark}
%
%
%
%
%
{\bf Proof of Lemma \ref{theoexampleos}:} First, using Lemma \ref{defprimreksys} we can
define a primitive recursively represented  OS based on the 
OFG. We look only at the example \refsub{exmplordfun}g in detail:
Let $\underline{0}:= \Sigmabar'()$, $\underline{l+1}:= \Sigmabar'(\underline{0},\underline{l}\plusbar 1)$,
$\underline{\nat}:= \{ \underline{l} \mid l \in \omega \}$.
Define $\T''$ together with $\krmhat:\T'' \ar_\omega \T''$,
$\lrmhat:\T'' \ar_\omega \T''$, $\length(\T'') \ar \nat$
recursively by:\\
If $m \in \omega$, $a_i \in \T ''$, $b_i \in \underline{\nat}$,
$b_i \not = \underline{0}$ for $i<m$, 
then\\ 
$t_0:= \Sigmabar'(a_1,b_1 ,\ldots, a_{m-1}, b_{m-1},a_m,b_m \plusbar 1) \in 
\T''$,\\ 
$\krmhat(t_0):= \{ a_1,b_1 ,\ldots, a_m, b_m \}$,
$\lrmhat(t_0):= 
\mycases{\{ a_1,b_m\} &
\text{if $m>1$ or $b_m \not = \underbar{0}$,}\\ 
\emptyset&\text{otherwise,}}$\\
$\length(t_0):= \max \{ \length(a_1) ,\ldots, \length(a_m),
\length(b_1) ,\ldots, \length(b_m) \} +1$.\\
If $a_1 ,\ldots, a_m,b_1 ,\ldots, b_m \in \T''$,
$a_1 ,\ldots, a_m \not = \underline{0}$,
$t_1:= \phibar{a_1 \cdots a_m \choose b_1 \cdots b_m}\in \T''$,
and we define $\krmhat(t_1)$, $\lrmhat(t_1)$ by translating
the conditions from Example \refcom{exmplordfun}g and
$\length(t_1)$ as before.
$\T''$, $\length$, $\krmhat$, $\lrmhat$ are primitive recursive.\\
Define $\T \subseteq \T''$, $\T' \subseteq \T ''$,
together with $f: \T' \ar \Arg$ inductively by:
If $t_0$ is as above,
$\krmhat(t_0) \subseteq \T$, $\eval(f(a_1)) > \cdots > \eval(f(a_m))$,
then $t_0 \in \T'$,\\
$f(t_0):= \Sigmabar'(\eval(f(a_1)),\eval(f(b_1)) ,\ldots, 
\eval(f(a_{m-1})), \eval(f(b_{m-1})),$\\
\hbox{\qquad \qquad}$\eval(f(a_m)),\eval(f(b_m))+ 1)$.\\
If $t_1$ is as above,
$\eval(f(b_1))> \cdots >  \eval(f(b_m)) $, then
$t_1 \in \T'$,\\ 
$f(t_1):= \phibar{\eval(f(a_1)) \cdots \eval(f(a_m)) \choose 
\eval(f(b_1)) \cdots  \eval(f(b_m))}$.\\
Further $t \in \T :\Iff t \in \T ' \land \NF(f(t) )$.\\
The conditions of Lemma \ref{defprimreksys} are now fulfilled
and therefore most conditions
of a primitive recursively represented 
OS are fulfilled, what is missing is verified
easily.\par 
The other OS are treated similarly. The constructions of new orderings
considered in \cite{setzvenedig} Lemma 3.5 
(especially the lexicographic ordering on pairs and
and strictly descending sequences) yield orderings such that
transfinite induction over it reduces to transfinite induction
over the underlying orderings in $\PRA$ and  therefore
we get actually get in all cases elementary OS.\QED \par 
However, transfinite
induction over weakly descending sequences reduces only
in $\HA$ to the underlying ordering, so the examples 
where we used weakly descending sequences do not yield elementary
OS.\par 
%
%
%
%%%%%%%%%%%%%%%%%%%%%%%%%%%%%%%%%%%%%%%%%%%
%%
%                                         %
%%%%%%%%%%%%%%%%%%%%%%%%%%%%%%%%%%%%%%%%%%%
%
%
%
\begin{theorem}
\laprint{theostrengthordsyslow}
\begin{enumerate} 
\ssubitem a 
The bound in Theorem \ref{theostrengthordsys} is sharp:
The  supremum of the order types of elementary OS is exactly the
Bachmann-Howard ordinal.
\ssubitem b The limit of the order type of ordinal systems from below
as introduced in \cite{setzvenedig}
is the Bachmann-Howard ordinal.
\end{enumerate} 
\end{theorem}
%
%
%
{\bf Proof:} \refact a:
Example \refcom{exmplordfun}i yields OS of strength
$\vartheta(\underbrace{\Omega^{\Omega^{\cdots^\Omega}}}_{i \text{ times}})$,
which in the limit reaches the Bachmann-Howard ordinal.\\
\refact b: by subsection \ref{equivonsfb}, \refact a and
the upper bound developed in the last two sections of 
\cite{setzvenedig}.\QED \par 
Note that the construction in 
the example used in \refact a really
exhausts the full strength of elementary OS: In the  OS related to $\calO_i$
$\prec '$ is built from $\prec$ 
using $i$-times nested lexicographic ordering on descending
sequences, $\TI$ over which reduces to the underlying ordering by
using formulas of increasing length. 
The union of the $\calO_i$, $\calO '$
yields an OFG, such that in the OS belonging to it
$\TI$ over $\prec '$ does no longer reduce to $\TI$ over $\prec$ in
$\PRA$, so it is no longer elementary. What is required are 
twice iterated ordinal systems.
%
%
%
%{\bf Introducing ordinals via closure sets.} The usual ordinal notation
%systems below the Bachmann-Howard ordinal, which lead to the definition
%of ordinal notation systems from below, are always based on
%ordinal functions, which can be defined using closure sets.
%Consider as an example 
%$\lambda \alpha.\omega^\alpha$. Then we can define $\omega^\alpha$
%simultaneously with sets $\C_\omega(\alpha)$ for $\alpha \in \Ord$
%(where $\Ord$ is the class of ordinals) by induction on $\alpha$:\\
%$\C_\omega (\alpha) =$ closure of $\{ 0 \} \cup \{ \omega^\gamma \mid \gamma<\alpha \} $ under $+$;
%$\omega^\alpha = \min \{ \gamma \in \Ord \mid \gamma \not \in \C_\omega (\alpha)$.\par 
%We can modify this definition slightly and define\\
%$\C_\omega (\alpha) = $ closure of $\beta \cup \{ 0 \} $ under 
%$\lambda \gamma< \alpha.\omega^\gamma$ and $+$, 
%$\omega^\alpha$ is definded as before.\\
%In the same spirit we can define addition by \\
%$\C_+(\alpha,\beta):= $ closure of $\alpha \cup \beta$ under
%$\lambda \rho<\beta.\alpha + \rho \} $,
%$\alpha + \beta = \min \{ \gamma \in \Ord \mid \gamma \not \in \C_+(\alpha,\beta) \}$,\\
%and\\ 
%$\phi_\alpha \beta := \min \{ \gamma \in \Ord \mid
%\gamma \not \in \C_\phi(\alpha,\beta)$ with
%$\C_\phi(\alpha,\beta):= $ closure of
%$\alpha \cup \beta \cup \{ 0 \} $ under $+$,
%$\lambda \gamma<\alpha .\lambda \rho.\phi_\gamma \rho$,
%$\lambda \gamma<\beta.\phi_\alpha \gamma$.\\
%We can define Sch{\"u}tte's Klammer symbols as follows: 
%for a Klammer symbol $A$ let $\Ord(A)$ be the set of ordinals contained
%in it, and $<_\lex$ the usual twice nested lexicographic ordering.
%Then define:\\
%$\C(A):= $ closure of $\bigcup \Ord(A) \cup \{ 0 \} $ under
%$\lambda \Ord(B).\phi(B)$ restricted to $B <_\lex A$\\
%(more formally the least set $M$ containing $\bigcup \Ord(A)$,
%$0$ such that $\Ord(B) \subseteq M$, $B <_\lex A$, then $\phi(B) \in M$).\\
%$\varphi(A):= \min \{ \gamma \mid \gamma \not \in \C(A) \}$.\par
%All these definitions can be generalized in the following sense:
%Assume some set of terms $\T$ built from ordinals. So every
%term $t$ has a set $\k(t)$ of ordinals, it is built from. Define an
%partial ordering $<'$ on these terms. Assume $<'$ is well-founded.
%Define by induction on $<'$ sets 
%$\C(t) \subseteq \Ord$ and $\eval: \C(t) \ar \Ord$ by
%$\C(t):=$ least set containing $\bigcup \k(t)$, such that
%if $s <'t$, $\k(s) \subseteq  \C(t)$, then $\eval(s) \in \C(t)$,
%$\eval(t):= \min \{ \gamma \in \Ord \mid \gamma \not \in \C(t)$.\\
%So for the definition of $+$, we define
%as set of terms $\T_+:= \{\underbar{0} \} \cup 
%\{ \alpha \underbar{+} \beta \mid  
%\alpha,\beta \in \Ord \} $,
%and as ordering on $\T_+$,
%$<_+$ defined by  
%$0 <_+ \alpha \underbar{+} \beta $, and 
%$\alpha \underbar{+} \beta <_+ \alpha ' \underbar{+} \beta '$
%iff $\alpha = \alpha ' \land \beta < \beta '$,
%$\eval(\alpha \underbar{+} \beta) = \alpha + \beta $.\\
%In case of $\lambda \rho .\omega^\rho$,
%let $\T_\omega:= \T_+ \cup \{ \underlinedef{\omega}^\alpha \mid 
%\alpha \in \Ord \} $,\\
%and the ordering $<_\omega$ on $\T_\omega$ contains
%$<_+$, has $t <_\omega \underlinedef{\omega}^\gamma$ for $t \in \T_+$ and 
%$\underlinedef{\omega}^\alpha <_\omega \underlinedef{\omega}^\beta$ iff
%$\alpha < \beta$.
%Similarly one can define orderings for all other functions mentioned before.\\
%If we select now normal forms, which means for each ordinal $\alpha$ we have
%at most one term $t$ such that $\eval(t) = \alpha$, which
%implies typically, that we do not only have
%$\k(t) \leq \eval(t)$ as we had before, but
%$\k(t) < \eval(t)$ (because if $\alpha \in \k(t)$, $\alpha =  \eval(t)$,
%then $\alpha$ would be ``a shorter'' notation than $t$,
%then the ordering $<'$ can be defined as a linear ordering,
%since $\C(t)$ will be linearly ordered (define 
%for $t,t'$ uncomparable $t <' t'$ iff $\eval(t) < \eval(t')$.\\
%If we have a set of terms $\T$ in normal forms as before, then
%we can define a set of ordinal notations $\OT$ together
%an interpretation $\ordinal:\OT \ar \Ord$ by:
%if $t \in \T$, $\k(t) \subseteq \ordinal[\OT]$,
%then we add a notation $t'$ for $t$ to $\OT$ and
%define $\ordinal(t'):= \eval(t)$. If we define now for
%$r,s \in \OT$, 
%$r \prec s \Iff \ordinal(r) < \ordinal(s)$,
%$r \prec ' s \Iff r' <' s'$,  $\k(r):= \ordinal^{-1}[\k(r')]$,
%where in the last two definitions $\eval(r') = \ordinal(r)$, 
%$\eval(s') = \ordinal(s)$, then
%we get an OS-structure, which fulfills (OS 1) (OS 2) and (OS 3).\\
%Now (OS 4) expresses that the ordering $<'$ is not just an
%arbitrary one, but a good one
%and this means that well-foundedness of $\k^{-1}(A)$ for
%$A \subseteq \Ord$ reduces to well-foundedness of $A$. However, this
%last definition is not a good one, since we have already that
%all of $A$ is well-ordered. Girard gave some definition in his 
%$\Pi_1^2$-logic.????? We have to make an exact comparision yet --- 
%Definition of Ordinal system based on ordinals only possible ????? 
%In our approach we have made this precise only for the set of 
%notations, and (OS 4) expresses what we mean
%by a good ordering $\prec '$. \par 
%We will give the following definition and lemma, in order
%to sharpen our intuition. It will not be needed later.
%%
%%
%%%%%%%%%%%%%%%%%%%%%%%%%%%%%%%%%%%%%%%%%%%%
%%%
%%                                         %
%%%%%%%%%%%%%%%%%%%%%%%%%%%%%%%%%%%%%%%%%%%%
%%
%%
%%
%\begin{deflemma}
%\laprint{lemclosureos}
%Let $\calF$ be a OS.\\
%For $s \in \T$, let
%$\C(s)$ be the least set, such that
%for all $t \preceq \k(s)$, $t \in \C(s)$ and 
%for all $t \in \T$, if $\k(t) \subseteq \C(s)$ and
%$t \prec ' s$, then $t \in \C(s)$.\\
%Then $\C(s)= \{ t \in \T \mid t \prec s \} $.
%\end{deflemma}
%%
%%
%%
%{\bf Proof:} ``$\subseteq$'' Induction on the definition
%of $\C(s)$. Assume $t \in \C(s)$ according to induction.
%If $t \preceq \k(s)$,
%$t \preceq \k(s) \prec s$.
%Otherwise $t \prec 's$, by IH $\k(t) \prec s$. If 
%$s \preceq t$, then $s \preceq' t$ or 
%$s \preceq \k(t)$ contradicting the linearity of $\prec '$
%and   $\prec$. ``$\supseteq$'': We show
%$\all  t \in \Cl_\calF.t \prec s \ar t \in \C(s)$ by induction
%on $t \in \Cl_\calF$. Assume $t$ according to induction.
%If $t \preceq \k(s)$, $t \in \Cl(s)$. Otherwise
%$t \prec ' s$, by IH and $\k(t) \prec t \prec s$ follows 
%$\k(t) \subseteq \Cl(s)$, $t \in \Cl(s)$.\par 
%%
%%
%%
%%%%%%%%%%%%%%%%%%%%%%%%%%%%%%%%%%%%%%%%%%%%
%%%
%%                                         %
%%%%%%%%%%%%%%%%%%%%%%%%%%%%%%%%%%%%%%%%%%%%
%%
%%
%%
%\begin{remark}
%\laprint{remcsprimrek}
%$\C(s)$ as defined in \ref{lemclosureos} can be defined primitive-recursively.
%\end{remark}
%%
%%
%%
%{\bf Proof:} Let a sequence $<t_1 ,\ldots, t_m>$
%a proof of $t_m \in \C(s)$, iff we can infer $t_i \in \C(s)$
%from knowing only that $t_i \in \C(s)$. One verifies,
%that the set of such proofs is primitive recursive and
%one can show that the set of terms, for which
%such proofs exist, is $\C(s)$ (i.e. fulfills the
%same closure and induction principles as $\C(s)$).
%
%
%
%
%
%
%
%\refact c Let $(\OT,\prec)$ be the usual ordinal notation
%system built from $0$, $\Omega$,\break 
%$\lambda \alpha,\beta.\omega^\alpha+\beta$ and
%$\psi$ (see for instance \cite{buchholzschuettebook}, but 
%omitting $\lambda \alpha.\Omega_\alpha$ except
%of the constant $\Omega_1$, written as $\Omega$, and
%of $\psi \alpha$ for $\alpha \not = 0$, and writing
%$\psi$ instead of $\psi_0$, or, alternatively, replacing 
%the basic functions in 
%\cite{buchholzanewsys} or \cite{pohlersbook} by
%our $0$ and $\lambda \alpha,\beta.\omega^\alpha+\beta$).
%Let $\omega_0(x):= x$, $\omega_{n+1}(x):=
%\omega^{\omega_m(x)}$. 
%Let $\calOT_m:= (\OT_m ,\prec_m):= 
%(\OT,\prec) \cap \{ c \in \OT_m \mid c \prec \psi(\omega_{n+1}(\Omega+1)) \} $.

%\begin{definition}{\rm 
%\laprint{defOmegapower}
%Assume $A$, $B$ are orderings,
%$0$ is the minimal element of $B$.
%\begin{enumerate}
%\ssubitem a $\Omega^A \cdot B$
%is the ordering\\
%$(\{ \Omega^{a_1} \cdot b_1 +\cdots+ \Omega^{a_m} \cdot b_m 
%\mid n \in \omega, a_i \in |A|,
%b_i \in |B|\setminus \{ 0 \} , a_m \prec_A \cdots \prec_A a_1 \},
%\prec')$,\\
%where elements
%$\Omega^{a_1} \cdot b_1 +\cdots+ \Omega^{a_m} \cdot b_m $ are
%ordered as
%$\left (\begin{array}{ccc}a_1&\cdots&a_m \\
%b_1&\cdots &c_m \end{array}\right)$ in
%$\left({B \choose A}\right)^\ast_\des$.
%(Here, if $\Omega^a \cdot b$ is a defined function, this
%definition has to be evaluated).
%Note that, if $A$, $B$ are primitive recursive, transfinite induction over
%$\Omega^A \cdot B$ reduces in $\PRA$
%to transfinite induction over $A$ and $B$.
%$\Omega_c^A \cdot B$,
%$\omega^A \cdot B$ are defined similarly.
%\ssubitem b $\Omega \uparrow_0(A):= A$,
%$\Omega\uparrow_{n+1}(A):= \Omega^{\Omega\uparrow_m(A)} \cdot A$.
%Note that, if $A$, $B$ are primitive recursive, transfinite induction over
%$\Omega^A \cdot B$ reduces in $\PRA$
%to transfinite induction over $A$ and $B$.
%$\Omega_c \uparrow_m(A)$ 
%$\omega \uparrow_m(A)$ is defined similarly.
%The set of components $\krmhat_m(a)$ for $a \in |\Omega_c \uparrow_m(A)|$
%is defined by: $\krmhat_m(a):= \{ a \} $,
%and $\krmhat_{n+1}( \Omega^{a_1} \cdot b_1 +\cdots+ 
%\Omega^{a_m}\cdot b_m):= \{ b_1 ,\ldots, b_m \} \cup 
%\krmhat_m[\{ a_1 ,\ldots, a_m \} ]$. 
%\end{enumerate}}
%\end{definition}
%%
%%
%%
%Let now $\prec_m$ be the ordering on $\OT_m$.
%Let $\prec_m'$ be the ordering on
%$\{ 0 \} \preccirc 
%(\omega^{\calOT_m} + \calOT_m)\preccirc \psi(\M_m) \cap \OT_m$, 
%where
%$\M_m:= \Omega \uparrow_m(\calOT_m))$.
%Define $\k:\OT_m \ar_\omega \OT_m$,
%$\k(0):= \emptyset$,
%$\k(\omega^a + b):= \{ a,b \} $ (if in normal form),
%$\k(\psi(c)):= \krmhat_m(c)$ (if in normal form), where
%$\krmhat_m(c)$ is the set of components of
%$\M_m$.\\
%Then $(\OT_m,\prec_m,\prec_m ',\k$ is an elementary OS:\\
%$\prec_m$, $\prec_m '$ are
%linear orderings provable in $\PRA$ is clear.
%$(\OS\ 1)$: For $0$ and $\omega^a +b $ this is clear
%and for $\psi(c)$ this follows by definition (we add
%$\psi(c)$ to $\OT$ only, if $c \in C(c)$ which means just
%$\k^n(c) \prec_m c$). $(\OS\ 2)$, $(\OS\ 3)$: easy. $(\OS\ 4)$: 
%$\prec_m '$ was defined using constructions, for which always
%holds, that $\TI$ over the resulting construction reduces in 
%$\PRA$ to its underlying orderings (the proofs can be found
%in \cite{setzvenedig}).\\
%Since the ordertype of $(\OT_m,\prec_m)$ is
%$\psi(\omega_m(\Omega+1))$ and 
%$\sup_{n \in \omega}\psi(\omega_m(\Omega+1))$
%is the BHO we are done.\QED
%\medskip 

%It seems to be more intuitive, to write the OS introduced in the
%proof of \refcom{theoremstrengthelemos}c slightly differently. First of all we note, that
%we could have replaced $\psi$ by the function $\vartheta$
%as introduced in \cite{rathjenweiermann} and the proof
%works with almost the same  argument. Further we can write
%the cantornormalform to the basis $\Omega$ instead
%as a matrix, i.e. we replace
%$\Omega^{a_1} \cdot  b_1 +\cdots+ \Omega^{a_m}\cdot b_m$ by
%$\left (\begin{array}{ccc}a_1&\cdots&a_m\\b_1&\cdots&b_m \end{array}\right)$.
%So the notation $\Omega^{\Omega^a \cdot b + d}\cdot e + \Omega^f \cdot g$
%which is   $\Omega^{\Omega^a \cdot b + \Omega^0 \cdot d}\cdot e
%+ \Omega^{\Omega^0 \cdot f } \cdot g$
%is replaced by \\
%$\left ( \begin{array}{cc} e &g\\
%\left (\begin{array}{cc}b & d \\a&0 \end{array} \right )&
%\left (\begin{array}{c}f \\0 \end{array} \right )\end{array} \right)\enspace .$\\
%These matrices are ordered by the same nested lexicographic ordering 
%as it was done for the terms in $\Omega$-Cantor-normalform.
%Now $\vartheta(A)$ can be regarded as an extended Klammer symbol.
%More precisely, define,  if $\Ord$ is the set of ordinals
%with ordering $<$,
%$\Mbb_m$ for $m \leq n$ together
%with $<_m$ on $\Mbb_m(X,<)$ by\\
%$(\Mbb_0,<_0):= \Ord$, 
%$(\Mbb_{k+1},<_{k+1}):= 
%{(\Ord \setminus \{ 0 \},<) \choose (\Mbb_m,<_m)}^\ast_\des $\\
%Define $\varphi_m:\Mbb_m \ar \Ord$ by\\
%$\varphi_0(\alpha):= \epsilon_\alpha$,
%and, if $D = \left(\begin{array}{ccc} \alpha_0 &\ldots&1+ \alpha_m \\
%                   A_0 &\ldots&A_m
%\end{array}\right)\in \Mbb_{k+1}$, then\\
%$\varphi_{k+1}(D)
%:= \alpha_m$th fixed point of all functions
%$\lambda \gamma.\omega^\gamma$  and\\ 
%$\lambda \gamma.\varphi_{k+1} 
%\left(\begin{array}{cccc} \beta_0 &\ldots&\beta_m&\gamma  \\
%                   B_0 &\ldots&B_m&B_{m+1} 
%\end{array}\right) $
%with\\ 
%$\left (\begin{array}{ccc} \beta_0 &\ldots&\beta_m  \\
%                   B_0 &\ldots&B_m
%\end{array}\right) <_m
%\left(\begin{array}{ccc} \alpha_0 &\ldots&\alpha_m \\
%                   A_0 &\ldots&A_m
%\end{array}\right)$\\ 
%and where $\beta_i$ and the ordinals $A_i$ are below some of the
%ordinals in $D$ or can be majorized by the same process,\\
%which is more precisely
%the least ordinal not in 
%$\C(D)$, where\\
%$\C(D)$ is the 
%closure of $0$ under 
%$\lambda \gamma.\omega^\gamma$ and
%$\gammavec \mapsto \varphi \left (\begin{array}{ccc} \beta_0 &\ldots&\beta_m  \\
%                   B_0 &\ldots&B_m
%\end{array}\right)$,
%where $\gammavec$  are the ordinals in $B_i$ and $\beta_i$
%and 
%$\varphi \left (\begin{array}{ccc} \beta_0 &\ldots&\beta_m  \\
%                   B_0 &\ldots&B_m
%\end{array}\right) <_m A$.\\
%In other words, $\varphi(D)$ is the least ordinal
%which cannot be reached using the basic functions
%and $\varphi(D')$ for $D' <_m D$, i.e. by functions 
%defined before $\varphi(D)$.\\
%The last definition is in the case $k=2$ equivalent
%to the definition of the Sch{\"u}tte-Klammer symbols
%based on $\lambda \gamma.\omega^\gamma$,
%and $\varphi_m$ are the straight-forward generalization of
%the Klammer symbols.\\
%If we define further
%$f_k : \Mbb_k \ar \Ord$ where $\Ord$ are the full ordinals, by
%$f_0(\alpha):= \alpha$, and, if $D$ is as above,
%$f_{k+1}(D):= 
%\Omega^{f_k(A_1)}\alpha_1 +\cdots+ 
%\Omega^{f_k(A_m)}\alpha_m$.
%Then $\varphi_k(D) = \vartheta(f_k(D))$.
%Therefore the ordinals above $\Omega$ can be seen as
%a more elegant way of writing matrices
%and the system based on the $\vartheta$-function
%is another description of the 
%extended Sch{\"u}tte-Klammer symbols. The notation systems constructed
%reach every ordinal below the BHO (therefore
%we can reach BHO ``from below'').


%\medskip
%{\bf Relationship between $\psi$- and $\vartheta$-function} The
%above considerations illustrate the relationship between the $\psi$- and
%$\vartheta$-function. For the $\psi$-function condition (OS 1) 
%is guaranteed by having $\psi(a)$ is a term in normal form
%only if for the ordinal denoted by $a$ we have
%$\ordinal(a) \in \C(\ordinal(a))$ which means exactly $\k(a) \prec \psi(a)$.
%In thhe case of the $\vartheta$-function, one
%forces $\vartheta(a)$ to be bigger than $k(a)$.\par  
%
%{\bf Equivalences of different ordinal systems.} There are several
%variations of the $\psi$- and $\vartheta$-function, which depend
%essentially on the choice of the basic functions: One
%can start with $0$ and $\lambda \alpha,\beta.\omega^\alpha + \beta$,
%or only with $0$ and addition
%(and additionally $\lambda \alpha.\Omega^\alpha$, 
%$0$, $1$ and addition,
%$0$, addition and the Veblen function, 
%$0$, addition and $n$-ary Veblen function or even
%add the Sch{\"u}tte Klammer symbols, based on the 
%basic functions added before. 
%Now we can see now immediately, that for all these systems
%we have $\psi(\epsilon_{\Omega+1})$ (or
%$\vartheta(\epsilon_{\Omega+1})$ is the BHO.
%First of all, all these systems are above the weakest
%where we add only $0$ and addition, which can be seen
%in \cite{buchholzanewsys} (Buchholz does not need
%$\Omega^\alpha$ since one can simulate
%this function by the function $\psi_2$)
%to reach the BHO. For $\vartheta$ we have
%always that, if $a \in \OT_\psi$, then
%(if we replace the symbol $\psi$ by $\vartheta$ 
%$a \in \OT_\vartheta$ (where $\OT_\psi$ and
%$\OT_\vartheta$ are the notation systems based
%on $\psi$, $\vartheta$ respectively) and
%we have for $a,b \in \OT_\psi$, $a \prec_\psi b \Iff
%a \prec_\vartheta b$, and therefore the ordertype
%of $\OT_\varphi$ is less than the ordertype of $\OT_\psi$
%(In \cite{rathjenweiermann} one can find
%a more precise calculation).
%For the other versions we see that we
%can interpret the system built from $0$, $+$ and
%$\lambda \alpha.\Omega^\alpha$ ($\psi$ or $\vartheta$-version)
%into it in the same way as the interpretation of the
%$\psi$-system into the $\vartheta$-system and get
%again a lower bound.\par
%For the upper bound one sees just, that all variations can
%be generated by an elementary OS: We take always as
%$\Arg$ $\T$ and order the $\psi$ (or $\vartheta$-terms) as
%before above all other notations and order them with
%respect to the arguments of it and the nested lexicographic
%ordering. Now only the terms generated by
%the basic functions need to be ordered,
%$0$ will be always the least element. In
%case of the function addition (in normal form)
%(with notations being $(a_1 ,\ldots, a_m)$
%for $a_1 +\cdots+ a_m$), we have to work
%a little bit, since we have here the lexicographic
%ordering on weakly descending sequences, $\TI$ on which
%does not reduce to $\TI$ on the underlying ordering. However,
%we can replace this by 
%by tupels $(a_1,m_1 ,\ldots, a_m,m_m)$, where
%$m_i$ are of the form $\Succ^k(0)$ and the
%sequence $a_1 ,\ldots, a_m$ are strictly
%descending, as a notation for
%$a_1 \cdot m_1 +\cdots+ a_m \cdot m_m$ and have
%to add the successor function. 
%Let $\nat ':= \{ S^k(0) \mid k \in \omega  \setminus \{ 0 \} \} $
%ordered as is $\omega$.
%Order now the  initial
%functions by
%$S(\OT_m) \preccirc (\OT_m \cdot \nat'_-)_{\des}^\ast$ 
%we get an elementary OS from below and the underlying
%notation system is (in $\PRA$ provable) isomorphic
%to the original one.
%If we have apart from addition the $\varphi$-function,
%the $\phi$-terms will be ordered above the addition terms
%and ordered lexicographically, if we have
%additionally the Klammer symbols they will be ordered
%above all other functions and ordered by lexicographic ordering
%on descending sequences based on lexicographic ordering of
%pairs. That we have an OS can now be in all cases verified easily.

\section{$n$-times Iterated Ordinal Systems}
\laprint{sectniterate}
\subsection{Introduction}
The usual way of getting beyond the Bachmann-Howard ordinal is to 
violate condition (OS 1), which was the basis for our analysis in
the last two sections of \cite{setzvenedig}. However, 
the foundationally more interesting 
approach seems to keep the condition that the ordinals are built
from below and instead weaken the requirement that 
$\TI$ over $\prec '$ 
reduces to $\TI$ over $\prec$ in an elementary way, namely in $\PRA$.\par 
If we look at the OS corresponding to 
Example \refcom{exmplordfun}k we can see that we generated
by meta recursion a sequence of matrices of increasing complexity.
We can replace this meta recursion by using a second OS:
Let $\calF_0:=(\T_0,\prec,\prec',\k,\length)$ be the 
primitive recursive (but non-elementary) OS based on 
$\calO$, defined by the method used in Lemma \ref{theoexampleos}.
Let $\T_1:= \{ A \mid \phibar(A) \in \T_0 \}$ with the ordering 
$A \prec_1 A' :\Iff  \phibar(A) \prec' \phibar(A')$.
Define $\k_{0,j}: \T_0 \ar_\omega \T_j$ for $i,j=0,1$ by:
$\k_{0,0}:= \k$, $\k_{0,1}(\Sigmabar'(\avec)):= \emptyset$,
$\k_{0,1}(\phibar(A)):= \{ A \}$ and let
for $D \subseteq \T_0$, $E \subseteq \T_i$, $D \restriction_{0,j}E:=
\k_{0,j}^{-1}(E)$. Then if $B \subseteq \T_0$,
$C \subseteq \T_1$ are well-ordered w.r.t. $\prec$, $\prec_1$, then
$(\T_0 \restriction_{0,0} B) \restriction_{0,1} C$
is well-ordered. If we take as $C$ the
set of matrices built from notations in $B$ only and, if we know
that $C$ is well-ordered, then $\T_0 \restriction_{0,0} B = 
(\T_0 \restriction_{0,0} B) \restriction_{0,1} C$ is well-ordered, and
we have shown that $\calF_0$ is an OS.\par
Now in order to show that $C$ is an well-ordered we use a second 
OS:
Define $\k_{1,i}: \T_1  \ar_\omega \T_i$ by: 
if $a \in \T_0$, $\k_{1,0}(a) := \{ a \}$,
$\k_{1,1}(a):= \emptyset$, and
if $A = 
{a_1 \cdots a_m \choose A_1 \cdots A_n}$, then
%\left (\begin{array}{ccc} \alpha_1 &\cdots&\alpha_m  \\
%A_1 &\ldots&A_m \end{array}\right)$, then
$\k_{1,0}(A):= \{ a_1 ,\ldots, a_n \} 
\cup \bigcup_{j=1}^m \k_{1,0}(A_j)$,
$\k_{1,1} (A):= \{ A_1 ,\ldots, A_m \}$ and define 
$C\restriction_{1,j}D$ as before.
Let $\calF_1:= (\T_1,\k_{1,1},\prec ',\prec ',\length)$.
and for  $B \subseteq \T_0$ 
$\calF_1[B]:= 
(\T_1 \restriction_{1,0} B,\prec_1,\prec_1,\k_{1,1},\length)$.
If $B$ is $\prec$-well-ordered, then $\calF_1[B]$ is now an ordinal system,
and therefore $C:= \T_1 \restriction_{1,0}B$ is well-ordered and
with this $C$ the above holds.\par 
$\calF_1$ is a system which
internalizes what was before meta-induction. It 
is a relatively weak OS which is the relativized extension of
the OS for the Cantor normal form. Expanding the OS we introduced in the
last section we can get now more complex ``matrices'', which can be
used by the first OS. As before, we will exhaust the strength of this concept
(which will be called $2$-OS) by using the hierarchy of
extended Klammer symbols. In order to go beyond this, we
can iterate the step from OS to $2$-OS once more and yield
$3$-OS, $4$-OS etc. We are going to formalize in the following 
$n$-OS for arbitrary natural numbers $n$.\par 
%
%
%
%
\subsection{$n$-Ordinal Systems}
In the example before we had as  basic structures two
ordinal structures 
$\calF_j:=(\T_j,\prec_j,\break 
\prec_j',\k_{j,j},\length_j)$ ($j=0,1$) together
with functions $\k_{i,1-i}:\T_i \ar_\omega \T_{1-i}$.
It can be visualized as follows:\\
\noindent%%%
%%% begin pic4.pictex
\font\thinlinefont=cmr5
%
\begingroup\makeatletter\ifx\SetFigFont\undefined
% extract first six characters in \fmtname
\def\x#1#2#3#4#5#6#7\relax{\def\x{#1#2#3#4#5#6}}%
\expandafter\x\fmtname xxxxxx\relax \def\y{splain}%
\ifx\x\y   % LaTeX or SliTeX?
\gdef\SetFigFont#1#2#3{%
  \ifnum #1<17\tiny\else \ifnum #1<20\small\else
  \ifnum #1<24\normalsize\else \ifnum #1<29\large\else
  \ifnum #1<34\Large\else \ifnum #1<41\LARGE\else
     \huge\fi\fi\fi\fi\fi\fi
  \csname #3\endcsname}%
\else
\gdef\SetFigFont#1#2#3{\begingroup
  \count@#1\relax \ifnum 25<\count@\count@25\fi
  \def\x{\endgroup\@setsize\SetFigFont{#2pt}}%
  \expandafter\x
    \csname \romannumeral\the\count@ pt\expandafter\endcsname
    \csname @\romannumeral\the\count@ pt\endcsname
  \csname #3\endcsname}%
\fi
\fi\endgroup
\mbox{\beginpicture
\setcoordinatesystem units <0.900000cm,0.90000cm>
\unitlength=1.00000cm
\linethickness=1pt
\setplotsymbol ({\makebox(0,0)[l]{\tencirc\symbol{'160}}})
\setshadesymbol ({\thinlinefont .})
\setlinear
%
% Fig CIRCULAR ARC object
%
\linethickness= 0.500pt
\setplotsymbol ({\thinlinefont .})
%
% arrow head
%
\plot  4.029 21.605  3.810 21.749  3.935 21.519 /
%
\circulararc 48.455 degrees from  4.286 20.320 center at  2.461 20.505
%
% Fig CIRCULAR ARC object
%
\linethickness= 0.500pt
\setplotsymbol ({\thinlinefont .})
%
% arrow head
%
\plot  2.801 22.566  2.540 22.543  2.781 22.441 /
%
\circulararc 46.397 degrees from  3.810 21.749 center at  2.249 20.664
%
% Fig CIRCULAR ARC object
%
\linethickness= 0.500pt
\setplotsymbol ({\thinlinefont .})
%
% arrow head
%
\plot  4.502 21.918  4.286 22.066  4.407 21.834 /
%
\circulararc 43.003 degrees from  4.921 20.320 center at  2.387 20.387
%
% Fig CIRCULAR ARC object
%
\linethickness= 0.500pt
\setplotsymbol ({\thinlinefont .})
\circulararc 61.928 degrees from  4.286 22.066 center at  3.016 20.849
%
% Fig CIRCULAR ARC object
%
\linethickness= 0.500pt
\setplotsymbol ({\thinlinefont .})
%
% arrow head
%
\plot 10.102 21.873  9.842 21.907 10.056 21.755 /
%
\circulararc 46.997 degrees from 11.430 20.320 center at  8.811 19.288
%
% Fig CIRCULAR ARC object
%
\linethickness= 0.500pt
\setplotsymbol ({\thinlinefont .})
%
% arrow head
%
\plot  7.722 22.565  7.461 22.543  7.702 22.440 /
%
\circulararc 11.812 degrees from  9.842 21.907 center at  5.583 10.716
%
% Fig CIRCULAR ARC object
%
\linethickness= 0.500pt
\setplotsymbol ({\thinlinefont .})
%
% arrow head
%
\plot  9.627 22.402  9.366 22.384  9.605 22.277 /
%
\circulararc 54.913 degrees from 12.065 20.320 center at  8.730 18.755
%
% Fig CIRCULAR ARC object
%
\linethickness= 0.500pt
\setplotsymbol ({\thinlinefont .})
%
% arrow head
%
\plot  7.706 22.635  7.461 22.543  7.721 22.508 /
%
\circulararc 22.620 degrees from  9.366 22.384 center at  8.017 17.701
%
% Fig CIRCULAR ARC object
%
\linethickness= 0.500pt
\setplotsymbol ({\thinlinefont .})
%
% arrow head
%
\plot  2.769 22.669  2.540 22.543  2.802 22.546 /
%
\circulararc 29.680 degrees from  7.461 22.543 center at  5.001 13.256
%
% Fig CIRCULAR ARC object
%
\linethickness= 0.500pt
\setplotsymbol ({\thinlinefont .})
%
% arrow head
%
\plot  4.691 22.190  4.921 22.066  4.776 22.284 /
%
\circulararc 61.928 degrees from  4.921 22.066 center at  3.334 20.320
%
% Fig CIRCULAR ARC object
%
\linethickness= 0.500pt
\setplotsymbol ({\thinlinefont .})
%
% arrow head
%
\plot  5.673 20.578  5.715 20.320  5.799 20.568 /
%
\circulararc 58.395 degrees from  5.715 20.320 center at  3.756 20.483
%
% Fig POLYLINE object
%
\linethickness= 0.500pt
\setplotsymbol ({\thinlinefont .})
\putrule from  2.540 24.130 to  2.540 20.320
%\putrule from  2.540 20.320 to  6.350 20.320
\putrule from  2.540 20.320 to  7.779 20.320
%
% Fig POLYLINE object
%
\linethickness= 0.500pt
\setplotsymbol ({\thinlinefont .})
\putrule from 10.160 24.130 to 10.160 20.320
%\putrule from 10.160 20.320 to 13.970 20.320
\putrule from 10.160 20.320 to 15.081 20.320
%
% Fig TEXT object
%
\put{\SetFigFont{12}{14.4}{rm}$(T_1,<_1')$} [lB] at  2.699 23.495
%
% Fig TEXT object
%
\put{\SetFigFont{12}{14.4}{rm}$(T_2,<_2')$} [lB] at 10.319 23.495
%
% Fig TEXT object
%
\put{\SetFigFont{12}{14.4}{rm}$(T_1,<_1)$} [lB] at  6.191 20.479
%
% Fig TEXT object
%
\put{\SetFigFont{12}{14.4}{rm}$(T_2,<_2)$} [lB] at 13.176 20.479
%
% Fig TEXT object
%
%\put{\SetFigFont{12}{14.4}{rm}$t$} [lB] at  2.223 22.543
\put{\SetFigFont{12}{14.4}{rm}$t$} [lB] at  2.023 22.543
%
% Fig TEXT object
%
%\put{\SetFigFont{12}{14.4}{rm}$r$} [lB] at  4.286 20.003
\put{\SetFigFont{12}{14.4}{rm}$r$} [lB] at  3.986 20.003
%
% Fig TEXT object
%
%\put{\SetFigFont{12}{14.4}{rm}$s$} [lB] at  4.921 20.003
\put{\SetFigFont{12}{14.4}{rm}$s$} [lB] at  4.621 20.003
%
% Fig TEXT object
%
%\put{\SetFigFont{12}{14.4}{rm}$t$} [lB] at  5.715 20.003
\put{\SetFigFont{12}{14.4}{rm}$t$} [lB] at  5.415 20.003
%
% Fig TEXT object
%
%\put{\SetFigFont{12}{14.4}{rm}$u$} [lB] at 11.430 20.003
\put{\SetFigFont{12}{14.4}{rm}$u$} [lB] at 11.130 20.003
%
% Fig TEXT object
%
%\put{\SetFigFont{12}{14.4}{rm}$v$} [lB] at 12.065 20.003
\put{\SetFigFont{12}{14.4}{rm}$v$} [lB] at 11.765 20.003
%
% Fig TEXT object
%
%\put{\SetFigFont{12}{14.4}{rm}$k_{1,1}$} [lB] at  3.886 21.273
\put{\SetFigFont{12}{14.4}{rm}$k_{1,1}$} [lB] at  3.886 21.273
%
% Fig TEXT object
%
\put{\SetFigFont{12}{14.4}{rm}$k_{1,2}$} [lB] at  7.461 22.860
%
% Fig TEXT object
%
\put{\SetFigFont{12}{14.4}{rm}$k_{1,1}(t)=$} [lB] at  0.318 22.241
%\put{\SetFigFont{12}{14.4}{rm}$k_{1,1}(t)=$} [lB] at  0.318 21.749
%
% Fig TEXT object
%
\put{\SetFigFont{12}{14.4}{rm} $\{r,s\}$} [lB] at  0.318 21.749
%\put{\SetFigFont{12}{14.4}{rm} $\{r,s\}$} [lB] at  0.318 21.241
%
% Fig TEXT object
%
\put{\SetFigFont{12}{14.4}{rm}$k_{1,2}(t)=$} [lB] at  0.318 21.033
%
% Fig TEXT object
%
\put{\SetFigFont{12}{14.4}{rm} $\{u,v\}$} [lB] at  0.318 20.525
\linethickness=0pt
\putrectangle corners at  0.318 24.155 and 13.995 19.952
\endpicture}

%
%
%
%%%%begin pic5.pictex
\noindent \font\thinlinefont=cmr5
%
\begingroup\makeatletter\ifx\SetFigFont\undefined
% extract first six characters in \fmtname
\def\x#1#2#3#4#5#6#7\relax{\def\x{#1#2#3#4#5#6}}%
\expandafter\x\fmtname xxxxxx\relax \def\y{splain}%
\ifx\x\y   % LaTeX or SliTeX?
\gdef\SetFigFont#1#2#3{%
  \ifnum #1<17\tiny\else \ifnum #1<20\small\else
  \ifnum #1<24\normalsize\else \ifnum #1<29\large\else
  \ifnum #1<34\Large\else \ifnum #1<41\LARGE\else
     \huge\fi\fi\fi\fi\fi\fi
  \csname #3\endcsname}%
\else
\gdef\SetFigFont#1#2#3{\begingroup
  \count@#1\relax \ifnum 25<\count@\count@25\fi
  \def\x{\endgroup\@setsize\SetFigFont{#2pt}}%
  \expandafter\x
    \csname \romannumeral\the\count@ pt\expandafter\endcsname
    \csname @\romannumeral\the\count@ pt\endcsname
  \csname #3\endcsname}%
\fi
\fi\endgroup
\mbox{\beginpicture
\setcoordinatesystem units <0.90000cm,0.90000cm>
\unitlength=1.00000cm
\linethickness=1pt
\setplotsymbol ({\makebox(0,0)[l]{\tencirc\symbol{'160}}})
\setshadesymbol ({\thinlinefont .})
\setlinear
%
% Fig CIRCULAR ARC object
%
\linethickness= 0.500pt
\setplotsymbol ({\thinlinefont .})
%
% arrow head
%
\plot  7.555 22.406  7.779 22.543  7.517 22.527 /
%
\circulararc 18.325 degrees from  7.779 22.543 center at  9.446 17.224
%
% Fig CIRCULAR ARC object
%
\linethickness= 0.500pt
\setplotsymbol ({\thinlinefont .})
%
% arrow head
%
\plot  6.011 21.559  6.191 21.749  5.943 21.667 /
%
\circulararc 25.361 degrees from  6.191 21.749 center at  8.652 17.859
%
% Fig CIRCULAR ARC object
%
\linethickness= 0.500pt
\setplotsymbol ({\thinlinefont .})
%
% arrow head
%
\plot  9.913 22.932 10.160 23.019  9.901 23.058 /
%
\circulararc 11.812 degrees from 10.160 23.019 center at 11.271 11.271
%
% Fig CIRCULAR ARC object
%
\linethickness= 0.500pt
\setplotsymbol ({\thinlinefont .})
%
% arrow head
%
\plot 11.109 21.856 10.954 22.066 10.992 21.807 /
%
\circulararc 14.250 degrees from 11.430 20.320 center at  4.207 19.288
%
% Fig CIRCULAR ARC object
%
\linethickness= 0.500pt
\setplotsymbol ({\thinlinefont .})
%
% arrow head
%
\plot 10.398 22.909 10.160 23.019 10.318 22.810 /
%
\circulararc 22.620 degrees from 10.954 22.066 center at  8.176 20.558
%
% Fig CIRCULAR ARC object
%
\linethickness= 0.500pt
\setplotsymbol ({\thinlinefont .})
%
% arrow head
%
\plot 10.418 23.066 10.160 23.019 10.409 22.939 /
%
\circulararc 45.667 degrees from 11.430 22.384 center at 10.041 21.193
%
% Fig CIRCULAR ARC object
%
\linethickness= 0.500pt
\setplotsymbol ({\thinlinefont .})
%
% arrow head
%
\plot 11.629 22.214 11.430 22.384 11.526 22.140 /
%
\circulararc 36.870 degrees from 12.065 20.320 center at  8.652 20.399
%
% Fig CIRCULAR ARC object
%
\linethickness= 0.500pt
\setplotsymbol ({\thinlinefont .})
%
% arrow head
%
\plot 12.460 22.172 12.700 22.066 12.538 22.272 /
%
\circulararc 34.432 degrees from 12.700 22.066 center at  9.893 18.444
%
% Fig CIRCULAR ARC object
%
\linethickness= 0.500pt
\setplotsymbol ({\thinlinefont .})
% arrow head
%
\plot 14.084 20.484 14.287 20.320 14.185 20.561 /
%
\circulararc 10.389 degrees from 14.287 20.320 center at  3.889 12.462
%
% Fig CIRCULAR ARC object
%
\linethickness= 0.500pt
\setplotsymbol ({\thinlinefont .})
%
% arrow head
%
\plot  4.857 21.453  5.080 21.590  4.819 21.574 /
%
\circulararc 36.870 degrees from  5.080 21.590 center at  6.112 18.336
%
% Fig POLYLINE object
%
\linethickness= 0.500pt
\setplotsymbol ({\thinlinefont .})
\putrule from  2.540 24.130 to  2.540 20.320
\putrule from  2.540 20.320 to  7.779 20.320
%
% Fig POLYLINE object
%
\linethickness= 0.500pt
\setplotsymbol ({\thinlinefont .})
\putrule from 10.160 24.130 to 10.160 20.320
\putrule from 10.160 20.320 to 15.081 20.320
\linethickness= 0.500pt
\setplotsymbol ({\thinlinefont .})
%
% Fig INTERPOLATED PT SPLINE
%
\plot  5.080 21.590 	 5.196 21.642
	 5.303 21.690
	 5.403 21.734
	 5.496 21.775
	 5.581 21.813
	 5.661 21.848
	 5.734 21.880
	 5.802 21.909
	 5.922 21.960
	 6.025 22.002
	 6.114 22.037
	 6.191 22.066
	 6.298 22.104
	 6.422 22.145
	 6.492 22.167
	 6.568 22.191
	 6.650 22.216
	 6.739 22.243
	 6.836 22.272
	 6.941 22.303
	 7.055 22.336
	 7.179 22.372
	 7.244 22.391
	 7.312 22.410
	 7.383 22.430
	 7.456 22.451
	 7.533 22.473
	 7.612 22.495
	 7.694 22.519
	 7.779 22.543
	/
%
% arrow head
%
\plot  7.551 22.412  7.779 22.543  7.517 22.535 /
%
%
% Fig TEXT object
%
\put{\SetFigFont{12}{14.4}{rm}$(T_1,<_1')$} [lB] at  2.699 23.495
%
% Fig TEXT object
%
\put{\SetFigFont{12}{14.4}{rm}$(T_2,<_2')$} [lB] at 10.319 23.495
%
% Fig TEXT object
%
\put{\SetFigFont{12}{14.4}{rm}$(T_1,<_1)$} [lB] at  6.191 20.479
%
% Fig TEXT object
%
\put{\SetFigFont{12}{14.4}{rm}$k_{2,1}(w)=$} [lB] at  0.318 22.241
%\put{\SetFigFont{12}{14.4}{rm}$k_{2,1}(w)=$} [lB] at  0.318 21.749
%
% Fig TEXT object
%
\put{\SetFigFont{12}{14.4}{rm} $\{r,s\}$} [lB] at  0.318 21.749
%\put{\SetFigFont{12}{14.4}{rm} $\{r,s\}$} [lB] at  0.318 21.241
%
% Fig TEXT object
%
\put{\SetFigFont{12}{14.4}{rm}$k_{2,2}(w)=$} [lB] at  0.318 21.033
%
% Fig TEXT object
%
\put{\SetFigFont{12}{14.4}{rm} $\{u,v\}$} [lB] at  0.318 20.525
%
% Fig TEXT object
%
%%%\put{\SetFigFont{12}{14.4}{rm}$(T_2,<_2)$} [lB] at 14.446 20.479
\put{\SetFigFont{12}{14.4}{rm}$(T_2,<_2)$} [lB] at 14.087 21.003
%
% Fig TEXT object
%
%\put{\SetFigFont{12}{14.4}{rm}$w$} [lB] at  9.842 23.178
\put{\SetFigFont{12}{14.4}{rm}$w$} [lB] at  9.692 23.198
%
% Fig TEXT object
%
%\put{\SetFigFont{12}{14.4}{rm}$r$} [lB] at  3.334 20.003
\put{\SetFigFont{12}{14.4}{rm}$r$} [lB] at  3.234 20.003
%
% Fig TEXT object
%
%\put{\SetFigFont{12}{14.4}{rm}$s$} [lB] at  4.763 20.003
\put{\SetFigFont{12}{14.4}{rm}$s$} [lB] at  4.663 20.003
%
% Fig TEXT object
%
%\put{\SetFigFont{12}{14.4}{rm}$u$} [lB] at 11.430 20.003
\put{\SetFigFont{12}{14.4}{rm}$u$} [lB] at 11.330 20.003
%
% Fig TEXT object
%
%\put{\SetFigFont{12}{14.4}{rm}$v$} [lB] at 12.065 20.003
\put{\SetFigFont{12}{14.4}{rm}$v$} [lB] at 11.965 20.003
%
% Fig TEXT object
%
%\put{\SetFigFont{12}{14.4}{rm}$w$} [lB] at 14.287 20.003
\put{\SetFigFont{12}{14.4}{rm}$w$} [lB] at 14.187 20.003
%
% Fig TEXT object
%
\put{\SetFigFont{12}{14.4}{rm}$k_{2,2}$} [lB] at 11.271 21.273
%
% Fig TEXT object
%
\put{\SetFigFont{12}{14.4}{rm}$k_{2,1}$} [lB] at  4.445 21.907
\linethickness=0pt
%\putrectangle corners at  0.318 24.155 and 15.107 19.964
%\putrectangle corners at  0.318 24.155 and 13.995 19.952
\endpicture}

%
%
%
\noindent For simplicity assume $\T_0 \cap \T_1 = \emptyset$ and write 
$\prec$, $\prec '$, $\length$  instead of $\prec_j$, $\prec_j'$,
$\length_j$ and write $\restriction_j$ instead of $\restriction_{i,j}$ as
defined before.\par 
In the example $\calF_j$ fulfilled all conditions of an OS except (OS 4). 
We used that, if $A \subseteq \T_0$,
$B \subseteq \T_1$ are well-ordered, then
$(\T_0 \restriction_0 A) \restriction_1 B$
is $\prec '$-well-ordered. 
In order to have that $\calF_1[A]$ is
an OS, we need as well that under the same conditions for
$A$, $B$ $(\T_1 \restriction_0 A) \restriction_1 B$ is well-ordered.
Further we needed that  $(\T_0 \restriction_0 A) \restriction_1
(\T_1 \restriction_0 A) = \T_0 \restriction_0 A$ for 
all $A \subseteq \T_0$. 
Only $\supseteq$ needs to be fulfilled, which is equivalent to
$\k_{1,0}[\k_{0,1}(a)] \subseteq \k_{0,0}(a)$.
In order to get that $\calF_1[A]$ is an OS-structure,
we need $\k_{1,1}: \T_1 \restriction_0 A \ar_\omega \T_1 \restriction_0 A$
for all $A \subseteq \T_0$, 
i. e. $\k_{1,0}[\k_{1,1}(a)] \subseteq \k_{1,0}(a)$.
The  generalization to $n$-ordinal systems is now straight-forward  and we
get  the following definition: 
%So we get the following conditions:\\
%\condition{2-\OS\ 1}$\calF_i$ fulfill (OS
% 1)  and (OS 3)\\
%\condition{2-\OS\ 2}$\length[\k_{i,j}(a)] < \length(a)$ for
%$j=0,1$, $a \in \T_i$.
%\condition{2-\OS\ 3}$\k_{1,0} \circ \k_{1,1} \subseteq \k_{1,0}$ and
%$\k_{1,0} \circ \k_{0,1} \subseteq \k_{0,0}$.\\
%\condition{2-\OS\ 4}If $A_i \subseteq \T_i$ are $\prec$-well-ordered,
%$(\T_j \restriction_0 A_0) \restriction_1 A_1$ are
%$\prec'$-well-ordered.\\ 
%
%
%
%%%%%%%%%%%%%%%%%%%%%%%%%%%%%%%%%%%%%%%%%%%
%%
%                                         %
%%%%%%%%%%%%%%%%%%%%%%%%%%%%%%%%%%%%%%%%%%%

%
%
%
\begin{definition}{\rm 
\laprint{defnos}
\begin{enumerate}
\ssubitem a An \underlinedef{$n$-times iterated Ordinal-System-structure},
in short 
\underlinedef{$n$-OS-struc-}\break \underlinedef{ture} 
is a triple $\calF = ((\calG_i)_{i < n},(\k_{i,j})_{i,j < n},
(\length_i)_{i < n})$
such that $\calF_i:= $\\
$(\calG_i,\k_{i,i},\length_i)$ are OS-structures and
$\k_{i,j}: \T_i \ar_\omega  \T_j$ ($i,j<n$).\\
In the following, when introduced as an $n$-OS,
$\calF$ will be always as above (where $n$ can be replaced by
any natural number), 
$\calF_i= (\calG_i,\k_{i,i})$, 
$\calG_i = (\T_i,\prec_i,\prec_i')$ and 
and we will usually write 
$((\calG_i),(\k_{i,j}),\length_i)$ instead of 
$((\calG_i)_{i <n},(\k_{i,j})_{i,j < n})$. Further we assume
$\T_i$ are always disjoint and write $\prec$, $\prec '$, $\length$
instead of $\prec_i$, $\prec'_i$, $\length_i$.
\ssubitem {ab}
If $\calF$ is an $n$-OS-structure  as above,
$A \subseteq \T_i$, $B \subseteq \T_j$, $A \restriction_{j} B:= 
A \cap \k_{i,j}^{-1}[B]$. 
If $B_j \subseteq \T_j$,  then
$A \restriction_{j=l}^{m} B_j:= A \cap \bigcap_{j=l}^m \k_{i,j}^{-1}[B_j]$. 
We write $\restriction_{i<j}$, $\restriction_{i\leq j}$
for $\restriction_{i=0}^{j-1}$ and $\restriction_{i=0}^{j}$.
If $\calF$ is an
$n+1$-OS-structure and $A \subseteq \T_0$,
then  
$\calF [A]:= 
((\T_{i+1} \restriction_0 A$, $\prec_{i+1}$, $\prec_{i+1}')_{i<n-1}$,
$(\k_{{i+1}, {j'+1}})_{i,j'<n-1}$, $(\length_{i+1})_{i<n-1}))$.
(More precisely we have to restrict 
$\prec_{i+1}$, $\prec'_{i+1}$, $\k_{i+1,j'+1}$ and $\length_{i+1}$
to $\T_{i+1} \restriction_0 A$.)\par
%\ssubitem{ac} Define simultaneously inductively sets 
%$\Cl^{\calGvec,\krmvec}_i$ $i < n$ 
%or, if $\calGvec,\krmvec$ is the usual choice,
%$\Cl_i$ inductively by: 
%If $t \in \T_i$, $\k_{i,j}(t) \subseteq \Cl_j$ for
%all $j<n$, then
%$t \in \Cl_i$.
\ssubitem b An $n$-OS-structure $\calF$ is
an \underlinedef{$n$-times iterated Ordinal System},
in short \underlinedef{$n$-OS}, if for
all $i,j,l< n$ the following holds\\
\condition{n-\OS\ 1}$\calF_i$
fulfill (OS 1), (OS 2) and (OS 3);\\
\condition{n-\OS\ 2}if
$l < j$, $i,j<n$, then 
$\k_{j,l} \circ \k_{i,j} \subseteq \k_{i,l}$;\\
\condition{n-\OS\ 3}if $A_i\subseteq \T_i$ are $\prec$-well-ordered
($i<n$), $l<n$, then
$\T_l  \restriction_{i<n}A_i$ is $\prec'$-well-\break 
\conditionbox{}ordered.
\ssubitem c An $n$-OS $\calF$
is {\em well-ordered}, if $(\T_0,\prec)$ is, and its {\em order type} is
that of $(\T_0,\prec)$.
It is {\em primitive recursive}, if
the involved sets are primitive recursive subsets of the natural numbers,
all functions, relations are primitive recursive and all properties
except of the well-ordering condition can be shown in
primitive recursive arithmetic. It is {\em elementary}, if additionally
the well-ordering condition follows in $\PRA$ in the sense of
reducibility of transfinite induction in $\PRA$.
\ssubitem d Two $n$-OS are {\em isomorphic}, if there is are bijections
between the underlying sets, which respect $\k_{i,j}$, $\length_i$,
$\prec_i$, $\prec_i'$ for all $i,j < n$.
\end{enumerate}}
\end{definition}
%
%
%
We will need the following auxiliary definition:
%
%
%
%%%%%%%%%%%%%%%%%%%%%%%%%%%%%%%%%%%%%%%%%%%
%%
%                                         %
%%%%%%%%%%%%%%%%%%%%%%%%%%%%%%%%%%%%%%%%%%%
%
%
%
\begin{definition}
\laprint{defrelativizednos}{\rm
\begin{enumerate} 
\ssubitem a
A {\em relativized  $n$-OS}  is a triple $(\calF,\T_{-1},\prec,
(\k_{i,-1})_{0 \leq i < n})$ such that $\calF$ is an $n$-OS structure,
$(\T_{-1},\prec)$ is a linearly ordered set,
$\k_{i,-1}: \T_i \ar_\omega \T_{-1}$,
for $j>-1$ $\k_{j,-1} \circ \k_{i,j} \subseteq  \k_{i,-1}$ and such that
$\calF$ fulfills the conditions of an $n$-OS 
except for condition$(n-\OS\ 3)$, which is replaced by:
if $A_i\subseteq \T_i$ are $\prec$-well-ordered
($-1 \leq i<n$), 
then $\T_l  \restriction_{i=-1}^{n-1}A_i$ is $\prec'$-well-ordered
($0 \leq l < n$). 
\ssubitem b If $(\calF,\T_{-1},\k_{i,-1})$ is a relativized $n$-OS,
$A \subseteq \T_{-1}$, 
then $\calF[A]_{-1}$ is the relativization of $\calF$ to $A$ defined in
a straightforward way. 
\end{enumerate} 
}
\end{definition}
%
%
%
Well-ordering of a $n+1$-OS reduces to the well-ordering of an
$n$-OS as follows:
%
%
%%%%%%%%%%%%%%%%%%%%%%%%%%%%%%%%%%%%%%%%%%%
%%
%                                         %
%%%%%%%%%%%%%%%%%%%%%%%%%%%%%%%%%%%%%%%%%%%
%
%
%
\begin{lemma}
\laprint{nosisos}
Assume $\calF$ is an $n+1$-OS-structure.
\begin{enumerate} 
\ssubitem a If $A \subseteq \T_0$ and
$\calF$ fulfills $((n+1)-\OS\ 2)$, then $\calF[A]$ is
an $n$-OS-structure which fulfills $(n-\OS\ 2)$.
\ssubitem b If $\calF$ is an $n+1$-OS and
$A \subseteq \T_0$ is $\prec$-well-ordered, then
$\calF[A]$ is an $n$-OS.
\ssubitem c If $\calF$ is an $1$-OS, then
$\calF$ is an OS.
\ssubitem d
$n$-OS are well-ordered.
\ssubitem e For elementary $n$-OS, \refact d can be shown
in $\kpomega$ extended by the existence of $n-1$ admissible sets
$x_1 ,\ldots, x_{n-1}$ such that $x_1 \in x_2 \in \cdots \in x_{n-1}$.
\end{enumerate} 
\end{lemma}
%
%
%
{\bf Proof:} 
\refact a We need only to show that 
$\k_{i+1,j+1}: \T_{i+1}  \restriction_0 A \ar \T_{j+1} \restriction_0 A$,
which follows by $\k_{j+1,0}(\k_{i+1,j+1}(a)) \subseteq \k_{i+1,0}(a)$.
%We show, $\all x \in \Cl_{\calF_i}(\k_{i,0}(x) \subseteq A \ar 
%x \in \Cl_{\calF_1[A]})$ by induction on
%$x \in \Cl_{\calF_i}$. Now, if this holds for $\k_{i,i}(x)$,
%and $\k_{i,0}(x) \subseteq A$ by
%$\k_{i,0}[\k_{i,i}(x)]\subseteq \k_{i,0}(x) \subseteq A$ and the
%IH follow $\k_{i,i}(x) \in \Cl_{\calF_i[A]}$ and the assertion.\\ 
\refact b easy.
%($n$-OS 1) and ($n$-OS 2) are trivially fulfilled.\\
%Proof of ($n$-OS 3): Assume 
%$B_i \subseteq \T_{i+1}[A]$ is well-ordered, 
%then\\
%$\bigcap_{j < n}\k_{i+1,j+1}[A]^{-1}(B_j)= 
%\bigcap_{j < n}\k_{i+1,j+1}^{-1}(B_j) \cap 
%\k_{i+1,0}^{-1}(A)$ is with respect to the ordering
%$\prec_{i+1}'$ well-ordered.\\ 
\refact c: trivial.\\
\refact d We show by induction on $n \geq 1$:
If $\calF$ is an $n$-OS, then $(\T_i,\prec)$ are well-ordered for $i<n$.
$n=1$: \refact c and Theorem \refcom{lemintuitos}a. 
$n \ar n+1$: We show first $\calF_0$ is an OS. We have only to show
(OS 4). Assume $A \subseteq \T$ is $\prec$-well-ordered.
Then $\calF[A]$ is an $n$-OS, by IH it follows that 
$\T_i \restriction_0 A$ are well-ordered ($i=1 ,\ldots, n+1$).
Therefore $(\T_0 \restriction_0 A)\restriction_{i=1}^n (\T_i \restriction_0 A)$
is $\prec'$-well-ordered. But
$(\T_0 \restriction_0 A)\restriction_{i=1}^n (\T_i \restriction_0 A)
= \T_0 \restriction_0 A$, since $\k_{i,0}(\k_{0,i} (a)) \subseteq \k_{0,0}(a)$.\\ 
\refact e We show by Meta-induction on $n$ in $\kpomega$:
If $(\calF,\T_{-1},\prec,\k_{i,-1})$ is
a relativized $n$-OS, $A \subseteq \T_{-1}$ is well-ordered, and there are
$n-1$ admissibles above $\{ A,\calF,\k_{i,-1} \}$, then
$\T_i \restriction_{-1}A$ are $\prec$-well-ordered.\\
The case $n=1$ follows as in Theorem \refcom{lemintuitos}c. 
In the step $n \ar n+1$ 
let $\kappa$ be the least admissible ordinal
such that $\{ A,\calF,\k_{i,-1} \} \in \L_\kappa$. 
Define $a_\alpha$ as in  Lemma \refcom{lemintuitos}b
for $\alpha \leq \kappa$. This can be done, by the same argument
as in \refact d, using that $\calF[A]_{-1}[B]$ is an $n$-OS for
$B \subseteq \T_0$ $\prec$-well-ordered and that therefore,
if $B \in \L_{\kappa^+}$,
$(\T_i \restriction_{-1} A) \restriction_{0}B$ are well-ordered
($i=1 ,\ldots, n$),
where $\kappa^+$ is the next  admissible ordinal above $\kappa$.
If we replace now in the proof of  \refcom{lemintuitos}b 
$C$ by
$ \{ \alpha \in \kappa \mid A_{<\alpha} $ increasing$ \}$,
and the argument that there is no bijection between a set and $\Ord$
by that there is no ($\Delta_0$-definable) 
bijection between an element of $a$ and
$\kappa$, we conclude that 
$\{ a_\alpha \mid \alpha \in C \} = \T_0 \restriction_{-1} A$,
$\T_0 \restriction_{-1}A$ is well-ordered and therefore 
as well $\T_i \restriction_{-1}A$ are well-ordered.\QED
%
%
%
%%%%%%%%%%%%%%%%%%%%%%%%%%%%%%%%%%%%%%%%%%%
%%
%                                         %
%%%%%%%%%%%%%%%%%%%%%%%%%%%%%%%%%%%%%%%%%%%
%
%
%
\begin{theorem}
\laprint{theostrengthnordsys}
The strength of an $n$-OS ($n>0$) is less
than $|\ID_n| = \psi_{\Omega_1}(\epsilon_{\Omega_i +1})$
($\psi$ as in \cite{buchholzasimplified}).
\end{theorem}
%
%
%
{\bf Proof:}\\
By Lemma \refcom{nosisos}e and, since the strength of $\kpomega$ extended
by the existence of $n-1$ admissibles is $|\ID_n|$.\QED\par 
\medskip 
%
%
%From the well-ordering proof before we get the following procedure, which
%shows the well-ordering of all $\T_i$ in a way, which is completely from below:
%We start by defining $a_{\alpha_0} \in \T_0$ as for an OS. However,
%in order to get a well-defined definition we need that
%$\T_1 \restriction A_{\alpha_0},\ldots, \T_n \restriction A_{\alpha_0}$
%are well-ordered. So before defining $a_{\alpha_0}$, we will 
%we start a subprocess in which we enumerate in an increasing
%order all elements of $\T_i \restriction A_{\alpha_0}$ and
%therefore show that they are well-ordered. 
%We first define $a_{\alpha_0,\alpha_1} \in \T_1$ for the
%OS-structre $(\T_1 \restriction_0 A_{\alpha_0},\prec,\prec ',\k_1,
%\length)$, using again the process as for OS. 
%And as before, in order to make it work, we
%first define by recursion on  $\alpha_2$ 
%$a_{\alpha_0,\alpha_1,\alpha_2} \in \T_2$ for the OS structure
%$((\T_2 \restriction_0 A_{\alpha_0}) \restriction_1 A_{\alpha_0,\alpha_1} ,
%\ldots $
%etc. Since we have only finitely many OS-structures, eventually
%we will arrive at the last OS, enumerate all elements of $\T_{n-1}$ 
%relativized to the sets defined before, and then can proceed one step
%further with the definition of $a_{\alpha_0 ,\ldots, \alpha_{n-2}}$. 
%Again this process stops and we can 
%finally proceed with the definition of 
%$a_{\alpha_0 ,\ldots, \alpha_{n-3}}$ etc.
%At the end we get that $A_{\alpha_0,\alpha_1} = 
%\T_1 \restriction A_{\alpha_0}$ for some $\alpha_1$.
%Now we enumerate all elements
%of $\T_2 \restriction_0 A_{\alpha_0}
%= (\T_2 \restriction_0 A_{\alpha_0}) \restriction_1 A_{\alpha_0,\alpha_1}$,
%then of $\T_3 \restriction_0 A_{\alpha_0}$ etc. in the same
%way as before and once we have finished we can 
%select $a_{\alpha_0}$. Now we have to start to define
%$a_{\alpha_0+1,\alpha_1}$, $a_{\alpha_0+1,\alpha_1,\alpha_2}$  as before
%from scratch.\par 
%The following Lemma formalizes this argument:
%%
%%
%%
%%%%%%%%%%%%%%%%%%%%%%%%%%%%%%%%%%%%%%%%%%%%
%%%
%%                                         %
%%%%%%%%%%%%%%%%%%%%%%%%%%%%%%%%%%%%%%%%%%%%
%%
%%
%%
%\begin{lemma}
%\laprint{lemintuitnos}
%\begin{enumerate}
%\ssubitem a Let $\calF$ be an $n$-OS,
%$\bot \not \in \T_i$.  
%Let $\lesstil$ be the following ordering on
%$\Ord^n:= \{ (\alpha_0 ,\ldots, \alpha_k) \mid k<n \land \all i \leq k.
%\alpha_i \in \Ord  \}$:
%$(\alpha_0 ,\ldots, \alpha_k) \lesstil 
%(\beta_0 ,\ldots, \beta_l): \Iff 
%l<k \land \all i \leq l(\beta_i = \alpha_i) \lor 
%\exists j \leq \min \{ k,l \}(\all i<j(\alpha_i = \beta_i) \land 
%\alpha_j < \beta_j)$.\\
%Then $\lesstil$  is a well-ordering, and by recursion on 
%$(\alpha_0,\ldots, \alpha_{k})$ 
%we define 
%$a_{\alpha_0 ,\ldots, \alpha_k} \in \T_k \cup \{ \bot \}$ as follows:\\
%Let $A_{\alpha_0 ,\ldots, \alpha_k}:= 
%\{ a_{\alpha_0 ,\ldots, \alpha_{k-1},\gamma} \mid \gamma  < \alpha_k \} \setminus \{ \bot \}$.\\
%We define that a  sequence $(\beta_0 ,\ldots, \beta_l)$ is ``good'' iff
%\begin{itemize}
%\item For all $i=0 ,\ldots, l$ and all $\gamma < \gamma' < \beta_i$ we have:
%if $a:= a_{\beta_1 ,\ldots, \beta_{i-1},\gamma'} \not = \bot$,
%then $b:= a_{\beta_1 ,\ldots, \beta_{i-1},\gamma} \not = \bot$
%and $b \prec_i a$.
%\item For all $i=0 ,\ldots, l$, $A_{\beta_0 ,\ldots, \beta_i}
%\segment \bigcup_{j<i}A_{\beta_0 ,\ldots, \beta_j}$.
%\end{itemize} 
%These definitions are well-defined, if we have defined the $a_{\gamma_0
%,\ldots, \gamma_i}$ involved already.\\
%Assume now, that we have defined $a_\gammavec$ for 
%$\gammavec \lesstil (\alpha_0 ,\ldots, \alpha_k)$.
%Define now inductively for  $i= k+1 ,\ldots, n-1$\\ 
%$\alphabar_i[\alpha_0 ,\ldots, \alpha_k]:=$\\
%\hbox{\qquad}$\min \{ \gamma \mid a_{\alpha_0 ,\ldots, \alpha_{k},\alphabar_{k+1}[\alpha_0
%,\ldots, \alpha_k] ,\ldots, \alphabar_{i-1}[\alpha_0 ,\ldots, \alpha_k],
%\gamma }= \bot  
%\lor $\\
%\hbox{\qquad }$(\alpha_0 ,\ldots, \alpha_{k},\alphabar_{k+1}[\alpha_0 ,\ldots, 
%\alpha_k] ,\ldots, \alphabar_{i-1}[\alpha_0 ,\ldots, \alpha_k],\gamma  )$
%is not good $\} $.\\
%Let for $i=k+1 ,\ldots, n-1$, 
%$\alpha_i:= \alphabar_i[\alpha_0 ,\ldots, \alpha_k]$.\\
%$B:= (\T_k \restriction_{i<n}A_{\alpha_0 ,\ldots, \alpha_i}) \setminus 
%A_{\alpha_0 ,\ldots, \alpha_k}$.\\
%If $B = \emptyset$ or $(\alpha_0 ,\ldots, \alpha_n)$ is not
%good, then $a_{\alpha_0 ,\ldots, \alpha_k}:= \bot$,
%otherwise $a_{\alpha_0 ,\ldots, \alpha_k}:= \min B$.
%%\begin{cases}
%%a_{\alpha_0 ,\ldots, \alpha_k}:= 
%%\min_{\prec '}((\T_k \restriction_{i \leq k} A_{\alpha_0 ,\ldots, \alpha_i}) 
%%\setminus A_{\alpha_0 ,\ldots, \alpha_k})
%%&\text{if
%%$A_{\alpha_0 ,\ldots, \alpha_k}$ is increasing and}\\
%%&T_k \restriction_{i \leq k}A_{\alpha_0 ,\ldots, 
%%\alpha_i} \not \subseteq A_{\alpha_0 ,\ldots, \alpha_k},\\
%%\bot&\text{otherwise.}
%%\end{cases} $\\
%Then the following holds:
%\begin{enumerate} 
%\item $a_{\alpha_0 ,\ldots, \alpha_k}$ are well-defined;\\
%\end{enumerate} 


%%Then for all $k<n$, $\alpha_0 ,\ldots, \alpha_k \in \Ord$
%%there exist $\alpha_{k+1} ,\ldots, \alpha_{n-1}$ such that the
%%following holds
%%\begin{itemize} 
%%\item $A_{\alpha_0 ,\ldots, \alpha_j}$ is a segment of 
%%$\T_j \restriction_{i<j} A_{\alpha_0 ,\ldots, \alpha_i}$ for
%%$j=0 ,\ldots, k$,
%%\item $A_{\alpha_0 ,\ldots, \alpha_j} = \T_j \restriction_{i<k}
%%A_{\alpha_0 ,\ldots, \alpha_i}$ for $j = k+1 ,\ldots, n$
%%\item $(\T_k \restriction_{i \leq k}A_{\alpha_0 ,\ldots, \alpha_i})= 
%%(\T_k \restriction_{i < n}A_{\alpha_0 ,\ldots, \alpha_i})$ and therefore
%%$\prec'$ well-ordered.
%%\item $a_{\alpha_0 ,\ldots, \alpha_k}$ is well-defined.
%%\end{itemize} 
%%Especially therefore $\T_k = A_{\alpha_0 ,\ldots, \alpha_k}$ for
%%some $\alpha_0 ,\ldots, \alpha_k$ and is therefore well-orderd.
%\ssubitem b \refact a can be shown for 
%elementary $n$-OS in $\kpomega$ extended by the
%existence by an increasing sequences of $n-1$ admissibles.
%\end{enumerate}
%\end{lemma}
%%
%%
%%
%{\bf Proof:} Induction on $(\alpha_1 ,\ldots, \alpha_k)$ with
%ordering $\lesstil$. Following the lines of the last proof the
%assertion follows with $\alpha_{k+1} ,\ldots, \alpha_n$ being
%an increasing series of admissible ordinals bigger than
%$\alpha_1 ,\ldots, \alpha_k$ such that
%$cal
%$

%Unwinding of the last proof gives both results. \QED \par 
%Note that,
%although we have in the process to iterated the process for
%$\alpha_1 \leq \kappa_1$,
%$\alpha_2 \leq \kappa_2$, \ldots,
%for some increasing series of admissible ordinals $\omega_1^\ck =  \kappa_1
%,\kappa_2 ,\ldots$,
%the proof shows that the recursion terminates for $\alpha_1  \leq \gamma_1$,
%$\alpha_2 \leq \gamma_2$,\ldots with
%$\gamma_i < \omega_1^\ck$. Therefore the $\kappa_i$ are needed only
%for the proof, not for the recursion carried out and therefore
%play only the role of ``ordinals big enough''.\par 
%%
%%
%%Note, that we can show for 
%%$\alpha_1 ,\ldots, \alpha_n$
%%and $k=n,n-1 ,\ldots, 1$ that there
%%exists a $\gamma<\omega_1^\ck$ such that
%%$A_{\alpha_1 ,\ldots, \alpha_{k-1},<\gamma} = 
%%\T_k \restriction_{i=0}^{k-1} A_{\alpha_1 ,\ldots, \alpha_i}$:
%%If we have shown this for $k'>k$ we conclude
%%that $\T_k'\restriction_{i=0}^{k-1} A_{\alpha_1 ,\ldots, \alpha_i}$
%%are well-ordered, therefore the process of defining
%%$a_{\alpha_1 ,\ldots, \alpha_{k-1},\gamma}$ is  well-defined stops
%%for some $\gamma<\omega_1^\ck$ and therefore
%%$A_{\alpha_1 ,\ldots, \alpha_{k-1},<\gamma}$ exhausts $\T_k$
%%restricted as before. 
%%
%%
%%

\subsection{Constructive Well-ordering Proof.}

We will give now a proof of Lemma \refcom{nosisos}e which can be 
formalized even in constructive theories like Martin-L{\"o}f's type theory 
extended by an at most $n$ times nested ${\mathrm W}$-type or
intuitionistic $\ID_n$:\\ 
Define inductively $\M_i, \Acc_i \subseteq
\T_i$: $\M_{i}:= \T_i \restriction_{j<i} \Acc_j$,
$\Acc_i:= \Acc_{\prec}(\M_i)$.
Let $\Acc_i':= \T_i \restriction_{j<n}\Acc_j$.
$(\Acc_i,\prec)$, 
$(\Acc_i',\prec')$ are well-ordered.
We show by induction on $n-i$ 
$\Acc_i = \M_i$. Assume $i$ according to induction. 
We show $\all s \in \Acc_i'.s \in \Acc_i$ by side-induction
on $(\Acc_i',\prec')$:\\
$s \in \M_i$. Further
$\all r \in \M_i(r \prec s \ar r \in \Acc_i)$ by 
(side-side-)induction on $\length(r)$: Assume
$r$ according to induction,
$r \in \M_i \land r \prec s$.
If $r \preceq \k_{i,i}(s) \subseteq \Acc_{i}$,
it follows $r \in \Acc_i$. Otherwise
$r \prec ' s$. We show $r \in \Acc_i'$.
If $j<i$, $\k_{i,j}(r) \subseteq \Acc_j$.
$\k_{i,i}(r) \prec r \prec s$,  
if $j<i$, $\k_{i,j}[\k_{i,i}(r)] \subseteq 
\k_{i,j}(r) \subseteq \Acc_j$, therefore
$\k_{i,i}(r) \subseteq \M_i$ and
by side-side-IH $\k_{i,i}(r) \subseteq \Acc_i$.
For $i<j<n$ we show by side$^3$-induction on $j$
$\k_{i,j}(r) \subseteq \Acc_j $.
If $l < j$, $\k_{j,l}[\k_{i,j}(r)] \subseteq 
\k_{i,l}(r)\subseteq \Acc_l$ by IH, $\k_{i,j}(r) \subseteq \M_j$,
by main-IH $\k_{i,j}(r) \subseteq \Acc_j$.
Therefore $r \in \Acc_i'$, $r \prec' s$, by side-IH
$r \in \Acc_i$, and the side-side-induction and therefore
as well the side-induction are complete.
Now follows by induction on $\length(s)$ 
$\all s  \in \M_i. s \in \Acc_i$,
$\Acc_i = \M_i$ and the main induction is finished.
Now it follows $\T_0=\M_0=\Acc_0$ and we are done.\par 

\subsection{$n$-Ordinal Function Generators}
As before we are going to define the ordinal systems which exhaust the
strength of elementary 
$n$-ordinal systems. As there, we want as a side result as well 
to get functions defined on arbitrary ordinals.
Again of course this detour via ordinals is not necessary, one 
could define easily the $n$-ordinal systems  purely syntactically.

We need to represent the higher ordinal systems we used as ordinals and
will do it in the usual way by taking ordinals of higher number classes:
Let $\Omega_0:= 0$,
$\Omega_n:= \aleph_n$. Ordinals in $[\Omega_n,\Omega_{n+1}[$ can be
regarded as objects referring to ordinals in $[0,\Omega_n[$ and therefore
be considered as representatives of the higher OS. In a refined approach 
one could replace $\Omega_n$ by the $n$-th admissible ordinal $\Omega_n^\rec$
(starting with $\Omega_1^\rec:= \omega_1^\ck$).\par  
In the definition of (non-iterated) OFG we closed the sets $\C(a)$ under
$(\k(b) \subseteq \C(a) \land b <' a )\ar \eval(b) \in \C(a)$ with no
restriction on $b$ being in normal form. But one could easily show that
having the restriction of $b$ being in normal form yields the same result,
since, if $b$ is not in normal form, $\eval(b) \in \k(b)$. In the case 
of $n$-OFG, normal form will again mean that $\k(b) \subseteq \C_i(b)$.
However, this condition can be violated by having $\eval_j(c) \in \k(b)$
such that $j>i$ and $\k(c) \cap [\Omega_i,\Omega_{i+1}[> \eval(a)$.
So we do no longer have $a \notin \NF  \ar \eval(a) \in \k(a)$ and the
above argument does no longer go through. We will use the 
easiest way of dealing with these problems: we add only
 $\eval(c)$ to $\C(a)$, if $\NF(c)$ holds.
%
%
%%%%%%%%%%%%%%%%%%%%%%%%%%%%%%%%%%%%%%%%%%%
%%
%                                         %
%%%%%%%%%%%%%%%%%%%%%%%%%%%%%%%%%%%%%%%%%%%
%
%
%
\begin{definition}
\laprint{defnOFG}{\rm
\begin{enumerate}
\ssubitem a An {\em $n$-ordinal function generator}, in short
{\em $n$-OFG} is a quadruple
$\calO:= (\Arg_i,\k_i, \lrm_i,<_i')_{i<n}$ such that 
$\Arg_i$ are classes (in set theory),
$\k_i,\lrm_i: \Arg_i \ar_\omega \Ord$,
$\all a \in \Arg_i.\lrm_i(a) \subseteq \k_i(a)$,
and  $<_i'$ is a well-ordered relation on $\Arg_i$
($i<n$).  We assume in the following that 
$\Arg_i$ are disjoint and omit therefore the index
$i$ in $\k$, $\lrm$, $<'$. Let in the following
$\calO$ be  as above.
\ssubitem b $\Omega_0:= 
%%%%\Omega_0^\rec 
0$, $\Omega_{n+1}:= \aleph_{n+1}$,
%%%$\Omega_{n+1}$ is the $n+1$th admissible ordinal, starting
%%%with $\Omega_1:= \omega_1^\ck$.
\ssubitem c An $n$-OFG is {\em cardinality based},
if for  all $B \subseteq \Ord$ countable $\k_i^{-1}(B)$ is  countable 
and for $a \in \Arg_i$,
$\lrm_i(a) \subseteq \k_i(a) \cap [\Omega_i,\Omega_{i+1}[$. 
In this case we define $\k_{i,j}(a):= \k_{i}(a) \cap [\Omega_j,\Omega_{j+1}[$.
%%%It is {\em recursively based}, if there exists an in $\kpomega$
%%%definable $\Sigma$-function symbol $F$ such that
%%%$F(a) = \{ b \in \Arg \mid \k(a)  \subseteq b \} $,
%%%$\k$, $\lrm$ are $\Sigma$-definable functions,
%%%$\prec$ is $\Delta$-definable,
%%%$\lrm(a) \subseteq \k_i(a) \cap [\Omega_i^\rec,\Omega_{i+1}^\rec[$ 
%%%and the properties in
%%%\refact a can be shown in $\kpomega$.\par 
\ssubitem d If $\calO$ as above is a cardinality based  $n$-OFG, we define
by main recursion on $n-i$  ($i=1 ,\ldots, n$) by side-recursion on 
$<'$ ordinals for $a \in \Arg_i$ 
$\eval_i(a) \in \Ord$ and subsets
$\C_i(a) \subseteq \Ord$ by:
$\C_i(a):= \bigcup_{l=0}^\omega \C_i^l(a) $, where 
$\C_i^0(a):= [0,\Omega_i[\cup (\bigcup (\k_i(a) \cap \Omega_{i+1}))
\cup \lrm(a)$,
$\C_i^{l+1}(a):= \C_i^l(a) \cup$\\
\hbox{\qquad \qquad \quad}$ \bigcup_{i \leq j < n}
\{\eval_j(b) \mid 
(j=i \ar b <' a), b \in \Arg_j, \NF_j(b), \k(b) \subseteq \C_i^l(a) \} $
and $\NF(b) :\Iff \NF_j(b) :\Iff \eval_j(b) \in \C_j(b)$.\\
$\eval_i(a):= \min \{ \alpha  \mid \alpha \not \in \C_i(a) \} $,
$\eval(a):= \eval_i(a)$ if $a \in \Arg_i$, and\\
$\NF_i:= \{ a \in \Arg_i \mid \NF(a) \}$.
%%%\ssubitem e If $\calO$ is as above but recursivity based, then
%%%we define $\eval^\rec_i(a)$, $\C_i^\rec(a)$, $\C_i^{n,\rec}$ as 
%%%$\eval_i(a)$, $\C_i(a)$, $\C_i^n(a)$,
%%%but referring to $\Omega_i^\rec$ instead
%%%of $\Omega_i$.
\ssubitem f 
%%$\Cl_{i,j}'\subseteq \Arg_i$, 
For cardinality based $n$-OFG $\calO$ we define
$\Cl_{i}\subseteq \NF_i$ simultaneously for 
all $i< n$ inductively defined by:
if $b \in \NF_i$, $\k(b) \subseteq \bigcup_{i<n}\eval_i[\Cl_i]$, 
then $b \in \Cl_i$. $\Cl:= \bigcup_{i<n}\Cl_i$,
$\Arg[\Cl]_i:= \{ a \in \Arg_i \mid\k(a) \subseteq \eval[\Cl]\} $,
$\Arg[\Cl]:= \bigcup_{i<n}\Arg[\Cl]_i$. Note
$\Cl_i \subseteq \Arg[\Cl]_i$.\\
Assuming, that $\eval_i \restriction \NF_i$ is
injective, which will be shown later, we define
$\length: \Arg[\Cl] \ar \nat$,
$\k_{i,j}^0,\k_{i,j}': \Arg[\Cl]_i \ar_\omega \Cl_{j}$ 
simultaneously inductively
by:\\
$\k_{i,j}^0(b):= \eval_j^{-1}[\k_{i,j}(b)] \cap \Cl_j$.
$\k_{i,j}'(b):= \k^0_{i,j}(b) \cup
\bigcup_{j< l<n}\k'_{l,j}[\k^0_{i,l}(b)]$.\\
$\length(b):= \max(\length[\bigcup\k'_{i,i}(b)] \cup \{ -1 \})+1$.
\ssubitem g Define for $a,b \in \Arg_i$
$a \prec b :\Iff a \prec_i b :\Iff \eval_i (a) < \eval_i(b)$.
\end{enumerate}}
\end{definition}
%
%
%

%
%
%
%%%%%%%%%%%%%%%%%%%%%%%%%%%%%%%%%%%%%%%%%%%
%
%                                         %
%%%%%%%%%%%%%%%%%%%%%%%%%%%%%%%%%%%%%%%%%%%
%
%
%
\begin{lemma}
\laprint{lemnOFG1}
Let $\calO$ be a cardinality based $n$-OFG.
\begin{enumerate} 
\ssubitem a $\C_i(a)$ is the least set $M$ such that 
$\C_i^0(a) \subseteq M$ and such that, if 
$b \in \Arg_j$, $j = i \land b <' a $ or $j>i$, and, if $\NF(b)$ 
and $\k(b) \subseteq M$, then $\eval_j(b) \in \M$.
\ssubitem b $\C_i(a) \cap \Omega_{i+1} \segment \Ord$.\\
Therefore especially $\eval_i(a) = \C_i(a) \cap \Omega_{i+1}$ 
\ssubitem c If $a \in \Arg_i$,
then $\eval_i(a) \in [\Omega_i,\Omega_{i+1}[$.
\ssubitem{ca} If $a,b \in \NF_i$, then
$\eval(a) < \eval(b) \Iff (a <' b \land \k_{i,i}(a) < \eval(b)) \lor 
\eval(a) \leq \k_{i,i}(b)$.\\
Especially $\eval \restriction \NF_i$ is injective
and therefore $\length$ and $\k'_{i,j}$ are well-defined.
\ssubitem{cb} If $a \in \Arg[\Cl]_i$, $l<j$, then
$\k_{l,j}'[\k_{i,l}'(a)] \subseteq \k_{i,j}'(a)$.
\ssubitem{cd} If $a \in \Arg_i$, $b \in \Cl_j \cap \C_i(a)$,
$i,j,l<n$, $i \leq j$,
then $\eval[\k'_{j,l}(b)] \subseteq \C_i(a)$.\\
If $a \in \Cl_i$, then $\eval[\k'_{i,j}(a)] \subseteq \C_i(a)$.
%\ssubitem d For $a,b \in \Arg[\Cl]_i$ we have
%\begin{eqnarray*} 
%\eval_i(a) < \eval_i(b) &\Iff& (a <' b \land \k'_{i,i}(a) < 
%\eval_i(b))
%\lor \eval_i(a) < \k'_{i,i}(b) \lor \eval_i(a) \leq \lrm(b)\\
%\eval_i(a) = \eval_i(b) &\Iff& (a <' b \land \eval_i(b) = \max( \k'_{i,i}(a)) 
%\land \eval_i(b) \not \in \lrm(a)) \lor \\
%&& (b <' a \land \eval_i(a) = \max(\k'_{i,i}(b)) \land \eval_i(a) \not \in 
%\lrm(b)) \lor \\
%&&a=b
%\end{eqnarray*} 
%\ssubitem e For $a,b \in \Cl_i$, 
%$\eval_i(a) < \eval_i(b) \iff (a <' b \land \k'_{i,i}(a)< 
%\eval_i(b)) 
%\lor \eval_i(a) \leq \k'_{i,i}(b) $.
%Especially follows that $\eval_i \restriction \{ a \in \Arg.\NF(a) \}$ is
%injective and therefore
%$\length:\Arg[\Cl] \ar \omega$.
\ssubitem f $\calF:= ((\Cl_i,\prec_i)_{i<n},(\k'_{i,j})_{i,j<n},
(\length_i)_{i<n})$ is an $n$-ordinal system.
We will call any $n$-OS structure isomorphic to $\calF$ 
an {\em $n$-OS based on $\calO$}.
%%%\ssubitem d \refact a - \refact c hold for recursivity based 
%%%$n$-OFG mutatis mutandis.
\end{enumerate} 
\end{lemma}
%
%
%
{\bf Proof:} \refact a is clear, since $\k(a)$ is finite.\\ 
\refact b: Main induction on $n-i$, Side Induction on $a \in \Arg_i$:
Assume assertion for $b <' a$. We show 
$\all \alpha  \in \C_i^n(a)\cap \Omega_{i+1}.
\alpha \subseteq \C_i(a)$ by side-side-induction
on $n$: $n=0$: clear, since $\lrm_i(a) \subseteq \k_i(a)$.
$n \ar n+1$: If $\alpha  \in (\C_i^{n+1}(a) \setminus \C_i^n(a))\cap 
\Omega_{i+1}$, then $\alpha = \eval_i(b)$ with
$\k(b) \subseteq \C_i^n(a)$,
by side-side-IH and, since $\lrm(b) \subseteq \k(b)$, 
$\bigcup (\k(b) \cap \Omega_{n+1} ) \cup \lrm(b) \subseteq \C_i(a)$,
and by an immediate induction $\C_i^l(b) \subseteq \C_i(a)$ for
all $l \in \omega$, 
$\C_i(b) \subseteq \C_i(a)$, therefore
$\alpha = \min \{ \gamma \mid \gamma \not \in \C_i(b) \}
\subseteq \C_i(a)$.\\ 
\refact c By an induction using the condition ``cardinality based
OFG'' it follows that the cardinality of $\C_i^n(a)$ and therefore 
as well that of $\C_i(a)$ is $< \Omega_{i+1}$. We conclude that 
there exists an $\alpha  < \Omega_{i+1}$ such that
$\alpha \not \in \C_i(a)$.\\
%\refact e: \refact a - \refact b are as before.  \refact c
%Induction on $a$: Main induction on $n-i$, side-induction
%on $a$: We see easily 
%that $\C_i^n(a)$ can be defined as a $\Sigma$-function symbol
%in  $i$, $a$ and $n$, therefore as well $\C_i(a)$.\\
\refact{ca} ``$\Leftarrow$'' If $a<'b$, $\k_{i,i}(a) < \eval(b)$,
$\NF(a)$,
it follows easily $\k(a) \subseteq \C(a) \subseteq \C(b)$,
$\eval_i(a) \in \C(b) \cap \Omega_{i+1} = \eval_i(b)$.
If $\eval(a)  \leq \k_{i,i}(b)$, 
follows by $\k_{i,i}(b) < \eval(b)$ the assertion.
``$\Rightarrow$'' If the right side is false, the right side holds
for $\eval_i(b)< \eval_i(a)$ or $a=b$, so the left side is false.\\
\refact{cb}: easy induction on $\length(a)$.\\
\refact{cd}: First assertion by induction on $\length(a)$,
side-induction on $\length(b)$.
Second assertion: by $\k(a) \subseteq \C_i(a)$.\\
%\refact d: As in Lemma \refcom{lemordgen}{eb}, using \refact{cb}.\\ 
%\refact e ``$\Leftarrow$'': $\eval_i(a) \in \C_i(b)$.
%``$\Ar$'': If the right condition does not hold, then
%$a = b$ or the right condition holds with $a$, $b$ interchanged,
%therefore $\eval_i(b) \leq \eval_i(a)$.\\
\refact f: by the above (in order to show $\k_{i,i}'(a)< \eval(a)$ 
use $\k_{i,i}'(a) \subseteq \C_i(a)$).\QED
%
%
%
%%%%%%%%%%%%%%%%%%%%%%%%%%%%%%%%%%%%%%%%%%%
%%
%                                         %
%%%%%%%%%%%%%%%%%%%%%%%%%%%%%%%%%%%%%%%%%%%
%
%
%
%\begin{lemdef}
%\laprint{defprimreknsys}
%\begin{enumerate} 
%\ssubitem a 
%$\calO$ be an $n$-OFG as usual. Define $\eval_i^\rec$,
%$\C_i^\rec$ as before, but with 
%$\Omega_i$ replaced by $\Omega_i^\rec$.
%Assume for all $i \geq 1$ the following
%For $i \leq j,k<n$ assume $\T_{i,j} \subseteq \T_{i,j}' \subseteq 
%\T''_{i,j}$, $\T_{i,i-1}:= \T_{i,i-1}:= \T''_{i,i-1}:+
%[0,\Omega_i^\rec[$,
%$f_{i,j}: \T_{i,j}' \ar \Arg_j$ ($j \geq i$),
%$\k'_{i,j,k}: \T''_{i,j} \ar_\omega \T''_{i,k}$ ($j \geq i$, $k \geq i-1$) 
%$\lrm'_{i,j}: \T'_{i,j} \ar_\omega \T'_{i,j}$ ($j \geq i$),
%$\length: \T'_{i,j} \ar \nat$ ($j \geq i$).
%Assume that for $a \subseteq [0,\Omega_i^\rec[$ $\T''_{i,j}[a]:=
%\{ a \in \T''_{i,j} \mid \k_{i,j,-1} \subseteq a \}$
%is a $\Sigma$-function in $a$ and that $f_{i,j}$, $\k'_{i,j,k}$,
%$\lrm'_{i,j}$, $\length$ are $\Sigma$-functions.
%Let for $a,b \in \T'_{i,j}$,
%$a \prec' b \Iff f(a) <' f(b)$, 
%$a \prec  b \Iff \eval_j(f(a)) < \eval_j(f(b))$,
%$a \approx b \Iff \eval_j(f(a))= \eval_j(f(b))$.
%Assume the following
%\begin{itemize} 
%\item $a \in \T_{i,j}' \ar \k_{i,j,k}(a) \subseteq \T_{i,k}$ ($k \geq i$).
%\item $f[\k_{i,j,k}(a)] = \k'_{i,j,k}(f(a))$, $f[\lrm '(a)] =  \lrm(f(a))$,
%$\length(a)> \length[\k(a)]$.
%\item For every subsets $A_{i,j}$ of $\T_{i,j}$ ($j \geq i-1$)
%$f[\{ t \in \T '_{i,j} \mid \all k \geq i-1.\k'_{i,j,k}(t) \subseteq A_{i,k}\}] 
%= 
%\{ a \in \Arg  \mid \k(a) \subseteq A_{i,i-1} \cup \bigcup_{j \geq i}
%\eval_j[f_{i,j}[A_{i,j}]] \} $.
%\item If $A_{i,j} \subseteq \T'_{i,j}$, $f_{i,j} \restriction A_{i,j}$ is injective ($j \geq i$),
%then $f_{i,j} \restriction \{ t \in \T '_{i,j} \mid \all k \geq i.
%\k '_{i,j,k}(t) \subseteq A_{i,k} \} $
%is injective.
%\item For $a,b \in \T$ we can determine by a $\Delta$-formula 
%from $\prec$ and $\approx$  restricted to $\k(a) \cup \k(b)$ 
%(coded as a finite list)
%whether $a,b \in \T '$ and, if this is the
%case, whether $a \prec ' b$.
%\item $a \in \T_{i,j}  \Iff a \in \T '_{i,j}\land f(a) \in \NF \land 
%\all k \geq i.\k_{i,j,k}(a) \subseteq \T_{i,k}$.
%\end{itemize} 
%Then $\T_{i,j}[a]$, $\T'_{i,j}[a]$ are $\Sigma$-definable
%in $a$,  $\prec'$, $\prec$, $\approx$ are $\Delta$-definable,
%$(\T,\prec,\prec ',\k',\length)$ is a OS based on $\calO$ 
%and $\eval_i(a) \in [\Omega_i,\Omega_{i+1}[$ for $a \in \Arg_i$.
%In this case $\calO$ is called a {\em recursivity based
%ordinal system}
%\ssubitem b If in case $i=0$ the above holds for
%a cardinality based or recursivity based ordinal system
%with $\Delta$-definable,
%$\Sigma$-definable replaced by primitive recursive (especially
%$\T''_{0,j}$ is a primitive recursive subset of $\nat$) and,
%in case of cardinality based, with $\Omega_i^\rec$ replaced by
%$\Omega_i$, then the conclusion holds as well with $\Sigma$-definable,
%$\Delta$-definable replaced by primitive recursive.
%\end{enumerate} 
%\end{lemdef}
%%
%%
%%
%{\bf Proof:} The results on $\Sigma$-/$\Delta$-definability
%and on primitive recursiveness
%follow exactly as in Lemma \ref{defprimresksys}. In order to
%show that $\eval_i(a) < \Omega_{i+1}^\rec$, which will be done
%by main induction on $n-i$, side-induction on $a \in \Arg_i$,
%we show that there exists sets $\C_{i,j}'(a) \subseteq 
%\T_{i+1,j}$ ($j \geq i$) which are 
%$\Delta_0$ in $[0,\Omega_i^\rec[$ such that
%$\eval_j[f_{i+1,j}[\C_{i,j}'(a)]] = \C_i(f(a)) \cap [\Omega_j^\rec,\Omega_{j+1}^\rec[$,
%$j \geq i+1$ and
%$\C_{i,i}'(a) = \C_i(f(a)) \cap [0,\Omega_{i+1}^\rec[$. 
%$\C_{i,j}'(a)$ will be the least collection of sets such that
%$[0,\Omega_i^\rec[ \cup \bigcup (\k'_(a) \cap \Omega_{i+1}^\rec)
%\cup \lrm(a) \subseteq \C_{i,i}'(a)$,
%that if ???
%
%
%
%
%
%%%%%%%%%%%%%%%%%%%%%%%%%%%%%%%%%%%%%%%%%%%
%%
%                                         %
%%%%%%%%%%%%%%%%%%%%%%%%%%%%%%%%%%%%%%%%%%%%
%%
%%
%%
%\begin{definition}
%\laprint{defnOFG}{\rm
%Let $\calO$ be  a fixed $n$-OFG 
%\begin{enumerate}
%\ssubitem a Define $\T_i \subseteq [\Omega_i,\Omega_{i+1} [$ simultaneousl
%for all $i$ by:\\
%If $a \in \Arg_i$, $\NF(a)$,
%$\k(a) \subseteq \bigcup_{i<n}\eval_i[\T_i]$, then
%$a \in \T_i$.\\
%$\k'_{i,j}(a):= \eval_j^{-1}[\k(a) \cap [\Omega_j,\Omega_{j+1}[$.\\
%$\k_{i,j}(a):= \k'_{i,j}(a) \cup \bigcup_{j' < j}\k_{j,j'}[\k'_{i,j}(a)]$.\\
%$\length(a):= \max \length[ \bigcup_{j<n}\k_{i,j}(a)]+1$ for 
%$a \in \T_i$.
%For $a, b \in \T_i$ $a \prec b \Iff \eval_i(a) < \eval_i(b)$.\\
%Then $((\T_i)_i<n, (\k_{i,j})_{i,j<n},\prec ,<', \length)$
%or any isomorphic structure (isomorphic defined in a straightforward
%way) called the $n$-OS based on $\calO$. 
%\end{enumerate}}
%\end{definition}
%
%
%
%
%
%
%%%%%%%%%%%%%%%%%%%%%%%%%%%%%%%%%%%%%%%%%%%
%%
%                                         %
%%%%%%%%%%%%%%%%%%%%%%%%%%%%%%%%%%%%%%%%%%%
%
%
%
%\begin{lemma}
%\laprint{lemOFGareOS}
%If $\calO$ is an $n$-OFG, the $n$-OS based on it is actually an
%$n$-OS.
%\end{lemma}
%%
%%
%%
%{\bf Proof:}\\
%
%
%
%%%%%%%%%%%%%%%%%%%%%%%%%%%%%%%%%%%%%%%%%%%
%%
%                                         %
%%%%%%%%%%%%%%%%%%%%%%%%%%%%%%%%%%%%%%%%%%%
%
%
%
\begin{example}
\laprint{exmplnofg}{\rm
\begin{enumerate} 
\ssubitem a
Let \\
$\Arg'_0:= \CNF'_0([0,\Omega_1[,]0,\omega[)$,\\ 
$\Arg'_{i+1}:= \CNF'_{i+1}([0,\Omega_{i+2}[,]0,\Omega_{i+1}[)
\setminus \CNF'_{i+1}(\{ 0 \},]0,\Omega_{i+1}[)$,\\
$\k'_0$, $\lrm'_0$ as $\k$, $\lrm$ in Example \refcom{exmplordfun}d
(with $\CNF'$ replaced by $\CNF_0'$),\\ 
$\k'_{i+1}(\CNF'_{i+1}(\alpha_1, \beta_1 ,\ldots, \alpha_m,\beta_m)):=
\{ \alpha_1 ,\ldots, \alpha_m,\beta_1 ,\ldots, \beta_m \}$,\\
$\lrm'_{i+1} ( \CNF'_{i+1}(\alpha_1,\beta_1 ,\ldots, \alpha_m,\beta_m)):=
\mycases{\{ \alpha_1\} &
\text{if $m>1$ or $\beta_m> 1$,}\\ \emptyset&\text{otherwise.}}$\\
$(|\Arg'_i|,\k'_{i},\lrm'_{i},<_{\Arg'_i})$ is a cardinality based $n$-OFG
such that\\ 
$\eval_0(\CNF'_0(\alpha_1 ,n_1 \ldots, \alpha_m ,n_m))
= \omega^{\alpha_1}n_1 +\cdots+ \omega^{\alpha_m}n_m$,\\
$\eval_{i+1}(\CNF'_{i+1}(\alpha_1 ,\beta_1 \ldots, \alpha_m ,\beta_m))
= \Omega_{i+1}^{\alpha_1} \beta_1 +\cdots+ \Omega_{i+1}^{\alpha_m}\beta_m$.
\ssubitem b 
Let $\calS_l$, $\k^l$  be defined as in Example \refcom{exmplordfun}i,
but with the restriction to ordinals $< \Omega_{n+1}$ always.
Let $\Arg_i'$, $\k_i'$, $\lrm_i'$ be as  before,
$\calF_i:= \Arg_i' \preccirc \psibar_i(\calS_l)$,
$\k_i \restriction \Arg_i':= \k_i'$, 
$\lrm_i\restriction \Arg_i':= \lrm_i'$,
$\k_i(\psibar_i(a)):= \k^l(a)$,
$\lrm_i(\psibar_i(a)):= \k^l(a) \cap [\Omega_i,\Omega_{i+1}[$
for $a \in |\calS_l|$.
The resulting cardinality based $n$-OFG 
can be seen as a generalization of the
extended Sch{\"u}tte Klammer symbols.
\ssubitem c Let $f^n_l: \calS_l \ar \Ord$,
$f^n_0(\alpha):= \alpha$,
$f^n_{l+1} 
{\beta_1 \cdots \beta_m \choose A_1 \cdots A_m}:=
%\left(\begin{array}{ccc}\beta_1& \cdots&\beta_m\\
%A_1  &\cdots &A_m\end{array} \right):=  
\Omega_{n}^{f^n_l(A_1)}\beta_1 +\cdots+ \Omega_{n}^{f^n_l(A_m)} \beta_m$.
%then one can easily see that in $\calO_i$ 
%$\eval_{j}(\psi_j(A))= \psi_j(f_i(A))$, where $\psi_j$ is the
%generalization of the $\psi$-function in \cite{rathjenweierman}
%defined by:
%$\C'_i(\delta,\rho):= $ closure of $[0,\Omega_i] \cup \rho$ under 
%$+$, $\lambda \xi \omega^\xi$,
%$\psi_j '$ for $j>i$, $\psi_i$ restricted to ordinals
%$\xi < \delta$, $\psi_i(\delta):= 
%\min \{ \rho \mid \delta \in \C_i(\delta,\rho) \land 
%\C_i(\delta,\rho) \cap \Omega_{i+1} \subseteq \rho \} $.
If we restrict now $\psibar_i(\calS^l)$ to 
$\psibar_i(A)$ such that $f^n_l(A) \in \C_{\Omega_{i+1}}'(f^n_l(A))$,
where $\C_{\Omega_{i+1}}'(a)$ is the $\C$-set  for the 
function $\psi_{\Omega_{i+1}}$ as in \cite{buchholzasimplified}, but
with $0$, $+$, $\omega^\cdot$ as basic function, 
we get $\eval_i(\psibar_i(A)) = \psi_{\Omega_{i+1}}
(f^n_l(A))$. 
Again, with this
last modification we get that in the resulting ordinal notation system
$\prec_i$ and $\prec_i'$ coincide for terms of the 
form $\psibar_i(A)$.
\end{enumerate} 
}
\end{example}
%
%
%
%
%
%%%%%%%%%%%%%%%%%%%%%%%%%%%%%%%%%%%%%%%%%%%
%%
%                                         %
%%%%%%%%%%%%%%%%%%%%%%%%%%%%%%%%%%%%%%%%%%%
%
%
%
\begin{remark}
\laprint{remprimreksysn} 
\begin{enumerate} 
\ssubitem a The straightforward generalization of Lemma
\ref{defprimreksys} holds as well for cardinality based $n$-OFG with the
exception  that we can conclude that $a \prec b$  is 
primitive recursive only for $a \in \T$, $b \in \T'$.
Further under the assumptions of this generalized lemma
the function corresponding to $\k'_{i,j}$, which takes the place of 
$\k_{i,j}$ in the resulting $n$-OS, 
is primitive recursive.
\ssubitem b In order to replace $\Omega_i$ by $\Omega_i^\rec$
(the $i$ th admissible, starting with $\Omega_0=0$, $\Omega_1 = \omega_1^\ck$)
we need some additional conditions, which essentially express that we
have terms as in \refact a for the higher number classes and
arguments, but based on ordinals in $[0,\Omega_i^\rec[$. They can
be developed in a similar way as the conditions in \refact a,
but because of lack of space, we omit them here.
\end{enumerate} 
\end{remark}
%
%
%
{\bf Proof of \refact a:} As in Lemma \ref{defprimreksys}. Only the 
primitive recursive determination of $r \prec s$ follows as follows:
Note that $\krmhat_{i,j}$ corresponds to $\k_{i,j}^0$.
Let for $s \in \T'_i$
$\C^j(s):= \{ r \in \T_j \mid \eval(f(r)) \in \C_i(f(s)) \}$.
We determine for $s \in \T'_i$, $r \in \T_j$ 
such that $\length(r), \length(s) \leq \length(t)$ by
recursion on $\length(r)$ primitive recursively whether
$r \in \C^j(s)$:
If $j< i$, $r \in \C^j(s)$.
If $j = i$,
$r \in \C^j(s) \Iff 
r\preceq \krmhat_{i,i}(s) 
\lor (r \preceq' s \land \all l<n(\krmhat_{i,l}(r) \subseteq \C^l(s)))$
and, if 
$i < j$,  $r \in \C^j(s) \Iff 
\all l<n.\krmhat_{i,j}(r) \subseteq \C^l(s)$.
Then in case $i=j$ follows $r \prec s \Iff r \in \C^i(s)$ and  
$t \in \T_i' \Iff 
\NF(f(t)) 
\Iff \all j<n.\krmhat_{i,j}(t)\subseteq \C^j(t)$.\QED
%
%
%
%
%
\begin{lemma}
\laprint{lemexampleosn}
\begin{enumerate} 
\ssubitem a
For the OFG in the Examples \ref{exmplnofg},
there exists an elementary OS represented by them.
\ssubitem b
The supremum of the strength of $n$-OS is $|\ID_n|$.
\end{enumerate} 
\end{lemma}
%
%
{\bf Proof:} \refact a as for Lemma \ref{theoexampleos},
\refact b as for Theorem \ref{theostrengthordsyslow}.\QED


%\refact b 
%As in the proof of Theorem \refcom{theoremstrengthelemos}c we 
%start with the system in \cite{buchholzschuettebook} (or 
%a modified version of \cite{buchholzanewsys}), and omit
%now $\psi_\alpha$ and $\Omega_\alpha$ for $n \preceq \alpha$.\\
%Let $m \in \omega$ be fixed, 
%$\OT$ be the set of ordinal notation systems built
%from ordinalnotations $\prec \omega_m(\Omega_n+1)$ only,
%and let $\T_0:= \OT \cap \Omega_1$,
%$\T_{i+1}:= \OT \cap [\Omega_{i+1},\Omega_{i+2}[$,
%$\T_{\leq i}:= \bigcup_{j \leq i}\T_j$.
%Let the $\prec'$ on $\T_i$ correspond to the
%ordering on
%$\{ \Omega_i \} \preccirc 
%((\omega^{\T_{\leq i}}\cdot \T_{\leq i})\restriction \T_i)
%\preccirc (\psi_i(\M) \restriction \T_i)$,
%where $M:= \Omega_{n} \uparrow(m)\T_{\leq n}$
%We define $\k: \T_{\leq n} \ar_\omega \T_{\leq n}$ by
%induction on the length of the terms.
%$\k(0):= \k(\Omega_k):= \emptyset$,
%$\k(\omega^a+b):= \{ a,b \} $, $\k(\psi(a)):= 
%\krmhat(a)$, $\krmhat$ defined according to defintion
%\refcom{defOmegapower}b.
%Now define $\k'_{i,j}: \T_i \ar_\omega \T_j$, 
%$\k'_{i,j}(t) := \k(t) \cap \T_{j}$.
%The condition about $\Cl_i$
%in Lemma \refcom{lemgeneratenos}b  are fulfilled for 
%$(\calGvec,(\k_{i,j}'))$,
%therefore we get a $n$-OS-struc
%$(\calGvec,\k_{i,j})$
%which fulfills ($n$-OS 2) and such
%that $\calF_i$ fulfill (OS 3).\\ 
%$\calF_i$ fulfill (OS 1): For $a = 0,\Omega_1$,
%$\omega^b+c$ in normal form follows immediately $\k_{i,i}(a) \prec a$.
%If $a= \psi_i(c)$ in normal form follows
%$c \in \C_i(c)$ and by induction on $\C_i(x)$ immediately
%$\all x.
%\all j \geq i.\all y \in \C_i(x) \cap \T_j.\k_{i,j}(y) \subseteq \C_i(x)$,
%$\k_{i,i}(c) \subseteq \C_i(c)\cap \Omega_i $, $\k_{i,i}(c) \prec \psi_i(c)$.\\ 
%Condition (OS 2) and the well-ordering condition follow exactly 
%as in the proof of Theorem 
%\refcom{theoremstrengthelemos}c.\\ 
%Now we have defined a system with ordertype $\psi_0(\omega_m(\Omega_n+1))$
%and the assertion follows.\QED \\
%\medskip 
%In order to make the definition of $n$-OS easier, we have the following lemma.
%%%%%%%%%%%%%%%%%%%%%%%%%%%%%%%%%%%%%%%%%%%%
%%%
%%                                         %
%%%%%%%%%%%%%%%%%%%%%%%%%%%%%%%%%%%%%%%%%%%%
%%
%%
%%
%\begin{lemma}
%\laprint{lemgeneratenos}
%Let $(\calGvec,\krmvec)$ be an $n$-OS.
%\begin{enumerate}
%\ssubitem a 
%Assume ($n$-OS 2), and for $\calF_i$ (OS 3) holds.
%Then $\Cl_i = \T_i$ for all $i < n$.
%\ssubitem b Let $\Cl_i$  be defined as in \refact a
%(but without having the conditions from \refact a)
%and assume $\Cl_i= \T_i$ for all $i$
%Define by induction on $t \in \Cl_i$ simultaneously
%for all $i,k<n$ and by side-induction on $n-k$ 
%$\k_{i,k}'(t):= 
%\k_{i,k}(t) \cup \bigcup_{k<j<n}\k_{j,k}[\k_{i,j}(t)]$.
%$\krmvec':= (\k'_{i,j})_{i,j < n}$.
%Then $(\calGvec,\krmvec ')$,
%is a $n$-OS-structure which fulfills
%($n$-OS 2) and such that with
%with $\calF_i':= (\calG_i,\k_{i,i}')$,
%$\T_i = \Cl_{\calF_i'}$.\\ 
%\end{enumerate}
%\end{lemma}
%%
%%
%%
%{\bf Proof:}\par
%\refact a: We show by induction
%on $n-i$ $\Cl_{\calF_i}= \T_i$ and
%assume $i$ according to induction. We show by side-indduction
%on $a \in \Cl_{\calF_i}$,
%$\all a \in \Cl_{\calF_{i}}.
%(\all j<i.\k_{i,j}(a) \subseteq \Cl_j) \ar a \in \Cl_i$
%Assume $a$ according to induction, $\k_{i,j}(a) \subseteq \Cl_j$
%for $j<i$. If $j>i$, $\k_{i,j}(a) \subseteq \Cl_j$ by main IH.
%If $j < i$, then 
%$\k_{i,j}[\k_{i,i}(a)] \subseteq \k_{i,j}(a) \subseteq \Cl_j$.
%By side-IH therefore $\k_{i,i}(a) \subseteq \Cl_i$.
%Sinc $\k_{i,j}(a) \subseteq \Cl_j$ for $j<i$ follows
%$a \in \Cl_i$.\par 
%\refact b: The conditions ($n$-OS 2) are trivially fulfilled.
%Let $\Cl_i'$, be defined as
%$\Cl_i$, but referring to the new formed structure.
%We show simultaneously by induction on
%$x \in \Cl_i$ 
%$\all x \in \Cl_i.x \in \Cl_i'$.
%Assume $x \in \Cl_i$ according to induction.
%By IH $\k_{i,j}(x) \subseteq \Cl_j'$.
%Therefore $\k_{k,j}'[\k_{i,k}(x)] \subseteq \Cl_j'$ 
%for all $j<k$,
%$\k_{i,j}'(x) \subseteq \Cl_j'$, $x \in \Cl_i$.\QED  

%{\bf Equivalences of different ordinal systems} follows as in the last
%section. Additionally we here that, if we define
%$\C_k(\alpha)$ as closed under the full function $\lambda \beta.\psi_n(\beta)$
%for $n>k$ (without the restriction to $\beta< \alpha$) and one
%of the usual choices of initial functions, then for the resulting
%ordinal notation system $\OT'$ we have, that 
%$\psi_0\epsilon_{\Omega_n+1}$ denotes the same
%orderinal as in the system $\OT$ where we have the
%restriction. This follows, since
%obviously $\OT \subseteq \OT'$ and, if $r,s \in \OT$,
%$r \prec s$ in $\OT$ iff $r \prec s$ in $\OT '$, so 
%the ordertype of $\OT '$ is stronger. But the approximations
%of both systems  can be seens as part of an elementary $n$-OS
%from below, therefore 
%$\psi_0\epsilon_{\Omega_n+1}$ denotes in both systems an ordinal
%$< |\ID_n|$.

\section{Transfinitely Iterated Ordinal Systems}
\laprint{sectsigmaiterate}

\subsection{Definition of $\sigma$-OS}

The na{\"{\i}}ve way of extending the approach used  in the last section to
the transfinite does not work. In the proof of well-ordering
of $n$-OS we always reduced well-ordering of an
$n$-OS $\calF_n$ to $\calF_n[A]$ for well-ordered sets
$A \subseteq \T_0$. If we try this with $n$ replaced by $\omega$, we 
will reduce 
the well-ordering of an $\omega$-OS to the  well-ordering of
an $\omega$-OS, which does not work. 
In the proof using the
accessibility predicate we had to use induction by $n-i$ instead of
$i$, and this induction does not work if $n$ is replaced by
$\omega$.\par 
Proof theoretically, when moving to at least $\omega$-iterated 
OS, we go beyond the strength of $\Pi_1^1-\CA$, so
$\Pi^1_1$-arguments, which we used in the intuitive  well-ordering
proofs there, do no longer work.\par 
Our proof proceeded in some sense by induction on the 
lexicographic ordering of $(\T_0 ,\T_1 ,\ldots, T_{n-1} )$.
Whereas the  lexicographic ordering
on tuples of fixed length is well-ordered, if the underlying orderings are,
this does no longer hold for tuples of arbitrary length. However,
for sequences descending along some well-ordering this holds. 
The solution for our problem is now the following:
Introduce a function $\level:\T_\mu \ar \L$,
where \break 
$(\L,\prec_\L)$ is an ordering.
Let $\T_\mu^{\leq l}$ ($\T_\mu^{\prec l}$) 
be the restriction of $\T_\mu$
to those $a$ such  that $\level_\mu (a) \leq l$
($\level_\mu (a) \prec_\L l$). Now let
$\T_\mu^{\leq l}$ refer to $\T_\nu^{\prec l}$ only (if $\mu < \nu$), 
i.e. $\level_\nu[\k_{\mu,\nu}(t)]$\break
$\prec_\L \level_\mu(t)$ for
$\mu < \nu $, and assume $\T_\mu^{\leq l} \segment \T_\mu$,
i.e. $r \prec_\mu s \ar \level_\mu(r) \preceq_\L \level_\mu(s)$.
Then we will proceed by working on
sequences 
$\T_{\mu_1}^{\leq l_1},\T_{\mu_2}^{\leq l_2}
,\ldots $ such that $l_1 \succ_\L   l_2 \succ_\L \cdots$.\par 
We need now that  $(\L,\prec_\L)$ is a well-ordering. But to demand this 
directly would be a too strong requirement. What suffices is, to
have functions $\krmtil_\mu: \L \ar \T_\mu$ such that,
if we define $\L \restriction_{\nu<\sigma} B_\nu$ as usual,
but referring to $\krmtil_\mu$,
well-ordering of $\L \restriction_{\nu<\sigma} B_\nu$ 
reduces to well-ordering of $B_\nu$. We only need additionally to demand
that, if we relativize the sets of terms and the levels to some set $B$,
the levels of the terms in the relativized sets are in the 
relativized set  of levels, i.e.
$\all a \in \T_\nu.\krmtil_\mu(\level(a)) \subseteq 
\k_{\nu,\mu}(a)$.\par 
A last necessary modification is that, since we are no longer 
introducing first the $n$th OS completely, then the $(n-1)$th OS completely,
etc.,
we need to demand that there is a descend in length
when moving to higher components, 
i.e. $\nu> \mu \ar \length[\k_{\mu,\nu}(r)]< \length(r)$.\par 
%
%
%
%Somehow our definition of OS corresponds to the definition of
%the $\psi$-function as pointed out at the end of last section,
%where we close $\C_i(\alpha)$ under the full function
%$\lambda \beta.\psi_j(\beta)$ for $j>i$, with no restriction
%$\beta < \alpha $. This definition does again not work for the
%definition of $\psi_\alpha$ for $\alpha \geq \omega$.  One overcomes
%this problem by defining the functions $\psi_\alpha$ simultaneously.\par 
%If we look at the porcess in Lemma \ref{lemintuitnos}, in
%its generalization an $\omega$-OS would call now for an $\omega$-OS as a 
%subprocess and this 
%gives a loop. We can avoid such a loop,  if we bound every of these
%processes to run up to some element of a new well-ordering in such
%a way that, when we call a subprocess, this subprocess will 
%be restricted to a smaller element of this well-ordering.\par 
%For the case of a $\sigma$-times iterated OS,
%we start now with a family of OS 
%$\calF_\alpha = (\calG_\alpha,\k_{\alpha,\alpha})$ for
%$\alpha< \sigma$ and functions $\k_{\alpha,\beta}:
%\T_\alpha \ar_\omega \T_\nu$.
%We add now a new ordered set $(\L, \prec_\L)$ of levels
%together with functions $\level_\alpha:\T_\alpha \ar \T_\sigma$ and
%component functions $\krmtil_\alpha:\L \ar \T_\alpha$.
%The relativization of $\L$ to some well-ordered subsets of 
%$\T_\alpha$ should be now well-ordered and $\level_\alpha$ should
%respect the komponents of its underlying sets:
%$\krmtil_\nu(\level_\alpha(r))\subseteq \k_{\alpha,\beta}(r)$;
%we can weaken this condition, to
%$\krmtil_\nu(\level_\alpha(r))\subseteq \k_{\alpha,\beta}(r)
%\cup 
%\bigcup_{\xi<\sigma}
%\krmtil_{\beta}[\level_{\xi}[\k_{\alpha,\xi}(r)]]$,
%i.e. the components need essentially 
%only occur in a heriditarily subterm of $r$.\\
%
%
%
Apart from the conditions stated before we
need the (na{\"{\i}}ve) generalization of the
conditions for $n$-OS, which will include
that for every $a$ only for finitely many $\nu$ $\k_{\mu,\nu}(a) 
\not = \emptyset$.
We can now (although for the concrete
examples this is not necessary, but it might be useful in the future)
weaken the condition (OS 2) in the sense that if $r \prec_\mu s$,
then one new alternative is that 
$\level_\mu(r) \prec_\L\level_\mu(s)$.\par  
%There is one more change to be made: 
%we have to strengthen the closure conditions.
%$\Cl_{\calF_\mu } = \T_\mu$ does no longer suffice, since we could
%have a term $f_0(f_1(f_2( \cdots$, where
%$f_i(f_{i+1}( ,\cdots \in \T_i$ having as only non empty
%componentset the set $i+1$th one, which is
%$\{ f_{i+1}( ,\cdots \} $. The new condition subsumes the old
%condition $\T_\mu = \Cl_{\calF_\mu}$.
%
%
%
%
%
%
%%%%%%%%%%%%%%%%%%%%%%%%%%%%%%%%%%%%%%%%%%%
%%
%                                         %
%%%%%%%%%%%%%%%%%%%%%%%%%%%%%%%%%%%%%%%%%%%
%
%
%
\begin{assumption}
\laprint{assordering}
In the following we assume that some 
well-ordering $(\Sigma,<)$
of order type $\sigma>0$, where $\sigma$ is an ordinal
below the
Bachmann-Howard ordinal, is given. Let $0= \min_<\Sigma$
We will identify $(\Sigma,<)$ with $\sigma$,
writing $\mu < \sigma$ for $\mu \in \Sigma$,
$\Omega_\mu$ for $\Omega_{f(\mu)}$, where
$f: \Sigma \ar \sigma$ is the order isomorphism, etc..
In the following we assume $\mu,\nu,\xi < \sigma$.
(Note that we use the same symbol
$<$ as for the ordering on ordinals).
\end{assumption}
%
%
%
%
%
%%%%%%%%%%%%%%%%%%%%%%%%%%%%%%%%%%%%%%%%%%%
%%
%                                         %
%%%%%%%%%%%%%%%%%%%%%%%%%%%%%%%%%%%%%%%%%%%
%
%
%
\begin{definition}{\rm 
\laprint{defsigmaos}
\begin{enumerate}
\ssubitem a A \underlinedef{$\sigma$-times iterated Ordinal-System-structure
relative}
\underlinedef{to $(\Sigma,<)$},
in short 
\underlinedef{$\sigma$-OS-structure} 
is a quadruple\\
$\calF= 
%%%%(\calGvec,\krmvec,\Lvec,\lengthvec)= 
((\calG_\mu),(\k_{\mu,\nu }),(\L,\prec_\L,(\level_\mu),
(\krmtil_\mu)),\length_\mu ) $\\ 
\hbox{\quad\ }$=((\calG_\mu)_{\mu <\sigma },(\k_{\mu,\nu })_{\mu,\nu< \sigma},
(\L,\prec_\L,(\level_\mu)_{\mu<\sigma},(\krmtil_\mu)_{\mu<\sigma}),
(\length_\mu)_{\mu<\sigma})$\\
such that, with $\calG_\mu = (\T_\mu,\prec_\mu,\prec'_\mu)$ and
$\calF_\mu:= (\calG_\mu,\k_{\mu,\mu},\length_\mu)$, it holds that
$\calF_\mu$ are OS-structures,
$\k_{\mu,\nu}: \T_\mu \ar_\omega  \T_\nu$, 
$(\L,\prec_\L)$ is an ordering,
$\krmtil_\mu: \L \ar_\omega \T_\mu$ and $\level_\mu:\T_\mu \ar \L$
($\mu,\nu \in \Sigma$).\\
We will always assume that $\T_\mu$ are disjoint and $\L$ is disjoint
from $\T_\mu$ and write therefore
$\prec$, $\prec'$, $\length$, $\level$ instead of
$\prec_\mu$, $\prec'_\mu$, $\length_\mu$, $\level_\mu$
and $\prec$ instead of $\prec_\L$. Let in the following $\calF$ be as above.
%
%
%
\ssubitem{ab}
If $\calF$ is a $\sigma$-OS-structure  as above,
$A \subseteq \T_\mu$, $B \subseteq \T_\nu$, $A \restriction_{\nu} B:= 
A \cap \k_{\mu,\nu}^{-1}[B]$.
If $B_\nu  \subseteq \T_\nu$ for all $\nu < \xi$, then
$A \restriction_{\nu < \xi} B_\nu:= 
A \cap \bigcap_{\nu < \xi} \k_{\mu,\nu}^{-1}[B_\nu]$. 
In the same way we define for $A \subseteq \L$
$A \restriction_\nu B$, $A \restriction_{\nu<\xi}B_\nu$,
referring to $\krmtil_\nu$ instead of $\k_{\mu,\nu}$.
%\ssubitem {ac} In an $\sigma$-ordinal system structure as above we define
%for $\mu<\nu<\sigma$, $a \in \T_\mu$,
%$\T^{\preceq a}_\nu:= \{ b \in \T_\nu \mid b \prec \k_{\mu,\nu}(a) \lor 
%\level(b) \prec \level(a) \} $. 
\ssubitem b A $\sigma$-OS-structure as above is
a \underlinedef{$\sigma$-times iterated Ordinal System} relative
to $\Sigma$,
in short \underlinedef{$\sigma$-OS}, if for all $\mu,\nu,\xi<
\sigma$ and $r,s \in \T_\mu$ the following holds\\
\condition{\sigma-\OS\ 1} $\k_{\mu,\mu}(r) \prec r$;\\
\condition{\sigma-\OS\ 2} $\mu \leq \nu \ar \length[\k_{\mu,\nu}(r)]< 
\length(r)$;\\
\condition{\sigma-\OS\ 3} if $r \prec  s$, then 
$r \preceq \k_{\mu,\mu}(s) \lor r \prec ' s \lor \level(r) \prec \level(s)$;\\
%$ r \preceq \k_{\mu,\mu}(s) \lor (r \prec ' s \land 
%\all \nu < \sigma.\mu \leq \nu \ar \k_{\mu,\nu}(r) \subseteq \T_\nu^{\preceq s})$.
%%$r \preceq \k_{\mu,\mu}(s) \lor (r \prec ' s  
%\lor \level(r)\prec \level(s)$;\\
%\condition{\sigma-\OS\ 3} $\level[\k_{\mu,\nu}(a)] <
%\level(a)$ for all $\mu,\nu$, $a \in \T_\mu$;\\
\condition{\sigma-\OS\ 4} if
$\xi < \nu$, then
$\k_{\nu,\xi} \circ \k_{\mu,\nu} \subseteq \k_{\mu,\xi}$;\\
\condition{\sigma-\OS\ 5} $\k_{\mu,\xi'}(r) = \emptyset$ for almost all
$\xi'< \sigma$;\\
\condition{\sigma-\OS\ 6} if $\mu<\nu$,  then
$\level[\k_{\mu,\nu}(r)]
\prec \level(r)$;\\
\condition{\sigma-\OS\ 7} if $r \prec  s$, then $\level(r)\preceq \level(s)$;\\
\condition{\sigma-\OS\ 8}
$\krmtil_\nu(\level(r))\subseteq \k_{\mu,\nu}(r)$;\\
\condition{\sigma-\OS\ 9} if $A_\xi \subseteq \T_\xi$ are $\prec$-well-ordered
($\xi< \sigma$), then
$(\T_\nu \restriction_{\mu< \sigma}A_\mu, \prec ')$ and \break 
\conditionbox{} $(\L \restriction_{\mu< \sigma}A_\mu, \prec)$ are well-ordered, too.
%
%
%
\ssubitem c A $\sigma$-OS $\calF$ is 
{\em well-ordered}, if $(\T_0,\prec)$ is (where $0=\min_<{\Sigma}$), 
and its {\em order type} is 
that of $(\T_0,\prec)$. It is 
{\em primitive recursive}, if
the involved sets (including $\Sigma$)
are primitive recursive subsets of the natural numbers
(parameterized in $\Sigma$, i. e.
$t \in \T_{\mu}$ is primitive recursive in $t$ and $\mu$),
all functions, relations (including $<$) 
are primitive recursive, 
the finitely many $\nu$ such that $\k_{\mu,\nu}(a)
\not = \emptyset$ can be computed primitive recursively from 
$\mu$, $\nu$, $a$, and all properties
(including linearity of $<$ and that the chosen $\nu$ such
that $\k_{\mu,\nu}(a) \not = \emptyset$ are the only ones)
except of the well-ordering condition can be shown in
primitive recursive arithmetic.
It is {\em elementary}, if additionally
the well-ordering condition follows in $\PRA$  in the sense of
reducibility of transfinite induction in $\PRA$ to transfinite induction 
on $\{ (\mu,a) \mid \mu < \sigma \land a \in \T_\mu \}$
with the lexicographic ordering 
$(\mu,a) < (\nu,b) \Iff \mu < \nu \lor 
(\mu = \nu \land a <_\mu b)$.
\ssubitem d ``Two $\sigma$-OS are isomorphic'' is defined
as for $n$-OS.
\end{enumerate}}
\end{definition}
%
%
%
{\bf Comparison with $n$-OS.} $n$-OS have now been defined twice,
since $\sigma$ can be finite. So more precisely we have to
distinguish between ``finite $n$-OS'' and ``transfinite $n$-OS''.
However, one can
see easily that a finite $n$-OS $\calF$ can be considered as a 
special cases of a transfinite $n$-OS. 
Let $\Sigma:= \{ 0 ,\ldots, n-1 \} $, $<$ be the usual ordering on 
$\Sigma$, $\L:= \{ n-1, n-2 ,\ldots, 0 \} $, $\krmtil_i(j):= \emptyset$,
$i \prec j :\Iff j < i$, $\level(r):= i$  for $r \in \T_i$. Replace
$\length_i(r)$ by $\length'_i(r):=
\max \{ \length_i(r) \} \cup \bigcup_{j \geq i}\length'_j[\k_{i,j}(r)]$
(defined by recursion on $n-i$ side-recursion on $\length(r)$).
One can easily see, that, if we extend
the structure by the above and replace $\length$ by
$\length'$, we get 
a transfinite $n$-OS. This illustrates again why
a na{\"{\i}}ve generalization of  $n$-OS to $\omega$-OS does not
work: We get the reverse  ordering of the natural numbers,
which is not well-ordered.
%
%
%%%%%%%%%%%%%%%%%%%%%%%%%%%%%%%%%%%%%%%%%%%
%%
%                                         %
%%%%%%%%%%%%%%%%%%%%%%%%%%%%%%%%%%%%%%%%%%%
%\begin{deflemma}
%\laprint{lemclosureossigma}
%\begin{enumerate} 
%\ssubitem a
%Let $(\calGvec,\krmvec)$ be a $\sigma$-OS, $\mu < \sigma$,
%$A_\nu \subseteq \T_\nu$ ($\nu < \mu$),
%$\Avec:= (A_\nu)_{\nu<\mu }$.
%Let $\M_\nu:= A_\nu$ for $\nu<\mu$,
%$\M_\nu:= \bigcap_{\xi<\mu}\k_{\nu,\xi}^{-1}(A_\xi)
%\cap \T_\nu$.
%For $s \in \T_\mu$, define simultanteously
%for all $\nu$
%inductively set $\C^\Avec_\nu(s) \subseteq \T$ by:
%If $\nu<\mu$, $\C^\Avec_\nu(s):= A_\nu$.\\
%If $r \in \M_\mu$, $r \preceq \k(s)$, $r \in \C^\Avec_\nu$.\\
%Assume now $r \in \M_\nu$, $\k_{\nu,\xi}(r) \subseteq 
%\C^\Avec_\xi(s)$ for all $\xi<\sigma$.
%If now $\level(r) \prec \level(s)$
%or $\nu = \mu$ and $r \prec ' s$, then
%$r \in \C^\Avec_\nu(s)$.
%Then $\C^\Avec_\mu(s)= \{ t \in \M_\mu  \mid t \prec s \} $,
%and for $\mu < \nu $,
%$\C^\Avec_\nu(s)= \{ t \in \M_\nu  \mid 
%\level(t) \prec \level(s) \land 
%\all \xi < \nu. \k_{\nu,\xi}(t) \subseteq 
%\C^\Avec_\xi(s) \} $,
%\ssubitem b
%If $C^\Avec_\nu(s)$ is defined as before, $\mu < \nu$,
%$r \in C^\Avec_\nu(s)$, then for all $\xi <\nu$
%follows $\k_{\nu,\xi}(r) \subseteq \C^\Avec_\xi(s)$.
%\end{deflemma}
%%
%%
%%
%{\bf Proof:} \refact a ``$\subseteq$'' Induction on the definition
%of $\C^\Avec_\nu(s)$. Assume $t \in \C^\Avec_\nu(s)$ according to induction.
%If $\nu>\mu$, the assertion follows by IH. Otherwise, if
%$\level(t) \prec \level(s)$ or
%$t \preceq_\mu \k_\mu(s)$ the assertion follows as well using the IH.
%If $t \prec ' s$, then by $\k_\mu(s) \subseteq
%\C^\Avec_\nu(s)$ follows
%$\k_\mu(t) \prec ' s$, and therefore
%$t \prec s$.\\
%``$\supseteq$'': We show the assertion using
%$\T_\nu = \Cl_\nu$ by induction on
%$t \in \Cl_\nu$. If $\nu> \mu$, we have for $\xi<\nu$,
%$\k_{\nu,\xi}(t) \subseteq \C^\Avec_\xi(s)$,
%and for $\nu \leq  \xi$ by induction
%pon $\xi$ and using $\level_\xi(\k_{\nu,\xi}(t)) \preceq 
%\level(t)$, $\k_{\xi,\delta}[\k_{\nu,\xi}(t)]
%\subseteq \k_{\nu,\delta}(t)$ and the IH that
%$\k_{\nu,\xi}(t) \subseteq \C^\Avec_\xi(s)$ and
%therefore $t \in \C^\Avec_\xi(s)$.
%If $\nu = \mu$ follows from $t \preceq_\mu  \k_{\mu,\mu}(s)$,
%therefore $t \in \C^\Avec_\mu(s)$
%or $\level (t)\prec  \level(s)$ and
%as before the assertion or $t \prec ' s$,
%and using that $\level_\xi[\k_{\mu,\xi}(t)]
%\prec \level(t) \preceq \level(t)$ similarly
%as in the case $\mu<\nu$ 
%$t \in \C^\Avec_\mu(s)$.\par 
%\refact b Immediate by induction on the definition
%%
%
%
%
\subsection{Constructive Well-ordering Proof.}
%%%%%%%%%%%%%%%%%%%%%%%%%%%%%%%%%%%%%%%%%%%
%%
%                                         %
%%%%%%%%%%%%%%%%%%%%%%%%%%%%%%%%%%%%%%%%%%%
%
%
%
\begin{theorem}
\laprint{theoremstrengthesigmaos}
\begin{enumerate} 
\ssubitem a Every $\sigma$-OS is well-ordered. 
\ssubitem b Every elementary $\sigma$-OS has order type below 
$|\ID_{\sigma}|= \psi(\epsilon_{\Omega_\sigma+1})$. 
\end{enumerate} 
\end{theorem}
%
%
%
{\bf Proof:} \refact a: Define by recursion
on $\mu \in \Sigma$ 
inductively $\M_\mu, \Acc_\mu \subseteq
\T_\mu$:\\ 
$\M_{\mu}:= \T_\mu \restriction_{\nu < \mu}\Acc_\nu$,
$\Acc_\mu:= \Acc_{\prec}(\M_\mu)$.\\
$\Acc_\mu':= \T_\mu \restriction_{\nu < \sigma} \Acc_\nu$,
$\Acc_\L':= \L \restriction_{\nu < \sigma \Acc_\nu}$.\\
Further, if $l \in \L$, $A \subseteq \T_\mu$
$A^{\preceq l}:=
\{ x \in A \mid \level(x) \preceq  l \} $,
similarly we define $A^{\prec l}$.\\
$(\Acc_\mu,\prec)$, 
$(\Acc_\mu',\prec')$ and
$(\Acc_\L',\prec)$  are well-ordered.\\
We show by induction on $l \in \Acc_\L'$
$$\all l \in \Acc_\L'. \all \mu < \sigma .
\M_\mu^{\preceq l} \subseteq \Acc_\mu \enspace ,$$
and
assume $l$ according to induction.\\ 
We show 
$$\all s \in {\Acc_\mu'}^{\preceq l}.s \in \Acc_\mu$$ 
by (side-)induction
on $(\Acc_\mu',\prec')$ and assume $s$ according to induction:\\
$s \in \M_\mu$. 
We define $\C^\nu(s) \subseteq \T_\nu$ ($\nu<\sigma$):\\
For $\nu<\mu$, $\C^\nu(s):= \Acc_\nu$.\\
$\C^\mu(s):= \{ r \in \M_\mu \mid r \prec  s \} $.\\
If $\mu<\nu$,
$\C^\nu(s):= 
\{ r \in \T_\nu \mid \level(r) \prec \level(s)\land 
\all \xi < \nu.\k_{\nu,\xi}(r) \subseteq  \C^\xi(s) \} $.\\
Note that by ($\sigma$-OS 4) for $\nu>\mu$\\
$\C^\nu(s)= \{ r \in \T_\nu \mid \level(r) \prec  \level(s)
\land \all \xi < \mu(\k_{\nu,\xi}(r) \subseteq \Acc_\xi)
\land \k_{\nu,\mu}(r) \prec s \land $\\
\hbox{\qquad \qquad\  }$\all \xi(\mu < \xi < \nu \ar \level[\k_{\nu,\xi}(r)]
\prec \level(s)) \}$.\\
The last  equation allows to define $\C^\nu(s)$ in $|\ID_\sigma|$ as needed 
in \refact b.\\
%
%
%
We prove
\fornum{\all \nu(\mu \leq \nu \ar \all \xi < \sigma.
\k_{\nu,\xi}[\C^\nu(s)]\subseteq \C^\xi(s))}{$\ast$}%
\noindent by induction on $\xi$:\\
For $\xi < \nu$ this follows by the definition of 
$\C^\nu(s)$.
Otherwise for $\xi'<\xi$,
$\k_{\xi,\xi'}[\k_{\nu,\xi}[\C^\nu(s)]] 
\subseteq \k_{\nu,\xi'}[\C^\nu(s)] \subseteq \C^{\xi'}(s)$
by IH. Let $r \in \C^\nu(s)$.
In case $\mu = \nu = \xi$ we have
$\k_{\nu,\xi}(r) \prec r \prec  s$
and, if $\mu < \xi$, it follows $\level[\k_{\nu,\xi}(r)]
\preceq \level(r) \preceq \level(s)$,
and one of the $\preceq$ is actually $\prec$,
so in both cases we get the assertion.\\
%
%
%\fornum{\all \nu.\mu \leq \nu\ar 
%\all r \in \T_\nu.r \in \C^\nu(s) \ar \all \xi.
%\krmtil_\xi(\level(r))\subseteq \C^\xi(s)}{$\ast \ast $}%
%follows now by induction on $\length(r)$ simultaneously for
%all $\nu$, using \\
%$\krmtil_\xi[\level[\C^\nu(s)]]
%\subseteq \k_{\nu,\xi}[\C^\nu(s)]
%\cup \bigcup_{\delta < \sigma}
%\krmtil_{\xi}[\level[\k_{\nu,\delta}[\C^\nu(s)]]]$
%and $(\ast)$.\\
%%
We show by (side-side-)induction on $\length(r)$
simultaneously for all $\nu$
$$\all \nu.\all r \in \C^\nu(s) .r \in \Acc_\nu$$
and assume
$r$ according to induction, $r \in \C^\nu(s)$.
If $\nu < \mu$, $r \in \Acc_\nu$.
Assume
$\mu \leq \nu$. 
By side-side-IH and $(\ast)$ follows for
all $\xi$, $\k_{\nu,\xi}(r)  \subseteq 
\Acc_\xi$ and
$\krmtil_\xi(\level(r)) \subseteq \k_{\nu,\xi}(r) 
\subseteq \Acc_\xi$,
$r \in \Acc_\nu'$,
$\level(r) \in \Acc_\L'$.\\
If $\mu = \nu$, 
$r \prec s$, $r \preceq \k_{\mu,\mu}(s) \subseteq
\Acc_\mu$, $r \in \M_\mu$, $r \in \Acc_\mu$
or $r \prec ' s$, $r \in \Acc_\mu '$, 
$\level(r) \preceq \level(s) \preceq l$, and
by side-IH $r \in \Acc_\mu$,
or $\level(r) \prec \level(s)$ and as in
the next case ``$\mu < \nu$'' follows the assertion.\\
If $\mu < \nu$,
$\level(r) \prec \level(s) \preceq l$,
$\level(r) \in \Acc_\L'$,
$r \in \M_\nu$,
by main IH $r \in \Acc_\nu$.\\ 
Now follows  $\all r \in \M_\mu(r \prec  s
\ar r \in \Acc_\mu)$, and, since $s \in \M_\mu$,
$s \in \Acc(\M_\mu) = \Acc_\mu$, and the side-induction is
complete.\\
Now by induction on $\length(s)$ it follows 
$\all s \in \M_\mu^{\preceq l}. s \in \Acc_\mu$: If $s \in 
\M_\mu^{\preceq l}$ we first show
$\all \nu.\k_{\mu,\nu}(s) \subseteq
\Acc_\mu$. For $\nu< \mu$ this follows by
assumption. For $\mu \leq \nu$ we show
this by induction on $\nu$. With the usual argument
we get using the IH $\k_{\mu,\nu}(s) \subseteq 
\M_\nu^{\preceq l}$, and therefore by IH
$\k_{\mu,\nu}(s) \subseteq \Acc_\nu$. Therefore
$s \in \Acc'_\mu$ and by the proven
statement of the side-induction it follows $s \in \Acc_\mu$.   
Therefore the main-IH is completed.\\ 
Now follows by induction on $\length(r)$, simultaneously
for all $\mu$,
$\all r \in \T_\mu(r \in \Acc_\mu \land 
\level(r) \in \Acc'_\L)$:
By IH $\k_{\mu,\nu}(r) \subseteq \Acc_\nu$,
$\krmtil_{\nu}(\level(r)) \subseteq \Acc_\nu$,
$\level(r) \in \Acc'_\L$, $r \in \M_\mu^{\preceq 
\level(r)}$,
$r \in \Acc_\mu$.\\
Therefore it follows $\T_\mu$ is well-ordered and we are done.\par 
\smallskip 
\refact b The proof of \refact a can be carried out
in $\ID_\sigma$. (Note, that transfinite induction over $\sigma$
is one of the axioms of $\ID_\sigma$).\QED 
%\smallskip 
%
%\refact c 
%Again we 
%start take the system in \cite{buchholzschuettebook} (or 
%a modified version of \cite{buchholzanewsys}), however
%have to change it as follows: 
%Instead of the function $\lambda\mu.\Omega_\mu$,
%we have as basic constants $\Omega_\mu$ for $\mu< \sigma$
%Instead of having the functions $\lambda \mu,\nu.\psi_\mu \nu$
%we define the functions $\lambda \nu.\psi_\mu \nu$ for
%all $\nu < \Sigma$. 
%Correspondingly, we add to the sets $\C_\sigma(\nu)$ 
%always all $\Omega_\xi$ 
%and $\psi_\nu \xi$, if $\xi \in \C_\sigma(\mu) \cap \xi$
%and $\nu < \sigma$. From this we define
%now the ordinal notation system $\OT'$ (ordinals
%below $\sigma$ are denoted by elements of $\Sigma$).
%$\OT$ be the set of ordinal notation systems built
%from ordinalnotations $\prec \omega_m(\Omega_\sigma+1)$ only,
%and let for $\mu \in \Sigma$,
%$\T_{\mu}:= \OT \cap [\Omega_{\mu},\Omega_{\mu}[$,
%$\T_{\leq \mu}:= \bigcup_{\nu \leq \mu}\T_\nu$.
%Let the $\prec'$ on $\T_\mu$ correspond to the
%ordering on
%$\{ \Omega_\mu \} \preccirc 
%((\omega^{\T_{\leq \mu}}\cdot \T_{\leq \mu})\restriction \T_\mu)
%\preccirc (\psi_\mu(\M) \restriction \T_\mu)$,
%where $M:= \Omega_{\sigma} \uparrow(m)\T_{< \sigma}$
%We define $\k: \T_{< \sigma} \ar_\omega \T_{< \sigma}$ by
%induction on the length of the terms.
%$\k(0):= \k(\Omega_\xi):= \emptyset$,
%$\k(\omega^a+b):= \{ a,b \} $, $\k(\psi(a)):= 
%\krmtil(a)$, $\krmtil$ defined according to defintion
%\refcom{defOmegapower}b.\\
%Let $\krmcheck: \T_{<\sigma} \ar_\omega \T_{< \sigma}$
%defined by
%$\krmcheck(t):= \k(t) \cup \krmcheck[\k(t)]$.\\
%Now define $\k'_{\mu,\nu},
%\krmcheck'_{\mu,\nu}: \T_\mu \ar_\omega \T_\nu$, 
%y$\k'_{\mu,\nu}(t) := \k(t) \cap \T_{\nu}$,
%$\krmcheck'_{\mu,\nu}(t) := \krmcheck(t) \cap \T_{\nu}$.
%Define $\k_{\mu,\nu}; \T_\mu \ar_\omega \T_\nu$,
%such that $\k_{\mu,\nu}(t) \subseteq \krmcheck_{\mu,\nu}(t)$
%(and therefore $\k_{\mu,\nu}(t)=\emptyset$ for almost all
%$\nu$) by: 
%$\k_{\mu ,\nu }(t):= 
%\k'_{\mu ,\nu }(t) \cup \bigcup_{\nu<\xi \in \Sigma}
%\k_{\xi ,\nu }[\k_{\mu,\xi}(t)]$.
%$\krmvec':= (\k'_{i,j})_{i,j < n}$
%(more precisely this definition is by induction on the
%reversed order $<$, which is possible, since
%$\k_{\mu,\nu}(t)= \emptyset$ if $\krmcheck'_{\mu,\nu}(t) = \emptyset$.\\
%We define $\L:= M$, and $\prec$ is the  ordering induced from $\OT$.
%$\krmtil_\mu(t):= \krmtil(t) \cap \T_\mu$.\\
%$\level(t):= \min \{ s \mid t \in \C_\mu(s) \} $,
%which can be defined  as
%$\max \{ s \mid \exists \nu \in \Sigma.\mu \leq  \nu  \land
%\psi_\nu(s) \in \krmcheck(s) \} $ and
%one verifies that this is primitive recursive.\\
%We verify, that for the primitive recursive sets
%$\C_\mu(s)$ corresponding to the sets $\C_\mu(\nu)$
%we have
%$t \in \C_\mu(s) \ar \all \nu < \sigma.\mu \leq \nu 
%\ar \k_{\mu,\nu}(t) \subseteq \C_\mu(s)$.\\
%We verify, that $((\T_\mu,\prec,\prec '),(\k_{\mu,\nu}),
%(\L,\prec,(\krmtil_\mu))$ is a $\sigma$-OS.\\ 
%($\sigma$-OS 1): For 
%$r = \sigma$ and $\omega^s+t$ this is clear,
%and for $r=\psi_\mu(s)$ this follows
%by $s \in \C_\mu(s)$, $\k_{\mu,\mu}(s) \subseteq 
%\C_\mu(s) \cap \T_\mu \prec \psi_\mu(s)$.
%($\sigma$-OS 2)????????????????????????
%($\sigma$-OS 3) follows as in the previous proofs,
%($\sigma$-OS 4) was enforced,
%($\sigma$-OS 5) since the components of a term have
%a smaller length.
%($\sigma$-OS 6) is only difficult for the term
%$\psi_\mu(r)$, and here we have
%$\psi_\mu(r) \not \in \C_\mu(r)$,
%$r \in \C_\mu(r)$, therefore $\k_{\mu,\nu}(r) \subseteq 
%\C_\mu(r)$, $\level[\k_{\mu,\nu}(r)]
%\prec r = \level(r)$. ($\sigma$-OS 7) is 
%easy and ($\sigma$-OS 8) can be verified by induction on the length
%of the terms. 
%In ($\sigma$-OS 9) we verify first
%that $\TI$ on $M$ relativezed to $A_\mu$
%reduces to well-ordering of the $A_\mu$ and
%of $(\Sigma,<)$, which implies the
%condition for the relativization of $\L$   and from which
%the assertion for the argument sets follows as usual.\\
%The resulting $\sigma$-OS has now strength
%$\psi_0(\omega_n(\Omega_\sigma +1))$
%and that this is in the limit the strength of $|\ID_\sigma|$
%is a fact usually referred to as ``folklore'', although for precisely the 
%system we have taken it has not been carried through.
%
%
\subsection{$\sigma$-Ordinal Function Generators}
The ordinal function generators referring to $\sigma$-OS are now
defined similarly as for $n$-OFG. The only difference is
that in the definition of $\C_\mu(a)$ we will 
refer only to $b \in \Arg_\nu$ ($\nu>\mu$)
such that $\level_\nu(b)< \level_\mu(a)$, which
is the obvious adaption of the principles for $\sigma$-OS to
OFG.\par 
%In the definition of $n$-OFG we could have added to 
%$\C_n(a)$ $\eval(b)$ only for those $b$ such that
%$\NF(b)$ holds and in the definition of $\NF(b)$,
%$\k_{n,l}(b)$ ($l> n$) did not play a role, since
%$\k_{n,l}(b) \subseteq \C_n(b)$ follows for all $b \in \Cl \cup
%\Arg[\Cl]$. In the $\sigma$-OFG, we have to demand
%for $\NF(b)$ that $\k_{\mu,\nu}(b) \subseteq \C_\mu(b)$ 
%for all $\nu \geq \mu$. We need to demand for
%adding $\eval(b)$ to $\C_\mu(a)$ that $\NF(b)$ holds,
%since we can no longer obtain from $b$ a $c$ such that
%$\eval(c)= \eval(b)$ and $\NF(c)$ by moving
%to the components of $b$.
%%%%%%%%%%%%%%%%%%%%%%%%%%%%%%%%%%%%%%%%%%%
%%
%                                         %
%%%%%%%%%%%%%%%%%%%%%%%%%%%%%%%%%%%%%%%%%%%
%
%
%
\begin{definition}
\laprint{defsigmaOFG}{\rm
\begin{enumerate}
\ssubitem a A {\em $\sigma$-ordinal function generator}, in short
{\em $\sigma$-OFG} is a triple
$\calO:= (\calA,\calL,\krmtil)$, where 
$\calA= 
(\Arg_\mu,\k_\mu, \lrm_\mu,<_\mu',\level_\mu)_{\mu<\sigma}$,
$\calL = (\L,<_\L)$,
$\Arg_\mu$ are classes (in set theory),
$\k_\mu,\lrm_\mu: \Arg_\mu \ar_\omega \Ord$,
$\level_\mu: \Arg_\mu \ar \L$,
$\all a \in \Arg_\mu.\lrm_\mu(a) \subseteq \k_\mu(a)$,
$<_\mu'$ is a well-ordered relation on $\Arg_\mu$,
$\all a,b \in \Arg_\mu(a <_\mu ' b \ar 
\level_\mu(a) \leq'_\mu \level_\mu(b))$,
$\calL$ is a well-ordering,
$\krmtil: \calL \ar_\omega \Ord$,
$\all a \in \Arg_\mu.\krmtil[\level_\mu(a)] \subseteq \k_\mu(a)$.\\
We assume in the following that 
$\Arg_\mu$ and $\L$ are disjoint and omit therefore the index
$\mu$ in $\k$, $\lrm$, $<'$, $\level$ and the index $\L$ in $<'$.
Let $\calO$ always be as above.
\ssubitem b $\Omega_0:= 
0$, $\Omega_\mu:= \aleph_\mu$ otherwise.
\ssubitem c A $\sigma$-OFG is {\em cardinality based},
if for  all $B \subseteq \Ord$ countable $\k_\mu^{-1}(B)$ is  countable, 
$\lrm_\mu(a) \subseteq \k_\mu(a) \cap [\Omega_\mu,\Omega_{\mu+1}[$
and 
$\all a \in \Arg_\mu.\all \nu>\mu.
\level_\nu[\k(a) \cap [\Omega_{\nu},\Omega_{\nu+1}[]
< \level_\mu(a)$.
In this case we define for $a \in \Arg_\mu$,
$\k_{\mu,\nu }(a):= \k(a) \cap [\Omega_\nu,\Omega_{\nu+1}[$
and for $a \in \L$ $\krmtil_\nu(a):= \k(a) \cap 
[\Omega_\nu,\Omega_{\nu+1}[$.
\ssubitem d If $\calO$ as above is a cardinality based  $\sigma$-OFG, we define
for $a \in \Arg_\mu$ simultaneously for all $\mu<\sigma$
by recursion on $\level_\mu(a) \in \L$,
side-recursion on $(<',\Arg_\mu)$, 
$\eval_\mu(a) \in \Ord$ and subsets
$\C_\mu(a) \subseteq \Ord$ by:
$\C_\mu(a):= \bigcup_{n=0}^\omega \C_\mu^n(a) $ where \\
$\C_\mu^0(a):= [0,\Omega_\mu[\cup (\bigcup (\k_\mu(a) 
\cap \Omega_{\mu+1}))
\cup \lrm_\mu(a)$,\\
$\C_\mu^{l+1}(a):= \C_\mu^l(a) \cup $\\
\hbox{\qquad \qquad \quad}$
\bigcup_{\nu \geq \mu}\{ \eval_\mu(b) \mid b \in \Arg_\nu,
((\nu = \mu \land b <' a \land \level(b) = \level(a)) \lor$\\
\hbox{\qquad \qquad \quad \qquad \quad}$
(\level(b) < \level(a))) \land  
\k(b) \subseteq \C_\mu^l(a) 
\land \NF_\nu(b) \} $,\\
where $\NF_\nu(b):\Iff \k(b) \subseteq \C_\nu(b)$.\\
$\eval_\mu(a):= \min \{ \alpha  \mid \alpha \not \in \C_\mu(a) \} $.
\ssubitem e Let $\calO$ be a cardinality based
$\sigma$-OFG. Referring to the definition of $\NF(b)$ as above and
assuming that $\eval \restriction \NF_\mu$ is injective, which
will be shown later, 
$\NF_\mu$, $\Cl_\mu$, $\Cl$,
$\Arg[\Cl]_\mu$ and $\Arg[\Cl]$ are defined as in Definition \ref{defnOFG}.\\
Further we define 
$\k^0_{\mu,\nu},\k'_{\mu,\nu}: \Arg[Cl]_\mu \ar_\omega \Cl_\nu$ by\\
$\k_{\mu,\nu}^0(b):= 
\eval_\nu^{-1}[\k_{\mu,\nu}(b)] \cap \Cl_\nu$,
$\k_{\mu,\nu}'(b):= \k^0_{\mu,\nu}(b) \cup
\bigcup_{\nu  \leq  \xi<\sigma}\k'_{\xi,\nu}[\k^0_{\mu,\xi}(b)]$,\\
$\length$ as in Definition \ref{defnOFG} and
$\level'_\mu:\Cl_\mu \ar \L$ by
$\level'_\mu(r):= \max \{ \level_\mu(r) \} 
\cup \level'_\mu[\k_{\mu,\mu}(r)]$.
\ssubitem f
For $r,s \in \Arg_\mu$ we define 
$r \prec s :\Iff r \prec_\mu s :\Iff \eval_\mu(r) < \eval_\mu(s)$.
%\ssubitem f We define $\level'_\mu: \Cl_\mu \ar \L$ by:
%Let simultaneously for $\mu \leq  \nu$, $\level'_{\mu,\nu}:
%\Cl_\nu \ar \L$ be defined by:\\
%Let for $\nu \geq \mu$, $a \in \Cl_\nu$, $\S_{\mu,\nu}(a):= 
%\mycases{\S(a)&\text{if }\nu> \mu\\
%a&\text{otherwise.}}$ Then
%$\level'_{\mu,\nu}(a):= 
%\max( \{ \S_{\mu,\nu}(\level_\nu(a)) \} \cup 
%\bigcup_{\xi \geq \mu } \S_{\mu,\xi}[\level'_{\mu,\xi}[
%\k_{\nu,\xi}(a)]])$.\\
%$\level'_\mu(a):= \level'_{\mu,\mu}(a)$.\\
%Further define for $l \in \L$, $\Cl_{\mu,\nu}^{\leq' l} \subseteq \Cl_\nu$
%simultaneously for all $\nu$ inductively by:\\
%$\Cl_{\mu,\nu}^{\leq ' l}  = \Cl_{\nu}$ for $\nu < \mu$.\\
%if $\mu \leq \nu$,
%$b \in \Cl_\nu$, $\mu \leq \nu $,
%$\k(b) \subseteq 
%\bigcup_{\mu \leq \nu }\Cl_{\mu,\nu}^{\leq'  l}$,
%$\level(b) < l \lor (\mu = \nu \land  \level(b) = l)$, then
%$b \in \Cl^{\leq' l}_\nu$.
\end{enumerate}}
\end{definition}
%
%
%
%%%%%%%%%%%%%%%%%%%%%%%%%%%%%%%%%%%%%%%%%%%
%
%                                         %
%%%%%%%%%%%%%%%%%%%%%%%%%%%%%%%%%%%%%%%%%%%
%
%
%
\begin{lemma}
\laprint{lemalphaOFG1}
Let $\calO$ be a cardinality based $\sigma$-OFG.
\begin{enumerate} 
\ssubitem a Lemma \refcom{lemnOFG1}a - \refsub{lemnOFG1}{cd}  holds mutatis mutandis. 
%\ssubitem b $\C_\mu(a) \cap \Omega_{\mu+1} \segment \Ord$.\\
%Therefore especially $\eval_\mu(a) = \C_\mu(a) \cap \Omega_{\mu+1}$ 
%\ssubitem c If $a \in \Arg_\mu $,
%then $\eval_\mu(a) \in [\Omega_\mu,\Omega_{\mu+1}[$.
%\ssubitem {ca} If $\mu \leq \nu$, $a \in \Cl_\nu$,
%then $\level'_{\mu,\nu}(a) = 
%\min \{ l \mid a \in \Cl_{\mu,\nu}^{\leq l} \} $,
%especially 
%$\level'_\mu(a) = \min \{ l \in \L \mid \eval_\mu(a)
%\in\Cl_{\mu,\mu}^{\leq l}\}$.
%\ssubitem b If $a,b \in \Cl_\mu$, 
%$\eval_\mu(a) < \eval_\mu(b)$, then $\level'(a)
%\leq \level'(b)$. 
%\ssubitem d For $a,b \in \Arg[\Cl]_\mu$ we have
%\begin{eqnarray*} 
%\eval_\mu(a) < \eval_\mu(b) &\Iff& (a <' b \land \k_{\mu,\mu}(a) 
%< \eval_\mu(b) \land \all \nu .
%(\nu > \mu \ar (\all \xi \in \k_{\mu,\nu}(a)
%(\xi  < \k_{\mu,\nu}(b) \lor \level(c) < \level(b))))
%\lor \eval_\mu(a) < \k_{\mu,\mu}(b) \lor \eval_{\mu}(a) 
%\leq \lrm(b) \lor \exists \mu<\nu . \exists \xi
% \in \k_{\mu,\nu}(b).\\
%\eval_\mu(a) = \eval_\mu(b) &\Iff& (a <' b \land 
%\eval_\mu(b) = \max( \k(a)) \land
%\eval_i(b) \not \in \lrm(a)) \lor \\
%&& (b <' a \land \eval_i(a) = \max(\k(b)) \land \eval_i(a) \not \in \lrm(b)) \lor \\
%&&a=b
%\end{eqnarray*} 
%\ssubitem e For $a,b \in \Arg_i$, $\NF(a)$, $\NF(b)$
%$\eval_i(a) < \eval_i(b) \iff (a <' b \land (\k(a) \cap \Omega_{i+1}) < 
%\eval_i(b)) 
%\lor \eval_i(a) \leq (\k(b) \cap \Omega_{i+1}) $.\\
%Especially: $\eval_i \restriction \{ a \in \Arg.\NF(a) \}$ is
%injective.
\ssubitem f $\calF:= ((\Cl_\mu,\prec_\mu)_{\mu<\sigma},
(\k'_{\mu,\nu})_{\mu,\nu < \sigma},
(\length_i)_{i<n}, (\L,<_\L),(\level_\mu ')_{\mu<\sigma},
(\length_\mu)_{\mu})$ is a $\sigma$-ordinal system.
We will call any $\sigma$-OS-structure isomorphic to $\calF$ 
a $\sigma$-{\em ordinal system} based on $\calO$.
%%%\ssubitem d \refact a - \refact c hold for recursivity based 
%%%$\sigma$-OFG mutatis mutandis.
\end{enumerate} 
\end{lemma}
%
%
%
{\bf Proof:} \refact a: We write \refact a.(x) for
the assertion corresponding to Lemma \ref{lemnOFG1}(x).
\refact a.\refsub{lemnOFG1}a: clear.
\refact a.\refsub{lemnOFG1}b: Similarly as in Lemma 
\refcom{lemnOFG1}b. In the argument, which showed there
$\C_i^l(b) \subseteq \C_i(a)$ and in the new context now shows
$\C_\mu^l(b) \subseteq \C_\mu(a)$, we use
that $b <' a$, therefore $\level(b) \leq \level(a)$.
%Main induction on $\level(a)$, 
%side-induction on $a \in \Arg_\mu$.
%Assume $a$ according to IH. 
%We show $\all \alpha \in \C_\mu^n(a)\cap \Omega_{\mu+1}.
%[0,\alpha[ \subseteq \C_\mu(a)$ by side-side-induction
%on $n$: $n=0$: clear, since $\lrm_\mu(a) \subseteq \k_\mu(a)
%\cap \Omega_{\mu+1}$.
%$n \ar n+1$: If $\alpha  \in (\C_\mu^{n+1}(a) \setminus \C_\mu^n(a))\cap 
%\Omega_{\mu+1}$, then $\alpha = \eval_\mu(b)$ with
%$\k(b) \subseteq \C_\mu^n(a)$,
%$b <' a$ and $\NF(b)$,
%by side-IH and, since $\lrm(b) \subseteq \k(b)$ 
%$\bigcup (\k(b) \cap \Omega_{\mu+1} ) \cup \lrm(b) \subseteq \C_\mu(a)$
%and, since $\level(b) \leq  \level(a)$,
%by an immediate induction $\C_\mu^n(b) \subseteq \C_\mu(a)$,
%$\C_\mu (b) \subseteq \C_i(a)$, therefore
%$\alpha = \min \{ \beta \mid \beta \not \in \C_\mu(b) \}
%\subseteq \C_\mu(a)$.\\ 
\refact a.\refsub{lemnOFG1}c: as before;
\refact a.\refsub{lemnOFG1}{ca}: as before, using
$a < ' b \ar \level(a) \leq \level(b)$; \refact a.\refsub{lemnOFG1}{cb}:
as before easy induction on $\length(a)$; \refact a.\refsub{lemnOFG1}{cd}: as before.\\
%\refact e: \refact a - \refact b are as before.  \refact c
%Induction on $a$: Main induction on $n-i$, side-induction
%on $a$: We see easily 
%that $\C_i^n(a)$ can be defined as a $\Sigma$-function symbol
%in  $i$, $a$ and $n$, therefore as well $\C_i(a)$.
%\refact e ``$\leftarrow$'': $\eval_i(a) \in \C_i(b)$.
%``$\ar$'': If the right condition does not hold, then
%$a = b$ or the right condition holds with $a$, $b$ interchanged,
%therefore $\eval_i(b) \leq \eval_i(a)$.\\
\refact f: The only problem are ($\sigma$-OS 6), ($\sigma$-OS 7): 
We show for $r,s \in \Cl$, 
that if $\eval_\nu(r) \in \C_\mu(s)$,
then, in case $\nu = \mu$, $\level'_\nu(r) \leq \level'_\mu(s)$, and,
in case $\nu> \mu$, $\level'_\nu(r) < \level'_\mu(s)$, by
main induction on $\length(s)$, side-induction on $\length(r)$, 
which is immediate.\QED 
%
%
%
%
%
%
%%%%%%%%%%%%%%%%%%%%%%%%%%%%%%%%%%%%%%%%%%%
%%
%                                         %
%%%%%%%%%%%%%%%%%%%%%%%%%%%%%%%%%%%%%%%%%%%
%
%
%
\begin{example}
\laprint{exmplsigmaofg}{\rm
\begin{enumerate} 
\ssubitem a
The straight forward generalization of Example \refcom{exmplnofg}a
together with $\L:= \{ 0 \}$, $\krmtil_\mu(0):= \emptyset$,
$\level_\mu(a):= 0$ yields a cardinality based $\sigma$-OFG for the
Cantor normal form with basis $\omega$ ($\mu=0$) and
$\Omega_\mu$ ($\mu>0$).
\ssubitem b 
Example \refcom{exmplnofg}b generalizes again to a 
cardinality based $\sigma$-OFG
with $\L:= \calS_l$ as in Example \refcom{exmplordfun}i,
$\krmtil(a):= \k^l(a)$,
$\level(\Sigma'_\mu(\tvec)):= 0$ (where $0 := () \in \L$),
$\level(\psibar_\mu(A)):= (A)$ (only in case $l=0$ we have
to modify $\L$ in order to make it disjoint from $\T_\mu$).
The resulting system can be seen as a further generalization of the
extended Sch{\"u}tte Klammer symbols. Note that, whereas
for $n$-OS for all terms $t= \psibar(A) \in \T'$
it holds $t \in \T$ (in the fixed point free version), this is no
longer the case here.
\ssubitem c 
If $f_\sigma^l$ is defined similarly and
we apply a similar restriction as in Example \refcom{exmplnofg}c,
then   we get $\eval_\mu(\psibar_\mu(a)) = 
\psi_{\Omega_{\mu+1}} (f_\sigma^l(a))$ 
with $\psi_{\Omega_{\mu+1}}$ the usual $\psi$-function
based on $0$, $+$, $\omega^\cdot$.
\end{enumerate} 
}
\end{example}
%
%%%%%%%%%%%%%%%%%%%%%%%%%%%%%%%%%%%%%%%%%%%
%%
%                                         %
%%%%%%%%%%%%%%%%%%%%%%%%%%%%%%%%%%%%%%%%%%%
%
%
%
\begin{remark}
\laprint{remremprimreksyssig}
Assume the  assumptions of the straight forward 
generalization of Lemma \ref{defprimreksys} to cardinality based 
$\sigma$-OFG. Additionally assume sets 
$\Lhat \subseteq \Lhat' \subseteq \Lhat''$  corresponding
to $\L$ such that $\Lhat''$ is primitive recursive, a 
function $f: \Lhat \ar \L$ and primitive recursive 
functions $\krmhattil_\mu: \Lhat'' \ar_\omega 
\T_\mu''$, $\levelhat_\mu: \T_\mu'' \ar \Lhat$ 
corresponding to $\krmtil_\mu$, $\level_\mu$. 
Define for $l,l' \in \L$, $l \prec_\L l' :\Iff f(l) <_\L f(l')$. 
Assume the adaption of
condition \refsub{defprimreksys}b of Lemma \ref{defprimreksys}
for the new structure and
\refsub{defprimreksys}a, \refsub{defprimreksys}c - \refsub{defprimreksys}f
for $\Lhat$, $\krmhattil$ instead of $\T$, $\krmhat$, where appropriate,
and omitting $\NF$.
E.g. the adaption of condition \refsub{defprimreksys}d to
$\L$ reads: If $A_\mu\subseteq \T'_\mu$, $f \restriction A_\mu$ is injective
($\mu < \sigma$),
then $f \restriction \{ t \in \L' \mid \all \mu < \sigma.
\krmhattil_\mu(t) \subseteq A_\mu \} $ is injective. 
Then the conclusion
of this lemma generalized to our setting holds as well
in the weakened version of Remark \ref{remprimreksysn}, and 
additionally $\Lhat$, $\Lhat'$, $\prec_\L$ are primitive recursive.
\end{remark}
%
%
%
{\bf Proof:} As before.\QED 
%
%
%
% Only the 
%determination of $r \prec s$, $r \sim s$ follows as follows:
%We can determine first for all $r \in \T_\nu
%=: \T$, $s \in \T'_\mu$, $\length(r), \length(s) \leq \length(t)$ by
%induction on $\length(r)$ whether
%$\eval(r) \in \C_\mu(s)$ by having:
%if $\nu< \mu$, $\eval(r) \in \C_\mu(s)$.
%If $\nu = \mu$,
%$\eval(r) \in \C_\mu(s) \Iff \eval(r) \preceq \k_{\mu,\mu}(s) 
%\lor r \preceq' s \land \k(r) \subseteq \C_\mu(s)$,
%and, if $\mu < \nu$  $\eval(r) \in \C_\mu(s) \Iff 
%\level(r) < \level(s) \land \k(r) \subseteq \C_\mu(s)$.
%Then $\eval(r) \prec \eval(s) \Iff \eval(r) \in \C_\mu(s)$ 
%for $r \in \Arg_\mu$, $s \in \Arg_\nu$. 
%Now for $t \in \Arg_\mu$ 
%$\NF(t) \Iff \k(t) \subseteq \C_\mu(t)$, if $t \in \Arg_\mu$.
%%%%%%%%%%%%%%%%%%%%%%%%%%%%%%%%%%%%%%%%%%%
%%
%                                         %
%%%%%%%%%%%%%%%%%%%%%%%%%%%%%%%%%%%%%%%%%%%
%
%
%
\begin{theorem}
\laprint{lemexampleossig}
\begin{enumerate} 
\ssubitem a For every OFG in the Example \ref{exmplsigmaofg}
there exists an elementary OS based on it.
\ssubitem b The supremum of the strength of $\sigma$-OS is $|\ID_\sigma|$.
\end{enumerate} 
\end{theorem}
%
%
{\bf Proof:} As for Theorem \ref{lemexampleosn}.\QED 
%
%
%
%
%
%
\bibliography{/home/setzer/tex/ordsys/submitted/litbank}
\end{document}


%%% Local Variables: 
%%% mode: latex
%%% TeX-dvi-file: "ordsys100498b"
%%% End: 
	
